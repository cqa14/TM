% \usepackage[usenames,dvipsnames]{pstricks}
% \usepackage{pstricks-add}
% \usepackage{epsfig}
% \usepackage{pst-grad} % For gradients
% \usepackage{pst-plot} % For axes
% \usepackage[space]{grffile} % For spaces in paths
% \usepackage{etoolbox} % For spaces in paths
% \makeatletter % For spaces in paths
% \patchcmd\Gread@eps{\@inputcheck#1 }{\@inputcheck"#1"\relax}{}{}
% \makeatother
% 
\begin{figure}[H]
\centering
\psscalebox{1.0 1.0} % Change this value to rescale the drawing.
{
\begin{pspicture}(0,-3.563024)(12.19,2.783024)
\psbezier[linecolor=black, linewidth=0.04](1.16,-2.1842432)(1.524922,-2.8041809)(1.7449551,-1.9486585)(1.9850701,-1.9678869862684736)(2.2251852,-1.9871155)(2.3291042,-2.7156715)(2.7057009,-2.3612618)(3.0822976,-2.006852)(4.5437365,0.30429232)(5.63,0.35302398)
\psbezier[linecolor=black, linewidth=0.04](1.16,-2.1842432)(1.6818448,-1.5846838)(1.6872189,-2.3899596)(1.9746262,-2.4301023071794474)(2.2620335,-2.4702451)(2.3189242,-1.9682242)(2.6012616,-1.9875557)(2.883599,-2.0068872)(4.5341597,-4.9381127)(10.167336,-2.676976)
\psframe[linecolor=black, linewidth=0.04, dimen=outer](6.6,0.973024)(5.6,-0.026976014)
\psframe[linecolor=black, linewidth=0.04, dimen=outer](5.94,1.253024)(5.78,0.95302397)
\psframe[linecolor=black, linewidth=0.04, dimen=outer](6.37,1.263024)(6.22,0.95302397)
\psbezier[linecolor=black, linewidth=0.04](6.58,0.393024)(7.6,0.613024)(8.0,-2.146976)(10.22,-2.1469760131835938)
\psframe[linecolor=black, linewidth=0.04, dimen=outer](11.28,-1.846976)(10.16,-2.966976)
\psarc[linecolor=black, linewidth=0.04, dimen=outer](10.745,-2.601976){0.385}{0.0}{176.2686}
\psline[linecolor=black, linewidth=0.04](10.98,-2.066976)(10.72,-2.546976)
\psline[linecolor=black, linewidth=0.04](0.57010525,-2.2577128)(0.38,-2.6107655)
\psline[linecolor=black, linewidth=0.04](0.57010525,-2.2667656)(0.74210525,-2.646976)
\psline[linecolor=black, linewidth=0.04](0.5610526,-2.3029761)(0.59726316,-1.5063444)
\psline[linecolor=black, linewidth=0.04](0.57010525,-1.7417128)(0.39810526,-1.9680287)
\psline[linecolor=black, linewidth=0.04](0.5882105,-1.7417128)(0.7602105,-2.004239)
\pscircle[linecolor=black, linewidth=0.04, dimen=outer](0.5995263,-1.300397){0.21047369}
\psline[linecolor=black, linewidth=0.04](6.8990116,1.560524)(6.7,1.190931)
\psline[linecolor=black, linewidth=0.04](6.8990116,1.5510472)(7.0790696,1.153024)
\psline[linecolor=black, linewidth=0.04](6.889535,1.5131403)(6.927442,2.3470938)
\psline[linecolor=black, linewidth=0.04](6.8990116,2.1006985)(6.7189536,1.8637798)
\psline[linecolor=black, linewidth=0.04](6.917965,2.1006985)(7.0980234,1.8258728)
\pscircle[linecolor=black, linewidth=0.04, dimen=outer](6.929811,2.5626898){0.2203343}
\psline[linecolor=black, linewidth=0.04](11.950106,-2.5977128)(11.76,-2.9507654)
\psline[linecolor=black, linewidth=0.04](11.950106,-2.6067655)(12.122106,-2.986976)
\psline[linecolor=black, linewidth=0.04](11.941052,-2.642976)(11.977263,-1.8463445)
\psline[linecolor=black, linewidth=0.04](11.950106,-2.081713)(11.778106,-2.3080287)
\psline[linecolor=black, linewidth=0.04](11.96821,-2.081713)(12.14021,-2.3442392)
\pscircle[linecolor=black, linewidth=0.04, dimen=outer](11.9795265,-1.6403971){0.21047369}
\rput[bl](-0.4,-3.126976){Charlie crée l'intrication}
\rput[bl](2.4,1.313024){Alice encode son message}
\rput[bl](9.66,-3.466976){Bob décode le message}
\end{pspicture}
}
\caption{Schéma du principe de superdense coding}
\label{fig:superdense-coding}
\end{figure}

