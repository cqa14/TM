% Abstract
\selectlanguage{english}
\begin{abstract}
\selectlanguage{french}

    L'ordinateur quantique, un terme qui peut terrifier certains qui voient cela comme
    de la magie noire, et en rebuter d'autres qui y voient un jouet pour les
    chercheurs.
    Néanmoins, il représente pour la plupart des personnes qui s'y intéressent
    un espoir énorme, une révolution dans le monde de l'informatique qui permettrait
    de résoudre des problèmes qui sont actuellement trop complexes pour nos
    ordinateurs classiques.\\
    Même s'ils ne les remplaceront pas, les ordinateurs quantiques vont offrir
    une accélération dans le calcul de certaines tâches, et permettre de résoudre
    des problèmes qui sont actuellement trop gros pour nos outils actuels.
    Ils pourraient aussi réaliser ces calculs avec moins d'énergie que ne le demande
    un ordinateur classique.\\ \\
    Dans ce travail, nous développerons les bases de l'informatique quantique,
    par les concepts théoriques nécessaires à la compréhension de ce domaine,
    mais également par la présentation de quelques algorithmes quantiques
    qui montrent un avantage par rapport à leurs équivalents classiques, en passant
    par des résultats qui ont été obtenus sur des ordinateurs quantiques réels et
    les enjeux dans le futur de cette technologie.\\ \\
    Nous avons essayé de démontrer la beauté de ce domaine, de part son ingéniosité,
    mais aussi des concepts mathématiques qui sont derrière, et qui sont souvent
    méconnus du grand public, mais surtout très intéressants et beau dans un sens.
    Néanmoins, quand ceux-ci sont trop complexes ou rébarbatifs, nous avons essayé
    de faire comprendre par des aspects intuitifs le fonctionnement de ces algorithmes,
    en essayant de les mettre en exemples avec des applications concrètes, dans la mesure
    du possible.
\end{abstract}
\selectlanguage{french}
\clearpage