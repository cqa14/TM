\chapter{Sur des machines à court terme}\label{ch:sur-des-machines-a-court-terme}

Quelques machines à court terme voir même actuelles permettent de faire tourner
des circuits ayant des intérêts pratiques. \\ \\
Par exemple de nombreuses simulations de phénomènes physiques, que ce soit dans le
tout petit comme le partenariat entre le CERN et IBM~\cite{ibm-cern} dans l'étude des particules
subatomiques via un ordinateur quantique ou dans le tout grand comme la
simulation d'un trou de ver réalisée par les ordinateurs de Google~\cite{wormhole-sim}.\\
Ces applications vont se multiplier dans les années à venir, et de plus gros
ordinateurs quantiques pourront être utilisés pour simuler des molécules pour
la pharmaceutique ou la chimie, mais également des matériaux pour l'industrie.\\ \\
Un des plus gros espoirs mis dans la technologie quantique est l'intelligence
artificielle.
En effet, les ordinateurs quantiques pourraient permettre d'entrainer des
modèles de machine learning de manière plus efficaces et donc de surpasser les
problèmes qui vont se poser bientôt avec les ordinateurs classiques, qui deviennent
limités de par leur taille pour continuer à faire des intelligences artificielles plus complexes.\\
Cela est à plus long terme, mais dans un futur proche, on peut faire travailler
des ordinateurs quantiques et classiques ensemble pour faire de l'intelligence
artificielle~\cite{ia-mixte}.
Malgré les problèmes de décohérence, les ordinateurs quantiques peuvent être
utilisés à cette fin, le bruit créé pouvant être même exploité pour modifier
les poids d'un réseau de neurones.\\ \\
Finalement, il s'agit également de déterminer la classe de complexité des ordinateurs
quantiques~\cite{wiki:complexity-qc}, afin de déterminer dans quelle mesure ils sont plus
puissants et permettre une meilleure compréhension de leurs capacités.\\ \\
De plus, on peut noter de nombreuses applications des technologies quantiques au-delà du domaine
calculatoire.
On a évoqué la transmission d'information, mais on peut également citer la génération de nombres
aléatoires par exemple.
D'autres applications peuvent paraitre plus étonnantes, comme une méthode de mesure de la
gravité utilisée ensuite pour détecter l'activité volcanique~\cite{AntoniMicollier2022}.
