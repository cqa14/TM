\chapter{Technologies de hardware}\label{ch:technologies-de-hardware}

Les ordinateurs quantiques sont des machines qui sont encore en développement.
De fait, de nombreuses technologies sont en compétition pour devenir la technologie dominante.\\ \\
Quelques critères ont été établi pour la technologie idéale :
\begin{itemize}
    \item Possibilité de l'agrandir physiquement afin d'augmenter le nombre de qubits
    \item Les qubits doivent être initialisables à l'état que l'on souhaite
    \item Les portes doivent être plus rapides que le temps de décohérence
    \item Le groupe de portes doit être universel
    \item Les qubits doivent être lisibles facilement
\end{itemize}
La technologie privilégiée actuellement est celle des qubits supraconducteurs,
comme celle utilisée par IBM et Google.\\
Cependant, d'autres technologies sont étudiées, comme les photons, sur lesquels on peut
utiliser au choix leur polarisation, leur présence ou leur absence, ou encore leur temps d'arrivée.
On peut aussi utiliser le spin des particules, ou encore comme dans le cas des supraconducteurs,
la charge, le courant ou l'énergie d'une jonction Josephson.\\
Chaque technologie présente des avantages et des inconvénients, ainsi que des défis à relever.
Par exemple, les qubits supraconducteurs doivent être contrôlés par l'extérieur, donc il faut par
exemple appliqué une tension précise à température ambiante, puis refroidir le tout à des températures
proches du zéro absolu, sans pour autant détruire les propriétés de la tension appliquée afin que le
qubit reste dans l'état voulu.
Les qubits supraconducteurs fonctionnant autour de 20 $[mK]$, il faut dans la procédure décrite au-dessus
refroidir de 300 $[K]$ à 20 $[mK]$, ce qui est un défi en soi.
Il y a donc été developpé un système fonctionnant autour de 4 $[K]$ qui permet de faciliter tout le
processus~\cite{cryo-cmos}.\\
On peut également citer les qubits de spin, qui eux présentent l'avantage d'avoir une meilleure
cohérence, mais qui sont plus difficiles à construire.
Ils demeurent assez miniaturisables, et sont contrôlables par des champs magnétiques, ce qui est
une différence majeure avec les qubits supraconducteurs.
Finalement, leurs conditions de fonctionnement sont plus faciles à atteindre, car ils fonctionnent
à des températures plus élevées par exemple.\\
Quelle que soit la technologie choisie, il y aura toujours la recherche de la réduction des erreurs,
car même si les qubits actuels sont fidèles à 99.9\%, cela n'est pas suffisant pour faire des
calculs complexes, parce que les erreurs s'accumulent et peuvent rendre le résultat faux.
De fait, plusieurs méthodes de correction d'erreurs sont étudiées, comme la répétition de qubits,
ou différents codes correcteurs d'erreurs~\cite{wiki:quantum-error-correction}.\\ \\
La recherche se fait autour de ces technologies, mais aussi autour de la manière de
stocker les qubits, puisqu'il n'existe pas encore de moyen de stocker un qubit de manière
stable, comme les RAM pour les bits classiques, ou encore les disques durs.
C'est probablement la plus grande difficulté à laquelle les chercheurs sont confrontés,
du fait de la décohérence des qubits, qui a déjà des effets sur les temps courts durant
l'exécution d'un programme.\\
Néanmoins, une qRAM est fondamentalement moins importante qu'une RAM classique, car
les qubits stockent eux-mêmes l'information, et on peut la déplacer sur un autre via
la téléportation quantique, ce qui permet d'avancer que le problème réside dans la
capacité de maintien du qubit dans un état stable sur une longue durée.\\ \\
Comme mentionné également dans la distribution de clés, l'envoi de qubits sur de longues
distances est également étudié et quelques solutions ont été trouvées, comme l'envoi
de photons sur fibre optique.
Cette partie est une des plus développées actuellement, et certaines entreprises
le proposent déjà en tant que produit commercial~\cite{idquantique}.
De plus, on est aussi capable de transmettre des photons intriqués par satellite,
ce qui a permis de faire une transmission sur plus de 1200 $[km]$~\cite{Liao2017}.\\ \\
D'autres domaines de recherches sont aussi utiles à ce domaine, comme la recherche
en matériaux supraconducteurs, car plus ceux-ci sont performants, plus les qubits
pourront être améliorés.