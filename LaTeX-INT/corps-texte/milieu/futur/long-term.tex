\chapter{Sur le long terme}\label{ch:sur-le-long-terme}

Un ordinateur quantique a des intérêts évidents pour la cryptographie, mais il
permet également via des algorithmes comme celui de Grover d'accélérer des problèmes
d'optimisation avec des contraintes, comme des trajets de livraison ou des
problèmes de planification.\\ \\
Il y a bien sûr toutes les applications citées dans le chapitre précédent, mais
poussée à l'extrême, comme une intelligence artificielle entraînée sur un
ordinateur quantique uniquement.\\
Citons également qu'un canal de communication quantique permettrait de
communiquer de manière plus sûre, par exemple via la distribution de clés
ou le superdense coding qui empêcherait l'interception d'un message complet.\\
Si l'on arrivait aussi à créer un stockage quantique, on pourrait enregistrer
nos données dessus et y accéder plus rapidement via l'algorithme de Grover par
exemple.\\ \\
Finalement, cela peut être utilisé par exemple pour du calcul pur de fonctions
mathématiques, en exploitant par exemple l'apparition intrinsèque de
nombres complexes dans les portes quantiques.
Même les opérations les plus simples comme la multiplication de deux nombres
pourraient être accélérées, car les méthodes les plus efficaces classiquement utilisent
des transformées de Fourier rapides, or les ordinateurs quantiques sont les
transformées de Fourier quantiques sont bien plus naturelles et efficaces que
ce que l'on peut faire avec un ordinateur classique~\cite{mult-quant}.
On peut également citer la résolution de systèmes d'équations linéaires, qui
est un problème fondamental en mathématiques et en physique dès que l'on
veut répondre à des questions de modélisation.
Cela peut être amélioré par l'utilisation d'ordinateurs quantiques, car cela
se résume à la résolution d'une équation d'algèbre linéaire de type $Ax = b$
où $A$ est une matrice carrée, $x$ est un vecteur inconnu et $b$ est un vecteur
connu.
Or, on a vu que les ordinateurs quantiques fonctionne d'un point de vue mathématique
sur ce système d'algèbre linéaire, et cela parait donc naturel que les problèmes
de ce type soient plus faciles à résoudre sur un ordinateur quantique~\cite{Zhang2022}.\\ \\
Notons que le développement des ordinateurs quantiques et l'avenir qui leur est
promis dépend beaucoup du spécialiste que l'on interroge.
Par exemple, en introduction, nous citions qu'un qubit parfait pourrait être réalisé
à partir de quelques milliers de qubits physiques, mais IBM par exemple citait un article
pendant le \href{https://qidis23.b2match.io}{Quantum Industry Day in Switzerland 2023} qui
estimait la possibilité d'atteindre un qubit logique à partir de 50 qubits physiques.
En discutant avec Edoardo Charbon, professeur à l'EPFL, il a éclairci ce point en disant
que cela dépendait de la qualité recherchée pour le qubit logique, et qu'il serait probablement
encore sujet à erreurs.
De son point de vue, il ne sait pas encore exactement où cela va mener, que ce soit du côté d'une
technologie qui serait dominante ou pas, et dans quel domaine cela se fera précisément.
On peut mettre cela en perspective avec le point de vue d'Alain Aspect, déjà présenté
dans le chapitre~\ref{ch:notions-theoriques}, qui lors de sa lecture du 2 octobre 2023 à l'EPFL,
voyait déjà une utilité purement scientifique à la recherche sur les ordinateurs quantiques,
car cela permet de voir jusqu'où la théorie quantique est vraie, et de la tester.
