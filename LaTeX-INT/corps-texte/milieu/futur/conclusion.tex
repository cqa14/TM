\chapter{Conclusion}\label{ch:conclusion}

Pour conclure, la technologie de l'ordinateur quantique est encore en
développement, mais son avenir est prometteur.
Ils pourront révolutionner le monde de l'informatique, et permettre une
accélération de la recherche dans de nombreux domaines.
Tout en se basant sur les principes assez complexes de la mécanique quantique,
une fois ceux-ci acceptés, il est possible d'envisager la puissance de calcul
qu'ils permettent, en faisant presque du calcul sur plusieurs valeurs en même
temps, à condition de savoir extraire l'information que l'on veut de l'état
quantique.
Les ordinateurs quantiques sont encore en développement, mais on les voit déjà
accessibles pour ceux intéressés, avec un grand enjeu d'apprentissage
autour de ce sujet~\cite{qiskit-anal}, que ce soit les algorithmes et autres applications, ou
la réalisation autour du hardware.
C'est également une technologie qui peut compléter les ordinateurs classiques,
en permettant d'accélérer un processus particulier.
C'est donc un domaine d'avenir, avec de nombreuses promesses qui commencent à
se concrétiser, comme la réalisation d'ordinateurs à plusieurs milliers de
qubits qui laissent à espérer un avantage majeur dans les décennies à venir.
