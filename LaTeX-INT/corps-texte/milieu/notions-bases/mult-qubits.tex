\chapter{Système à plusieurs qubits}\label{ch:systeme-a-plusieurs-qubits}

Une fois que l'on a compris le fonctionnement d'un qubit, il est pratique,
sinon indispensable, de pouvoir travailler avec plusieurs qubits.
En effet, pour le moment, nous ne pouvions rien faire de bien utile outre
créer des résultats pseudo-aléatoires.
La manipulation de plusieurs qubits permet de créer des états d'intrication,
qui sont la base de la puissance des ordinateurs quantiques, et qui permettent
en quelque sorte de faire des calculs en parallèle.

\section{Description du système}\label{sec:description-du-systeme}

Un système à plusieurs qubits est décrit similairement à un système à un seul
qubit.
Considérons deux qubits, $\ket{\psi} = \psi_1 \ket{0} + \psi_2 \ket{1}$ et
$\ket{\phi} = \phi_1 \ket{0} + \phi_2 \ket{1}$.
Le système est décrit par le produit tensoriel des deux qubits, soit
$\ket{\psi} \otimes \ket{\phi} = \psi_1 \phi_1 \ket{00} + \psi_1 \phi_2 \ket{01}
+ \psi_2 \phi_1 \ket{10} + \psi_2 \phi_2 \ket{11} = \ket{\psi \phi}$.
Cela revient donc à considérer un seul qubit à 4 états, $\ket{00}$, $\ket{01}$,
$\ket{10}$ et $\ket{11}$, et les coefficients sont obtenus en multipliant les
coefficients des deux qubits.
Les probabilités sont obtenues en prenant le module au carré des coefficients, comme
précédemment.\\
La logique est la même pour un système à $n$ qubits, que l'on considère comme un seul
vecteur à $2^n$ états.
Cette augmentation du nombre d'états est la raison pour laquelle les ordinateurs
quantiques sont théoriquement intéressants, et de plus la complexité des simulations
classiques, car elle augmente exponentiellement avec le nombre de qubits.\\ \\
Le formalisme de Dirac permet de mieux comprendre la notion d'intrication.
On parle d'intrication lorsque l'on ne peut pas décrire un système par le produit
de qubits individuels.
Par exemple, considérons le système $\ket{\psi} = \frac{1}{\sqrt{2}} \ket{00} +
\frac{1}{\sqrt{2}} \ket{11}$.
En reprenant l'égalité vue précédemment, $\ket{\psi} \otimes \ket{\phi} = \psi_1 \phi_1 \ket{00} + \psi_1 \phi_2 \ket{01}
+ \psi_2 \phi_1 \ket{10} + \psi_2 \phi_2 \ket{11}$, essayons de trouver les coefficients
$\psi_1$, $\psi_2$, $\phi_1$ et $\phi_2$.
En comparant les facteurs, on obtient un système d'équations :
\[\begin{cases}
  \ket{00} \ : \ \psi_1 \phi_1 &= \frac{1}{\sqrt{2}} \\
  \ket{01} \ : \ \psi_1 \phi_2 &= 0 \\
  \ket{10} \ : \ \psi_2 \phi_1 &= 0 \\
  \ket{11} \ : \ \psi_2 \phi_2 &= \frac{1}{\sqrt{2}}
\end{cases}\]
On voit que $\psi_1$ ou $\phi_1$ devrait être nul pour qu’il soit possible de définir les
états respectifs des deux sous-systèmes, or si l'on prend les autres équations, la
première implique que $\psi_1$ est non nul, et la dernière que $\phi_1$ est non nul.
On observe donc que le système n'est pas descriptible par le produit de deux qubits
individuels, et on dit qu'il est intriqué.\\ \\
Le produit tensoriel s'applique aussi aux matrices, et on peut donc définir similairement
l'application d'une porte sur un système à plusieurs qubits.
Par exemple si l'on souhaite appliquer une porte d'Hadamard sur le premier qubit d'un
système à deux qubits, on obtient une porte globale sur le système :
\[H \otimes I = \frac{1}{\sqrt{2}} \begin{pmatrix}
  1 & 1 \\
  1 & -1
\end{pmatrix} \otimes \begin{pmatrix}
    1 & 0 \\
    0 & 1
\end{pmatrix} = \frac{1}{\sqrt{2}} \begin{pmatrix}
    1 & 0 & 1 & 0 \\
    0 & 1 & 0 & 1 \\
    1 & 0 & -1 & 0 \\
    0 & 1 & 0 & -1
\end{pmatrix}\]
que l'on peut appliquer sur le système afin de trouver le nouveau vecteur d'état.

\section{Les portes}\label{sec:les-portes}

Mais l'intérêt des systèmes à plusieurs qubits n'est pas de faire ce que l'on sait déjà
faire avec un seul qubit, mais de pouvoir faire des choses nouvelles.\\ \\
La première porte que l'on va voir est la porte de CX (Controlled X, parfois aussi CNOT pour
Controlled NOT), qui est une porte à deux qubits.
Elle va prendre en entrée deux qubits, un qubit de contrôle et un qubit cible.
Si le qubit de contrôle est à 0, la porte ne fait rien, sinon elle applique une porte
X sur le qubit cible.
Elle est représentée par la matrice suivante :
\[CX = \begin{pmatrix}
  1 & 0 & 0 & 0 \\
  0 & 1 & 0 & 0 \\
  0 & 0 & 0 & 1 \\
  0 & 0 & 1 & 0
\end{pmatrix}\]
et on peut la représenter sur un circuit comme suit :
\begin{figure}[H]
    \[\shorthandoff{!}
    \scalebox{1.0}{
        \Qcircuit @C=1.0em @R=0.8em @!R { \\
        \lstick{\ket{c}} & \ctrl{1} & \qw & \qw\\
        \lstick{\ket{t}} & \targ & \qw & \qw\\
        \\ }}
    \]
    \caption{Circuit d'une porte CX}
    \label{fig:cnot}
\end{figure}
avec $\ket{c}$ le qubit de contrôle et $\ket{t}$ le qubit cible.
C'est une porte très importante, car elle permet de créer de l'intrication, de plus elle
agit comme une porte XOR entre les deux qubits, avec le deuxième qubit comme sortie, soit
symboliquement $\ket{c, t} \rightarrow \ket{c, c \oplus t}$ (le qubit cible va être modifié comme s'il
subissait une porte XOR entre lui et le qubit de contrôle).\\ \\
L'autre porte sur deux qubits que l'on va le plus utiliser est la porte CZ (Controlled Z).
Elle est similaire à la porte CX, mais au lieu d'appliquer une porte X sur le qubit cible,
elle applique une porte Z.
Ce qui est particulier avec cette porte, c'est qu'elle agit sur tout le système, et pas
seulement sur le qubit cible.
En effet, si le qubit de contrôle est à $\ket{0}$, la porte ne fait rien, sinon elle applique une
porte Z sur le qubit cible, ce qui revient à multiplier le vecteur d'état par $-1$.
Or, si le qubit cible est à $\ket{0}$, on a $\ket{0} \rightarrow -\ket{0} = \ket{0}$ et donc
l'état du système ne change pas.
En résumé, cette porte inverse la phase globale du système si et seulement si les deux
qubits sont à $\ket{1}$.
C'est pour cela qu'on la dessine comme suit :
\begin{figure}[H]
    \[\shorthandoff{!}
    \scalebox{1.0}{
        \Qcircuit @C=1.0em @R=1em @!R { \\
        \lstick{\ket{\psi_0} } & \ctrl{1} & \qw & \qw\\
        \lstick{\ket{\psi_1} } & \control\qw & \qw & \qw\\
        \\ }}
    \]
    \caption{Circuit d'une porte CZ}
    \label{fig:cz}
\end{figure}
et cela est également visible sur sa matrice :
\[CZ = \begin{pmatrix}
  1 & 0 & 0 & 0 \\
  0 & 1 & 0 & 0 \\
  0 & 0 & 1 & 0 \\
  0 & 0 & 0 & -1
\end{pmatrix}\]
Finalement, un paterne apparait et on peut noter la représentation générale d'une porte
controlée $U$ comme suit :
\[CU = \begin{pmatrix}
  1 & 0 & 0 & 0 \\
  0 & 1 & 0 & 0 \\
  0 & 0 & u_{11} & u_{12} \\
  0 & 0 & u_{21} & u_{22}
\end{pmatrix}\]
où $U = \begin{pmatrix}
  u_{11} & u_{12} \\
  u_{21} & u_{22}
\end{pmatrix}$ est la matrice de la porte $U$.\\
Cette porte générale donne aussi une idée de comment sont construites les matrices des
portes sur plusieurs qubits.
En effet, on remarque que les lignes correspond à l'effet de la porte sur les états
de base $\ket{00}$, $\ket{01}$, $\ket{10}$ et $\ket{11}$, et les colonnes, ce en quoi
la porte transforme les états de base.
Par exemple, sur la dernière matrice présentée, on que les deux premières lignes ont juste
des 1 dans la colonne correspondante, ce qui indique que pour les états $\ket{00}$ et
$\ket{01}$, la porte ne fait rien.
Sur les deux dernières lignes, on a des éléments uniquement sur les deux dernières colonnes,
ce qui indique que le premier qubit demeure à 1, et que le deuxième subit une porte $U$.\\ \\
Notons encore une porte sur trois qubits, la porte CCX (Controlled Controlled X), ou
porte de Toffoli.
Le principe est le même que pour la porte CX, mais avec deux qubits de contrôle, et
la porte X est appliquée sur le qubit cible si et seulement si les deux qubits de contrôle
sont à 1.
\begin{figure}[H]
    \[\shorthandoff{!}
    \scalebox{1.0}{
        \Qcircuit @C=1.0em @R=0.8em @!R { \\
        \lstick{\ket{c_0}} & \ctrl{1} & \qw & \qw\\
        \lstick{\ket{c_1}} & \ctrl{1} & \qw & \qw\\
        \lstick{\ket{t}} & \targ & \qw & \qw\\
        \\ }}
    \]
    \caption{Circuit d'une porte de Toffoli}
    \label{fig:toffoli}
\end{figure}
De nombreuses autres portes existent, mais elles ne valent pas la peine d'être mentionnées
ici, car elles ne seront pas utilisées dans ce travail, ou de manière assez succincte
pour être expliquées au fur et à mesure.

\section{Applications simples}\label{sec:applications-simples}

Ci-après deux exemples de circuits visant à illustrer certains avantages de l'informatique
quantique.
D'un côté, le protocole de superdense coding permet d'envoyer deux bits d'information
en n'envoyant qu'un seul qubit, et de l'autre, le protocole de quantum teleportation
qui transfère l'état d'un qubit sans transférer le qubit lui-même, en utilisant
deux bits classiques.
Ceux-ci permettent de saisir l'avantage intrinsèque de l'informatique quantique, dans
ces exemples où un qubit correspond à deux bits classiques, mais aussi de voir comment
l'intrication peut être utilisée pour transmettre de l'information.\\ \\
Mais avant de commencer, il faut savoir comment créer un état intriqué, comme décrit
dans la section~\ref{sec:description-du-systeme}.
On va expliquer la création d'un état intriqué dit équilibré, soit avec la même
probabilité d'être dans les différents états intriqués, pour un système de deux qubits
un tel état est nommé un état de Bell.
Restons sur le système de deux qubits, sur lequel on compte quatre états de Bell :
\begin{align*}
  \ket{\Phi^+} &= \frac{1}{\sqrt{2}}(\ket{00} + \ket{11})\\
  \ket{\Phi^-} &= \frac{1}{\sqrt{2}}(\ket{00} - \ket{11})\\
  \ket{\Psi^+} &= \frac{1}{\sqrt{2}}(\ket{01} + \ket{10})\\
  \ket{\Psi^-} &= \frac{1}{\sqrt{2}}(\ket{01} - \ket{10})
\end{align*}
On va montrer comment créer l'état $\ket{\Phi^+}$, et les autres états sont créés
de manière similaire.\\
Le circuit est le suivant :
\begin{figure}[H]
    \[\shorthandoff{!}
    \scalebox{1.0}{
        \Qcircuit @C=1.0em @R=0.2em @!R { \\
        \lstick{\ket{0}} & \gate{\mathrm{H}} & \ctrl{1} & \qw & \qw\\
        \lstick{\ket{0}} & \qw & \targ & \qw & \qw\\
        \\ }}
    \]
    \caption{Circuit de création de l'état $\ket{\Phi^+}$}
    \label{fig:phiplus}
\end{figure}
qui dont on se rend compte assez intuitivement qu'il a l'effet escompté.
En effet, on applique une porte de Hadamard sur le premier qubit, ce qui le met dans
un état superposé, et ensuite, on applique une porte $CX$, qui va mettre le deuxième
qubit dans le même état que le premier, et donc les deux qubits sont dans le même
état, et sont intriqués.
La porte de Hadamard fait changer le système $\ket{00} \rightarrow
\frac{\ket{0} + \ket{1}}{\sqrt{2}}\ket{0}$ puis la porte $CX$ applique comme une porte
$XOR$, donc on obtient $\frac{\ket{0} + \ket{1}}{\sqrt{2}}\ket{0} \rightarrow
\frac{\ket{0, 0 \oplus 0} + \ket{1, 1 \oplus 0}}{\sqrt{2}} = \frac{\ket{00} +
\ket{11}}{\sqrt{2}}$. \\ \\
On peut maintenant passer aux deux exemples en gardant en tête ce qui a été dit
précédemment.

\subsection{Superdense Coding}\label{subsec:superdense-coding}

Le principe du superdense coding~\cite{wiki:superdense-coding} est d'envoyer deux bits d'information en n'envoyant
qu'un seul qubit.
On va donc considérer trois personnes, Alice qui veut envoyer un message à Bob, et
Charlie qui va créer l'état intriqué.
Une fois son travail terminé, Charlie envoie un qubit à Alice, et un à Bob.
Alice va pouvoir ensuite encoder son message dans le qubit qu'elle a reçu, selon ce que
l'on va voir, et ensuite, elle va envoyer son qubit à Bob.
Bob va ensuite pouvoir décoder le message en appliquant les portes adéquates sur la
paire de qubits qu'il a reçu.
\import{images/notions-bases/quantique/sup-cod/}{sup-schem.tex}
Le circuit de Charlie est celui de la figure~\ref{fig:phiplus}, afin de créer un état
intriqué $\ket{\phi^+}$ duquel il envoie un qubit à Alice, et un à Bob.
\begin{figure}[H]
    \[\shorthandoff{!}
    \scalebox{1.0}{
        \Qcircuit @C=1.0em @R=0.2em @!R { \\
        \nghost{{q}_{A} :  } & \lstick{{q}_{A} :  } & \gate{\mathrm{H}} & \ctrl{1} & \qw & \qw\\
        \nghost{{q}_{B} :  } & \lstick{{q}_{B} :  } & \qw & \targ & \qw & \qw\\
        \\ }}
    \]
    \caption{Circuit de Charlie}
    \label{fig:charlie-sup}
\end{figure}
De l'autre côté, celui de Bob va être l'inverse de celui de Charlie, afin de décoder
le message.
Cela est possible, car comme dit plus tôt, les portes sont réversibles.
Ensuite, il va pouvoir mesurer afin de récupérer les deux bits d'information.
\begin{figure}[H]
    \[\shorthandoff{!}
    \scalebox{1.0}{
        \Qcircuit @C=1.0em @R=0.2em @!R { \\
        \nghost{{q}_{A} :  } & \lstick{{q}_{A} :  } & \ctrl{1} & \gate{\mathrm{H}} \barrier[0em]{1} & \qw & \meter & \qw & \qw & \qw\\
        \nghost{{q}_{B} :  } & \lstick{{q}_{B} :  } & \targ & \qw & \qw & \qw & \meter & \qw & \qw\\
        \nghost{\mathrm{{meas} :  }} & \lstick{\mathrm{{meas} :  }} & \lstick{/_{_{2}}} \cw & \cw & \cw & \dstick{_{_{\hspace{0.0em}0}}} \cw \ar @{<=} [-2,0] & \dstick{_{_{\hspace{0.0em}1}}} \cw \ar @{<=} [-1,0] & \cw & \cw\\
        \\ }}
    \]
    \caption{Circuit de Bob}
    \label{fig:bob-sup}
\end{figure}
Il reste à montrer comment Alice encode son message.
Alice reçoit le qubit $q_A$.
Sur le circuit tel qu'on la construit, Alice va influencer du point de vue mathématique
le qubit à droite (donc initialement on a $\ket{00} = \ket{0}_B \otimes \ket{0}_A$).
Après le passage dans la partie de Charlie, le circuit est dans l'état $\ket{\Phi^+} =
\frac{\ket{00} + \ket{11}}{\sqrt{2}}$.
Pour encoder son message, Alice à alors plusieurs possibilités :
\begin{itemize}
    \item Si elle veut envoyer le message $00$, elle ne fait rien, et le circuit reste
        dans l'état $\ket{\Phi^+}$.
    \item Similairement à l'ordinateur classique, elle peut inverser son qubit via une
        porte $X$, et le circuit passe dans l'état $\frac{\ket{01} + \ket{10}}{\sqrt{2}}$.
        Avec la mesure de Bob, on obtient alors $10$.
    \item L'avantage quantique est dans les deux derniers, où Alice peut encoder de
        l'information dans la phase du qubit.
        En effet, via la porte $Z$ elle peut l'inverser.
        Ainsi le circuit passe dans l'état $\frac{\ket{00} - \ket{11}}{\sqrt{2}}$ qui
        va donner $01$ à Bob.
    \item Enfin, elle peut inverser le qubit et inverser la phase, ce qui donne
        $\frac{\ket{01} - \ket{10}}{\sqrt{2}}$ et donc $11$ à Bob.
\end{itemize}
On peut vérifier les affirmations sur ce que recevra Bob en faisant le calcul.
Par exemple, $\frac{\ket{01} - \ket{10}}{\sqrt{2}} \rightarrow \frac{\ket{01} -
\ket{11}}{\sqrt{2}}$ après la $CX$, et finalement $\frac{\ket{01} - \ket{11}}{\sqrt{2}}
\rightarrow \frac{1}{\sqrt{2}}
\begin{pmatrix}
    1 & 0 & 1 & 0 \\
    0 & 1 & 0 & 1 \\
    1 & 0 & -1 & 0 \\
    0 & 1 & 0 & -1
\end{pmatrix} \cdot \frac{1}{\sqrt{2}}
\begin{pmatrix}
    0 \\
    1 \\
    0 \\
    -1
\end{pmatrix} =
\begin{pmatrix}
    0 \\
    0 \\
    0 \\
    1
\end{pmatrix} = \ket{11}$ après la $H$.
Il en va de même pour les autres cas. \\ \\
On peut résumer toutes les étapes en un circuit.
\begin{figure}[H]
    \centering
    \[\shorthandoff{!}
    \scalebox{1.0}{
        \Qcircuit @C=1.0em @R=0.2em @!R { \\
        \nghost{{q}_{A} :  } & \lstick{{q}_{A} :  } & \gate{\mathrm{H}} & \ctrl{1} \barrier[0em]{1} & \qw & \gate{\mathrm{I}} \barrier[0em]{1} & \qw & \ctrl{1} & \gate{\mathrm{H}} \barrier[0em]{1} & \qw & \meter & \qw & \qw & \qw\\
        \nghost{{q}_{B} :  } & \lstick{{q}_{B} :  } & \qw & \targ & \qw & \qw & \qw & \targ & \qw & \qw & \qw & \meter & \qw & \qw\\
        \nghost{\mathrm{{00} :  }} & \lstick{\mathrm{{00} :  }} & \lstick{/_{_{2}}} \cw & \cw & \cw & \cw & \cw & \cw & \cw & \cw & \dstick{_{_{\hspace{0.0em}0}}} \cw \ar @{<=} [-2,0] & \dstick{_{_{\hspace{0.0em}1}}} \cw \ar @{<=} [-1,0] & \cw & \cw\\
        \\ }}
    \]
    \[\shorthandoff{!}
    \scalebox{1.0}{
        \Qcircuit @C=1.0em @R=0.2em @!R { \\
        \nghost{{q}_{A} :  } & \lstick{{q}_{A} :  } & \gate{\mathrm{H}} & \ctrl{1} \barrier[0em]{1} & \qw & \gate{\mathrm{X}} \barrier[0em]{1} & \qw & \ctrl{1} & \gate{\mathrm{H}} \barrier[0em]{1} & \qw & \meter & \qw & \qw & \qw\\
        \nghost{{q}_{B} :  } & \lstick{{q}_{B} :  } & \qw & \targ & \qw & \qw & \qw & \targ & \qw & \qw & \qw & \meter & \qw & \qw\\
        \nghost{\mathrm{{10} :  }} & \lstick{\mathrm{{10} :  }} & \lstick{/_{_{2}}} \cw & \cw & \cw & \cw & \cw & \cw & \cw & \cw & \dstick{_{_{\hspace{0.0em}0}}} \cw \ar @{<=} [-2,0] & \dstick{_{_{\hspace{0.0em}1}}} \cw \ar @{<=} [-1,0] & \cw & \cw\\
        \\ }}
    \]
    \[\shorthandoff{!}
    \scalebox{1.0}{
        \Qcircuit @C=1.0em @R=0.2em @!R { \\
        \nghost{{q}_{A} :  } & \lstick{{q}_{A} :  } & \gate{\mathrm{H}} & \ctrl{1} \barrier[0em]{1} & \qw & \gate{\mathrm{Z}} \barrier[0em]{1} & \qw & \ctrl{1} & \gate{\mathrm{H}} \barrier[0em]{1} & \qw & \meter & \qw & \qw & \qw\\
        \nghost{{q}_{B} :  } & \lstick{{q}_{B} :  } & \qw & \targ & \qw & \qw & \qw & \targ & \qw & \qw & \qw & \meter & \qw & \qw\\
        \nghost{\mathrm{{01} :  }} & \lstick{\mathrm{{01} :  }} & \lstick{/_{_{2}}} \cw & \cw & \cw & \cw & \cw & \cw & \cw & \cw & \dstick{_{_{\hspace{0.0em}0}}} \cw \ar @{<=} [-2,0] & \dstick{_{_{\hspace{0.0em}1}}} \cw \ar @{<=} [-1,0] & \cw & \cw\\
        \\ }}
    \]
    \[\shorthandoff{!}
    \scalebox{1.0}{
        \Qcircuit @C=1.0em @R=0.2em @!R { \\
        \nghost{{q}_{A} :  } & \lstick{{q}_{A} :  } & \gate{\mathrm{H}} & \ctrl{1} \barrier[0em]{1} & \qw & \gate{\mathrm{Z}} & \gate{\mathrm{X}} \barrier[0em]{1} & \qw & \ctrl{1} & \gate{\mathrm{H}} \barrier[0em]{1} & \qw & \meter & \qw & \qw & \qw\\
        \nghost{{q}_{B} :  } & \lstick{{q}_{B} :  } & \qw & \targ & \qw & \qw & \qw & \qw & \targ & \qw & \qw & \qw & \meter & \qw & \qw\\
        \nghost{\mathrm{{11} :  }} & \lstick{\mathrm{{11} :  }} & \lstick{/_{_{2}}} \cw & \cw & \cw & \cw & \cw & \cw & \cw & \cw & \cw & \dstick{_{_{\hspace{0.0em}0}}} \cw \ar @{<=} [-2,0] & \dstick{_{_{\hspace{0.0em}1}}} \cw \ar @{<=} [-1,0] & \cw & \cw\\
        \\ }}
    \]
    \caption{Circuit complet : Charlie | Alice | Bob (circuit | mesure)}
    \label{fig:complete-circuit}
\end{figure}
Testons maintenant l'un de ces circuits avec un vrai ordinateur quantique.\\ \\
Comme vu avec le calcul, on est censé obtenir exactement la bonne valeur à chaque
fois.
Pourtant, les résultats obtenus ne semblent pas tout à fait d'accord avec la théorie.
Ci-suit l'histogramme des résultats obtenus avec le circuit qui code le message
01 (voir figure~\ref{fig:complete-circuit}).
\begin{figure}[H]
    \centering
    \import{images/notions-bases/quantique/sup-cod/}{sup-cod-his.tex}
    \caption{Histogramme des résultats obtenus avec le circuit qui code le message 01 \protect\footnotemark}
    \label{fig:histogramme-01}
\end{figure}
\footnotetext{Executé le 29.08.2023 sur la machine 'ibm\_quito', \textit{job id : cjmovmnijvusg3savs30}}
On se rend compte qu'il y a des résultats incorrects.
Parmi les 8192 essais, 7739 sont corrects, ce qui fait un taux de réussite de
94.5\%.
Les autres forment le bruit, qui sont les erreurs qui arrivent lors
du processus.
Cela illustre très bien le problème des ordinateurs quantiques actuels : ils
sont très sensibles aux erreurs.
Nous parlerons par la suite des causes et voie de correction de ces erreurs.

\subsection{Téléportation quantique}\label{subsec:quantum-teleportation}

Il existe en physique quantique un théorème de non-clonage~\cite{wiki:non-cloning} qui dit qu'il est
impossible de copier à l'identique un état quantique inconnu.
Notons que deux états intriqués ne peuvent pas être considérés comme des copies
de part leur mesure qui est identique.
\begin{proof}
    Soient $A$ et $B$ deux systèmes quantiques, avec $A$ dans un état
    $\ket{a}_A$ que nous souhaitons cloner et $B$ dans un état $\ket{b}_B$
    quelconque.
    Le système total est donc dans l'état $\ket{a}_A \otimes \ket{b}_B =
    \ket{a}_A \ket{b}_B$.\\
    On ne peut pas copier l'état $\ket{a}_A$ en le mesurant, car cela détruirait
    une partie de l'information.\\
    On ne peut donc qu'agir sur son opérateur d'évolution $U$ (les portes quantiques
    sont de tels opérateurs).
    On cherche $U$ tel que $U\ket{a}_A \ket{b}_B = \ket{a}_A \ket{a}_B$.
    Cela implique également que pour un état quelconque $\ket{\alpha}_A$ de $A$,
    $U\ket{\alpha}_A \ket{b}_B = \ket{\alpha}_A \ket{\alpha}_B$.\\ \\
    Rappelons que les opérateurs d'évolution sont unitaires, donc $U^\dagger U =
    I$, et que $\ket{\psi}^\dagger = \bra{\psi}$, ainsi que $M^\dagger N^\dagger
    = (NM)^\dagger$.
    Étudions donc le produit scalaire de $\ket{a}_A \ket{b}_B$ avec $\ket{\alpha}_A
    \ket{b}_B$ :
    \[
        \begin{split}
            \bra{b}_B \bra{a}_A \ket{\alpha}_A \ket{b}_B &=
            \bra{b}_B \bra{a}_A U^\dagger U \ket{\alpha}_A \ket{b}_B
            = (U \ket{a}_A \ket{b}_B)^\dagger (U \ket{\alpha}_A \ket{b}_B) \\
            &= (\ket{a}_A \ket{a}_B)^\dagger (\ket{\alpha}_A \ket{\alpha}_B)
            = \bra{a}_B \bra{a}_A \ket{\alpha}_A \ket{\alpha}_B
        \end{split}
    \]
    Donc $\bra{b}_B \bra{a}_A \ket{\alpha}_A \ket{b}_B = \bra{a}_B \bra{a}_A \ket{\alpha}_A
    \ket{\alpha}_B \Leftrightarrow \braket{a}{\alpha} = \braket{a}{\alpha}^2$, ce qui
    est possible que si $\braket{a}{\alpha} = 0$ ou $1$, donc que ces états sont
    identiques ou orthogonaux.\\ \\
    On a donc une contradiction avec l'hypothèse de départ supposant ces états
    quelconques.
    On a donc montré qu'il est impossible de cloner un état quantique inconnu par
    l'absurde.
\end{proof}
Ce théorème pose différents problèmes, notamment pour créer des méthodes de
suppression d'erreurs, car on ne peut pas copier le système pour le comparer
avec l'original.\\
Mais si la copie est impossible, on peut transmettre un état quantique en utilisant
un état intriqué et des bits classiques~\cite{wiki:quantum-teleportation}.\\ \\
Prenons à nouveau nos trois personnages : Alice, Bob et Charlie.
Charlie va toujours créer un état intriqué, et envoyer un qubit à Alice et un
à Bob.
Alice va ensuite appliquer des opérations sur son qubit et envoyer le résultat
à Bob, via un canal classique de communication, puis Bob va appliquer des opérations sur son
qubit afin de retrouver ce qu'Alice lui a envoyé.
\import{images/notions-bases/quantique/quant-tp/}{quant-tp-schem.tex}
Voyons donc concrètement comment cela fonctionne.
Le circuit à décortiquer est le suivant :
\begin{figure}[H]
    \[\shorthandoff{!}
    \scalebox{1.0}{
        \Qcircuit @C=1.0em @R=0.2em @!R { \\
        \nghost{\ket{\psi}} & \lstick{\ket{\psi}} & \qw & \qw \barrier[0em]{2} & \qw & \ctrl{1} & \gate{\mathrm{H}} \barrier[0em]{2} & \qw & \meter & \qw \barrier[0em]{2} & \qw & \qw & \qw & \qw & \qw\\
        \nghost{{q}_{A} :  } & \lstick{{q}_{A} :  } & \gate{\mathrm{H}} & \ctrl{1} & \qw & \targ & \qw & \qw & \qw & \meter & \qw & \qw & \qw & \qw & \qw\\
        \nghost{{q}_{B} :  } & \lstick{{q}_{B} :  } & \qw & \targ & \qw & \qw & \qw & \qw & \qw & \qw & \qw & \gate{\mathrm{X}} & \gate{\mathrm{Z}} & \qw & \qw\\
        \nghost{\mathrm{{crz} :  }} & \lstick{\mathrm{{crz} :  }} & \lstick{/_{_{1}}} \cw & \cw & \cw & \cw & \cw & \cw & \dstick{_{_{\hspace{0.0em}0}}} \cw \ar @{<=} [-3,0] & \cw & \cw & \cw & \control \cw^(0.0){^{\mathtt{0x1}}} \cwx[-1] & \cw & \cw\\
        \nghost{\mathrm{{crx} :  }} & \lstick{\mathrm{{crx} :  }} & \lstick{/_{_{1}}} \cw & \cw & \cw & \cw & \cw & \cw & \cw & \dstick{_{_{\hspace{0.0em}0}}} \cw \ar @{<=} [-3,0] & \cw & \control \cw^(0.0){^{\mathtt{0x1}}} \cwx[-2] & \cw & \cw & \cw\\
        \\ }}
    \]
    \caption{Circuit complet : Charlie | Alice (opération | mesure) | Bob}
    \label{fig:quant-tp-circuit}
\end{figure}
Alice veut transmettre l'état $\ket{\psi} = \alpha \ket{0} + \beta \ket{1}$ à
Bob.
Charlie va donc créer un état intriqué $\ket{e} = \frac{1}{\sqrt{2}}
(\ket{00} + \ket{11})$, puis transmettre le premier qubit $q_A$ à Alice et le
second $q_B$ à Bob.
Gardons en tête donc que le qubit le plus à droite est celui de Bob, et celui
le plus à gauche celui d'Alice $\ket{e} = \frac{1}{\sqrt{2}} (\ket{0}_A \ket{0}_B
+ \ket{1}_A \ket{1}_B)$. \\
Avec celui $\ket{\psi}$ que l'on veut transmettre, on a donc un système de
trois qubits : $\ket{\psi} \otimes \ket{e} = \frac{1}{\sqrt{2}} (\alpha \ket{0}
\otimes (\ket{00} + \ket{11}) + \beta \ket{1} \otimes (\ket{00} + \ket{11})) =
\frac{1}{\sqrt{2}} (\alpha \ket{000} + \alpha \ket{011} + \beta \ket{100} +
\beta \ket{111})$.\\
En appliquant le porte $CX$ avec le qubit de gauche comme contrôle et celui du
milieu comme cible, on obtient : $ \frac{1}{\sqrt{2}} (\alpha \ket{000} + \alpha
\ket{011} + \beta \ket{100} + \beta \ket{111}) \rightarrow \frac{1}{\sqrt{2}}
(\alpha \ket{000} + \alpha \ket{011} + \beta \ket{110} + \beta \ket{101})$.\\
En appliquant la porte $H$ sur le qubit de gauche, le facteur devant devient
$\frac{1}{\sqrt{2}} \cdot \frac{1}{\sqrt{2}} = \frac{1}{2}$, et les termes
$\ket{0ij} \rightarrow \ket{0ij} + \ket{1ij}$, et $\ket{1ij} \rightarrow
\ket{0ij} - \ket{1ij}$, donc on obtient : $\frac{1}{2} (\alpha (\ket{000} +
\ket{011} + \ket{100} + \ket{111}) + \beta (\ket{010} + \ket{010} - \ket{110} -
\ket{101}))$.\\
Ceci peut être réécrit de la manière suivante :
\[
    \begin{split}
        \frac{1}{2}( \quad & \ket{00}(\alpha \ket{0} + \beta \ket{1}) \\
        + & \ket{01}(\alpha \ket{1} + \beta \ket{0}) \\
        + & \ket{10}(\alpha \ket{0} - \beta \ket{1}) \\
        + & \ket{11}(\alpha \ket{1} - \beta \ket{0}) \quad )
    \end{split}
\]
Cela permet de voir que la mesure d'Alice va projeter le qubit de Bob dans l'un
des quatre états suivants :
\begin{itemize}
    \item $\ket{00} \rightarrow \alpha \ket{0} + \beta \ket{1}$
    \item $\ket{01} \rightarrow \alpha \ket{1} + \beta \ket{0}$
    \item $\ket{10} \rightarrow \alpha \ket{0} - \beta \ket{1}$
    \item $\ket{11} \rightarrow \alpha \ket{1} - \beta \ket{0}$
\end{itemize}
et donc Bob n'a plus qu'à appliquer les portes $X$ et $Z$ selon les bits reçus
pour retrouver l'état $\ket{\psi}$.\\
Alice mesure le qubit de droite sur le registre $crx$, et donc la porte $X$ est
appliquée uniquement si le bit est à 1, ce qui va inverser les 0 et les 1.
De même pour le signe, mais avec celui de gauche et le registre $crz$, ainsi
qu'avec la porte $Z$.\\ \\
On peut également le faire sans mesure si l'on souhaite mettre un autre qubit dans
l'état désiré en appliquant des portes $CX$ et $CZ$ depuis les bits que l'on
mesurait précédemment.
\begin{figure}[H]
    \[\shorthandoff{!}
    \scalebox{1.0}{
        \Qcircuit @C=1.0em @R=0.2em @!R { \\
        \nghost{{q}_{0} :  } & \lstick{{q}_{0} :  } & \qw & \qw \barrier[0em]{2} & \qw & \ctrl{1} & \gate{\mathrm{H}} \barrier[0em]{2} & \qw & \ctrl{2} & \qw & \qw & \qw\\
        \nghost{{q}_{1} :  } & \lstick{{q}_{1} :  } & \gate{\mathrm{H}} & \ctrl{1} & \qw & \targ & \qw & \qw & \qw & \ctrl{1} & \qw & \qw\\
        \nghost{{q}_{2} :  } & \lstick{{q}_{2} :  } & \qw & \targ & \qw & \qw & \qw & \qw & \targ & \control\qw & \qw & \qw\\
        \\ }}
    \]
    \caption{Téléportation quantique sans mesure}
    \label{fig:quant-tp-circuit-sans-mesure}
\end{figure}
Ce circuit a un intérêt de démonstration de la valeur d'un qubit question donnée,
mais n'est pas très intéressant à utiliser en tant que tel, c'est pour cela qu'on
le désigne comme un protocole.\\ \\
En résumé, on a vu deux circuits mettant en évidence des avantages et complications
liés à la nature quantique des qubits.
En plus de donner une idée de l'importance des concepts de superposition et
d'intrication, ces circuits permettent de voir que les qubits peuvent transporter
plus d'un bit d'information par qubit en quelque sorte.
