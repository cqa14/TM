\chapter{Un qubit}\label{ch:un-qubit}

L'informatique quantique est basée, similairement à l'informatique classique, sur un système pouvant prendre deux états,
dénommés 0 et 1, que l'on nomme un qubit.
La différence clé est que, dans le cas quantique, le système n'est pas décrit par la logique booléenne, mais suit les
lois de la mécanique quantique.
On a déjà vu un système à deux états, le spin dans la section~\ref{sec:physique}.
Comme dans ce cas, un qubit n’est pas dans un état bien défini par rapport à une observable donnée, mais peut être dans une superposition d'états,
soit et dans l'état 0 et celui 1, chacun avec une probabilité bien définie.

\section{Superposition d'états}\label{sec:superposition-d'etats}

Afin de décrire cette superposition du système, on utilise la notation de Dirac, ou notation bra-ket, qui est une
notation mathématique permettant de décrire les états d'un système quantique.
Nous n'introduirons ici que les principes utiles à la compréhension de l'informatique quantique.
Comme présenté préalablement un système quantique peut être dans un état qui n’est pas défini par rapport
à une observable donnée.
Par exemple dans le cas d’un spin, si pour cet état on mesure le spin dans la direction $z$ (définie par un
système d’axes cartésiens) on observera soit le résultat +1/2 avec une probabilité $\abs{\lambda_0}^2$ et, après la mesure,
le spin sera dans la direction $z$ notée 0, soit le résultat -1/2 avec une probabilité $\abs{\lambda_1}^2$ et après la mesure le
spin sera dans la direction $-z$ notée 1.
On dira alors qu’avant la mesure le spin était dans un état de superposition de 0 et 1 avec un coefficient
$\lambda_0$ et $\lambda_1$.
On va alors mettre ces coefficients dans un vecteur colonne, et nommé celui-ci un ket, noté $\ket{\psi}$ :
\[
    \ket{\psi} = \begin{pmatrix} \lambda_0 \\ \lambda_1 \end{pmatrix}
\]
Rappelons que les coefficients sont complexes, et que la somme des amplitudes doit être égale à 1, où l'amplitude est
le carré du coefficient, soit $\abs{\lambda_0}^2 + \abs{\lambda_1}^2 = 1$.\\ \\
Posons maintenant les états de base $\ket{0}$ et $\ket{1}$, qui sont les états dans lesquels le système peut être
déterministiquement, soit $\ket{0} = \begin{pmatrix} 1 \\ 0 \end{pmatrix}$ et $\ket{1} = \begin{pmatrix} 0 \\ 1
\end{pmatrix}$.
Ainsi, tout qubit d'état $\ket{\psi}$ peut être écrit comme une combinaison linéaire des états de base :
\[
    \ket{\psi} = \lambda_0 \ket{0} + \lambda_1 \ket{1}
\]
On définit également le bra, noté $\bra{\psi}$, comme le conjugué transposé du ket $\ket{\psi}$ :
\[
    \bra{\psi} = \begin{pmatrix} \lambda_0^* & \lambda_1^* \end{pmatrix}
\]
ce qui permet de définir le produit scalaire entre deux états $\ket{u}$ et $\ket{v}$ comme $\bra{u}\ket{v}$.\\
Via cette opération, on observe que l'on peut obtenir mathématiquement les coefficients d'un état en le multipliant
par un état de base :
\begin{gather*}
    \bra{0}\ket{\psi} = \lambda_0 \bra{0}\ket{0} + \lambda_1 \bra{0}\ket{1} = \lambda_0\\
    \bra{1}\ket{\psi} = \lambda_0 \bra{1}\ket{0} + \lambda_1 \bra{1}\ket{1} = \lambda_1
\end{gather*}
car $\bra{0}\ket{0} = 1$ comme les états de base sont normalisés, et $\bra{0}\ket{1} = 0$ car les états de base sont
orthogonaux.\\ \\
Notons que l'on pourrait choisir d'autres états de base, par exemple $\ket{+} = \frac{1}{\sqrt{2}} \ket{0} +
\frac{1}{\sqrt{2}} \ket{1}$ et $\ket{-} = \frac{1}{\sqrt{2}} \ket{0} - \frac{1}{\sqrt{2}} \ket{1}$, qui sont
orthogonaux, et donc forment une base ($\bra{+}\ket{-} = \left(\frac{1}{\sqrt{2}} \bra{0} + \frac{1}{\sqrt{2}} \bra{1}\right)
\left(\frac{1}{\sqrt{2}} \ket{0} - \frac{1}{\sqrt{2}} \ket{1}\right) = \frac{1}{2} \bra{0}\ket{0} - \frac{1}{2} \bra{0}\ket{1} +
\frac{1}{2} \bra{1}\ket{0} - \frac{1}{2} \bra{1}\ket{1} = 0$).
Cette notion de base sera plus détaillée ultérieurement, ainsi que la dénomination $\ket{+}$ et $\ket{-}$.

\section{Opérations sur un qubit}\label{sec:operations-sur-un-qubit}

Similairement à l'informatique classique, on peut effectuer des opérations sur un qubit, que l'on nomme des portes~\cite{wiki:qantum-gates}.
Ces portes décrivent l'évolution du système, et similairement aux qubits, on peut les décrire via la notation de Dirac,
mais également via de l'algèbre linéaire, comme des matrices unitaires, comme les coefficients d'un état sont
multipliés par un nombre complexe de module 1, et donc la norme du vecteur reste la même.
Cette section est aussi l'occasion de présenter les diagrammes de circuits quantiques, qui sont une représentation
graphique des opérations effectuées sur un qubit.\\ \\
La plus simple est comme toujours l'identité, qui ne fait rien, et est représentée par la matrice identité :
\[
    I = \begin{pmatrix} 1 & 0 \\ 0 & 1 \end{pmatrix}
\]
et sont effet sur un qubit est de le laisser inchangé, soit $\ket{\psi} = I \ket{\psi}$, et se calcule via le produit matriciel :
\[
    \begin{pmatrix} 1 & 0 \\ 0 & 1 \end{pmatrix} \begin{pmatrix} \lambda_0 \\ \lambda_1 \end{pmatrix} =
    \begin{pmatrix} \lambda_0 \\ \lambda_1 \end{pmatrix}
\]
qui revient schématiquement à juste avoir un qubit :

\begin{figure}[H]
    \[\shorthandoff{!}
    \scalebox{1.0}{
        \Qcircuit @C=1.0em @R=1.0em @!R { \\
        \lstick{\ket{\psi}} & \qw & \qw\\
        \\ }}
    \]
    \caption{Circuit quantique de l'identité}
    \label{fig:identite}
\end{figure}

pour lequel on ne dessine aucune porte, car l'état ne change pas.
Notons aussi que la porte peut être décrite par la notation de Dirac, soit $I = \ket{0}\bra{0} + \ket{1}\bra{1}$, car
$I \ket{\psi} = (\ket{0}\bra{0} + \ket{1}\bra{1}) \ket{\psi} = \ket{0} \bra{0}\ket{\psi} + \ket{1}
\bra{1}\ket{\psi} = \ket{0} \lambda_0 + \ket{1} \lambda_1 = \ket{\psi}$.\\ \\
On va passer ensuite aux portes dites de Pauli~\cite{wiki:pauli-matrices}, qui sont les portes de base de l'informatique quantique, et qui sont
les matrices de Pauli.
Celles-ci servent à la description du spin d'une particule, un système quantique à deux états, et il y en a trois,
une pour chaque axe de l'espace, soit $X$, $Y$ et $Z$.\\
La première est la porte $X$, qui est la plus simple car assimilable à la porte NOT de l'informatique classique, et
est représentée par la matrice suivante :
\[
    X = \begin{pmatrix} 0 & 1 \\ 1 & 0 \end{pmatrix}
\]
ou encore $X = \ket{0}\bra{1} + \ket{1}\bra{0}$, et son effet sur un qubit est de permuter les coefficients de
probabilité, soit $X \ket{\psi} = \ket{1}\bra{0} \ket{\psi} + \ket{0}\bra{1} \ket{\psi} = \ket{1} \lambda_0 +
\ket{0} \lambda_1 $ qui est bien l'inverse du qubit de départ.
On représente cette porte par le circuit quantique suivant :
\begin{figure}[H]
    \[\shorthandoff{!}
    \scalebox{1.0}{
        \Qcircuit @C=1.0em @R=0.2em @!R { \\
        \lstick{\ket{\psi}} & \gate{\mathrm{X}} & \qw & \qw\\
        \\ }}
    \]
    \caption{Circuit de la porte $X$}
    \label{fig:x-gate}
\end{figure}
La porte $Z$ parait un peu plus étrange, car elle change le signe, dit de phase, du coefficient de l'état 1,
soit $Z = \begin{pmatrix} 1 & 0 \\ 0 & -1 \end{pmatrix}$, et son effet sur un qubit est $Z \ket{\psi} =
(\ket{0}\bra{0} - \ket{1}\bra{1}) \ket{\psi} = \ket{0} \lambda_0 - \ket{1} \lambda_1 $.\\
La troisième porte est la porte $Y$, qui semble structurellement à un mélange des deux précédentes, avec une
unité imaginaire, soit $Y = \begin{pmatrix} 0 & -i \\ i & 0 \end{pmatrix}$, et son effet sur un qubit est
$Y \ket{\psi} = (i \ket{0}\bra{1}-i \ket{1}\bra{0}) \ket{\psi} = i \ket{0} \lambda_1 -i \ket{1} \lambda_0$.\\ \\
Définissons la notion de vecteur propre afin d'expliquer une des notions de visualisation de l'informatique quantique.
Un vecteur propre est un vecteur qui, lorsqu'il est multiplié par une matrice, donne un vecteur colinéaire au vecteur
de départ.
Posant une matrice $M$ et un vecteur $\vec{v}$, on a $M \vec{v} = \lambda \vec{v}$, avec $\lambda$ un scalaire.\\
Notons que les différentes portes de Pauli ont des vecteurs propres, pour $X$, on a $\ket{+} = \frac{1}{\sqrt{2}}
(\ket{0} + \ket{1})$ et $\ket{-} = \frac{1}{\sqrt{2}} (\ket{0} - \ket{1})$, pour $Y$, on a $\ket{\circlearrowleft} = \frac{1}{\sqrt{2}}
(\ket{0} + i \ket{1})$ et $\ket{\circlearrowright} = \frac{1}{\sqrt{2}} (\ket{0} - i \ket{1})$, et pour $Z$, on a $\ket{0}$ et
$\ket{1}$.
\begin{proof}
    Montons que $\ket{+}$ est un vecteur propre de $X$.
    On a $X \ket{+} = (\ket{0}\bra{1} + \ket{1}\bra{0}) \frac{1}{\sqrt{2}} (\ket{0} + \ket{1}) = \frac{1}{\sqrt{2}}
    (\ket{0}\bra{1}\ket{0} + \ket{0}\bra{1}\ket{1} + \ket{1}\bra{0}\ket{0} + \ket{1}\bra{0}\ket{1}) =
    \frac{1}{\sqrt{2}} (\ket{0} + \ket{1}) = \ket{+}$, donc $\ket{+}$ est bien un vecteur propre de $X$.
    On peut faire de même pour les autres vecteurs propres.
\end{proof}
Afin de visualiser l'état d'un qubit, on peut utiliser une sphère de Bloch~\cite{wiki:bloch-sphere}, qui est une sphère unitaire, et qui
est dans un espaces à trois dimensions, avec les axes $x$, $y$ et $z$ qui correspondent aux vecteurs propres de
$X$, $Y$ et $Z$ respectivement.
\begin{figure}[H]
    \centering
    \begin{blochsphere}[radius=2.5 cm,tilt=15,rotation=-30,opacity=0]
        \drawBallGrid[style={opacity=0.1}]{30}{180}

        \drawStatePolar[axisarrow = true, statewidth = 0.3]{x-Achse}{90}{90}
        \drawStatePolar[axisarrow=true, statewidth = 0.3]{y-Achse}{90}{0}
        \drawStatePolar[axisarrow = true, statewidth = 0.3]{z-Achse}{0}{0}

        \node[below] at (x-Achse) {\fontsize{0.15cm}{1em} \large $x$};
        \node[below] at (y-Achse) {\fontsize{0.15cm}{1em} \large $y$};
        \node[below] at (z-Achse) {\fontsize{0.15cm}{1em} \large $z$};

        \drawStatePolar[statecolor = red]{State}{150}{150}
        \labelLatLon{rightstate}{-20}{90};
        \node[text=red, below=0.7] at (rightstate) { \Large$\ket{\psi}$};

        \labelLatLon{up}{90}{0};
        \labelLatLon{down}{-90}{90};
        \node[above] at (up) {$\ket{0}$};
        \node[below] at (down) {$\ket{1}$};
    \end{blochsphere}
    \caption{Sphère de Bloch}
    \label{fig:bloch-sphere}
\end{figure}
L'effet des portes de Pauli sur la sphère de Bloch est de faire une symétrie par rapport à l'axe correspondant
à la porte.
Cette intuition par symétrie et rotation est très utile pour visualiser les portes quantiques.\\ \\
La suivante est justement la porte de Hadamard, qui est définie par $H = \frac{1}{\sqrt{2}} \begin{pmatrix} 1 & 1 \\ 1 & -1 \end{pmatrix}$,
et son effet sur la sphère de Bloch est de faire une rotation de $180^\circ$ autour de l'axe $y$ et de $90^\circ$
autour de l'axe $x$.
Cela à un effet intéressant sur les états de base, car $\ket{0}$ est envoyé sur $\ket{+}$ et $\ket{1}$ est envoyé
sur $\ket{-}$.
\begin{proof}
    On a $H \ket{0} = \frac{1}{\sqrt{2}} ((\ket{0}+\ket{1})\bra{0} + (\ket{0}-\ket{1})\bra{1}) \ket{0} =
    \frac{1}{\sqrt{2}} ((\ket{0}+\ket{1})\bra{0}\ket{0} + (\ket{0}-\ket{1})\bra{1}\ket{0}) =
    \frac{1}{\sqrt{2}} (\ket{0}+\ket{1} + 0) = \frac{1}{\sqrt{2}} \ket{0} + \frac{1}{\sqrt{2}} \ket{1} =
    \ket{+}$, et un raisonnement similaire pour $\ket{1}$.
\end{proof}
On remarque donc que cette porte permet de créer une superposition équilibrée.
On note aussi que $H^2 = I$, donc $H$ est sa propre inverse.
On étudiera des effets plus concrets de cette porte dans la section~\ref{sec:mesure}.\\ \\
Finalement, on évoquera rapidement les portes plus générales, les autres noms spécifiques de portes seront définis
si besoin.\\
La première est la porte de phase, qui est définie par $P(\phi) = \begin{pmatrix} 1 & 0 \\ 0 & e^{i \phi} \end{pmatrix}$.
Elle a pour effet de changer la phase du qubit de $\phi$, et est donc une généralisation de la porte $Z$ en générant
une rotation autour de l'axe $z$.\\
La seconde est la porte de rotation autour d'un axe sur la sphère de Bloch, qui est définie par
\[ R_x = \begin{pmatrix} \cos \frac{\theta}{2} & -i \sin \frac{\theta}{2} \\ -i \sin \frac{\theta}{2} & \cos \frac{\theta}{2} \end{pmatrix} \]
\[ R_y = \begin{pmatrix} \cos \frac{\theta}{2} & -\sin \frac{\theta}{2} \\ \sin \frac{\theta}{2} & \cos \frac{\theta}{2} \end{pmatrix} \]
\[ R_z = \begin{pmatrix} e^{-i \frac{\theta}{2}} & 0 \\ 0 & e^{i \frac{\theta}{2}} \end{pmatrix} \]
qui ont pour effet de faire une rotation de $\theta$ autour de l'axe $x$, $y$ et $z$ respectivement
(le travail de trigonométrie est laissé au lecteur, car l'exponentielle revient à cela par la formule d'Euler
$e^{i \theta} = \cos \theta + i \sin \theta$).\\ \\
Notons qu'un qubit est par définition une combinaison linéaire $\ket{\psi} = \alpha \ket{0} + \beta \ket{1}$,
ce qui sans perte de généralité peut être écrit comme $\ket{\psi} = \cos \frac{\theta}{2} \ket{0} + e^{i \phi} \sin \frac{\theta}{2} \ket{1}$.
Cela revient au même que de passer des coordonnées cartésiennes aux coordonnées sphériques, comme on peut le voir
via la sphère de Bloch.\\
La porte générale sur un qubit est donc définie par des angles, qui ressemble à une combinaison de rotation autour
des axes $x$, $y$ et $z$ :
\[
    U(\theta, \phi, \lambda) = \begin{pmatrix} \cos \frac{\theta}{2} & -e^{i \lambda} \sin \frac{\theta}{2} \\ e^{i \phi} \sin \frac{\theta}{2} & e^{i (\phi + \lambda)} \cos \frac{\theta}{2} \end{pmatrix}
\]
mais l'on se rend compte qu'il y a trois paramètres alors que plus haut, on a vu qu'il fallait deux angles pour
définir un qubit.
Néanmoins, on peut y donner une intuition, en appliquant la porte $U$ à un qubit $\ket{0}$, on obtient bien
$\cos \frac{\theta}{2} \ket{0} + e^{i \phi} \sin \frac{\theta}{2} \ket{1}$ comme énoncé plus haut, et sur $\ket{1}$,
on obtient $e^{i \lambda}(-\sin \frac{\theta}{2} \ket{1} + e^{i \phi} \cos \frac{\theta}{2} \ket{0})$.
Le facteur devant $\ket{1}$ est $e^{i \lambda}$, qui est une phase, et donc n'a pas d'effet sur la sphère de Bloch,
néanmoins n'étant présent que sur l'état $\ket{1}$, il va avoir un effet dès que l'on va traiter avec un état
superposé.\\ \\
Les portes se représentent par des carrés avec un symbole à l'intérieur, similairement à la porte $X$ dans le
cadre des diagrammes de circuits.
De plus on peut noter que pour toutes les portes, il existe une porte inverse, qui est la porte qui annule l'effet
de la porte, pour une porte générale $U$, sa porte inverse est $U^\dagger$ et on a $U U^\dagger = I$.

\section{Mesure}\label{sec:mesure}

Une fois que le système a subi des transformations, il faut récupérer l'information.
À la différence des opérations précédentes, la mesure est une opération irréversible, et qui va détruire
l'information contenue dans le qubit en vertu de la mécanique quantique.
Elle peut être interprétée comme la projection du vecteur état initial sur le vecteur de la sphère de Bloch
correspondant à la direction de la mesure ce qui fait perdre une partie de l'information.\\
De plus, lorsque l'on mesure un qubit en pratique, on va répéter l'expérience plusieurs fois, et on va selon les
cas obtenir des résultats différents qu'on va pouvoir analyser statistiquement.\\ \\
Prenons l'exemple d'un qubit dans l'état $\ket{+} = \frac{1}{\sqrt{2}} (\ket{0} + \ket{1})$, obtenu par l'application
de la porte $H$ sur le qubit $\ket{0}$.
La théorie de la physique quantique décrit que si on mesure ce qubit, on a une chance sur deux d'obtenir
$\ket{0}$ et une chance sur deux d'obtenir $\ket{1}$ (car les deux ont un coefficient de $\frac{1}{\sqrt{2}}$ qui
donne une probabilité de $(\frac{1}{\sqrt{2}})^2 = \frac{1}{2}$).\\
On enregistre le résultat de la mesure dans un bit classique, et tout le système peut être représenté par
le schéma suivant :
\begin{figure}[H]
    \[\shorthandoff{!}
    \scalebox{1.0}{
        \Qcircuit @C=1.0em @R=0.2em @!R { \\
        \nghost{{q} :  } & \lstick{{q0} :  } & \gate{\mathrm{H}} & \meter & \qw & \qw\\
        \nghost{\mathrm{{c} :  }} & \lstick{\mathrm{{c0} :  }} & \lstick{/_{_{1}}} \cw & \dstick{_{_{\hspace{0.0em}0}}} \cw \ar @{<=} [-1,0] & \cw & \cw\\
        \\ }}
    \]
    \caption{Circuit de mesure d'un qubit dans l'état $\ket{+}$}
    \label{fig:simple-circuit}
\end{figure}
Si on le fait sur une vraie machine quantique, on obtiendra parfois 0, parfois 1, dans la mesure effectuée
dans l'exemple, 0 a été obtenu 3901 fois et 1 a été obtenu 4291 fois, ce qui nous donne une probabilité proche
de ce qu'on attendait.
On peut émettre plusieurs hypothèses sur les raisons de cette différence, comme le fait que la machine quantique
soit sensible à l'environnement et donc pas parfaite, ou que l'échantillon soit trop petit pour avoir une statistique
parfaite.
Cela peut être utile de représenter les résultats obtenus sous forme d'histogramme, surtout lorsque l'on va
effectuer des mesures plus complexes.
\begin{figure}[H]
    \centering
    \import{images/notions-bases/quantique/}{real-mes-supp.tex}
    \caption[Histogramme des résultats de la mesure d'un qubit dans l'état $\ket{+}$]{Histogramme des résultats de la mesure d'un qubit dans l'état $\ket{+}$ \protect\footnotemark}
    \label{fig:measure}
\end{figure}
\footnotetext{Executé le 29.08.2023 sur la machine 'ibm\_quito', \textit{job id : cjmo77kvcjlre5ddj7h0}}
De plus, la base utilisée par convention est la base $\ket{0}$ et $\ket{1}$, soit la base nommée $z$.
Néanmoins, cela peut être intéressant de mesurer dans une autre base, par exemple la base $x$ qui est la base
$\ket{+}$ et $\ket{-}$.
Parfois cette mesure dans une autre base est faite avec l'ordinateur, mais on peut aussi la faire avec des
portes quantiques, par exemple pour mesurer dans la base $x$, on applique la porte $H$ avant de mesurer.\\ \\
Un qubit est donc un système quantique à deux états.
De cela, on peut exploiter ses propriétés pour faire des opérations, telles que la superposition d'états.
Ces opérations sont purement déterministes, et on peut les représenter par des portes quantiques.
Cet état déterministe va être perdu lors de la mesure, qui va donner un résultat aléatoire selon l'état du qubit,
et qui va détruire l'information contenue dans le qubit.
