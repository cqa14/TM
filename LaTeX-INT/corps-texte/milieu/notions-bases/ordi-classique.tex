\chapter{Un ordinateur classique}\label{ch:un-ordinateur-classique}

Avant de partir dans les concepts propres à l'algorithmie quantique, il est utile de présenter les concepts
de base de l'algorithmie classique.
Pour commencer, on va faire un tour des idées théoriques puis de l'implémentation matérielle.

\section{Logique}\label{sec:logique}

Dans le cadre de l'informatique, on utilise la logique pour décrire le comportement d'un ordinateur.
C'est une logique binaire, c'est-à-dire qu'on ne s'intéresse qu'à deux valeurs de vérité : vrai et faux.
On la nomme logique booléenne, du nom de son inventeur, George Boole, et elle est très pratique dans le cadre
de l'informatique, car elle permet de décrire simplement les opérations logiques de base.\\
En effet, les ordinateurs ont été conçus pour manipuler de telles valeurs, parce qu'elles peuvent être représentées
par un concept simple de ``on/off'' (allumé/éteint), qui peut être représenté par un courant électrique.
On utilisera plutôt les valeurs 1 et 0, mais le principe reste le même.\\
On peut alors définir des opérations sur ces valeurs, qui sont les opérations logiques de base.
Dans le cadre informatique, on les appelle les portes logiques~\cite{wiki:logic-gate}, mais on peut également utiliser la notation
mathématique de la logique afin de les décrire, ou encore les tables de vérité, qui décrivent le comportement
de ces opérations sur toutes les valeurs possibles, ou même le langage naturel.\\ \\
Définissons donc les opérations logiques de base, et leur comportement sur les valeurs de vérité.
Tout d'abord, considérons une entrée $A$ et une entrée $B$, qui peuvent prendre les valeurs 0 ou 1, ainsi qu'une
sortie $Q$, qui prendra également les valeurs 0 ou 1.
Tout d'abord prenons une seule entrée.
Il est utile d'avoir une opération d'identité, qui ne change pas la valeur de l'entrée, que l'on nommera
``Buffer'', qui pour une entrée $A$ et une sortie $Q$ se définit par $Q = A$.
On peut également définir une opération de négation, qui inverse la valeur de l'entrée, que l'on nommera
``NOT'', qui pour une entrée $A$ et une sortie $Q$ se définit par $Q = \neg A$.

\import{images/notions-bases/classique/gates/}{one-gate.tex}

Mais on ne peut pas faire grand-chose avec une seule entrée.
Ainsi, faisons de même avec deux, à commencer par la plus simple, qui est l'opération ``AND'', qui donne 1 si
et seulement si les deux entrées sont à 1, et 0 sinon, et qui se définit par $Q = A \land B$.
Le contraire de cette opération est l'opération ``OR'', qui donne 1 si au moins une des deux entrées est à 1,
et 0 sinon, et qui se définit par $Q = A \lor B$.
Pour ces deux portes, on peut également définir leur contraire, qui est l'opération ``NAND'' (NOT AND) et
``NOR'' (NOT OR), qui sont simplement les opérations AND et OR suivies d'une opération NOT\@.
Finalement, on peut définir l'opération ``XOR'' (eXclusive OR), qui donne 1 si et seulement si une et une seule
des deux entrées est à 1, et qui se définit par $Q = A \oplus B$, parfois également considérée comme la somme
par bit.
Similairement à NAND et NOR, on peut définir l'opération ``XNOR'' (eXclusive NOR), qui est simplement l'opération
XOR suivie d'une opération NOT\@.

\import{images/notions-bases/classique/gates/}{two-gates.tex}

À partir de ces opérations de bases, on peut construire tout ce que font les ordinateurs actuels.
En effet, on peut définir des opérations plus complexes, comme l'addition, la soustraction, la multiplication,
la division, etc.
Mais cela demande un peu plus de travail, et on va juste montrer par exemple comment construire une addition
de deux nombres binaires de 3 bits.\\
Afin de sommer deux nombres binaires, on peut les additionner bit à bit, en prenant en compte la retenue.
Par exemple, pour additionner 5 (101) et 3 (011), cela donne :
\begin{center}
    \begin{tabular}{cccc}
      & 1 & 0 & 1 \\
      + & 0 & 1 & 1 \\
      \hline
    \end{tabular}
    $\Rightarrow$
    \begin{tabular}{cccc}
        & & \textit{1} & \\
        & 1 & 0 & 1 \\
        + & 0 & 1 & 1 \\
        \hline
        & & & 0 \\
    \end{tabular}
    $\Rightarrow$
    \begin{tabular}{cccc}
        & \textit{1} & \textit{1} & \\
        & 1 & 0 & 1 \\
        + & 0 & 1 & 1 \\
        \hline
        & & 0 & 0 \\
    \end{tabular}
    $\Rightarrow$
    \begin{tabular}{cccc}
        \textit{1} & \textit{1} & \textit{1} & \\
        & 1 & 0 & 1 \\
        + & 0 & 1 & 1 \\
        \hline
        & 0 & 0 & 0 \\
    \end{tabular}
    $\Rightarrow$
    \begin{tabular}{cccc}
        \textit{1} & \textit{1} & \textit{1} & \\
        & 1 & 0 & 1 \\
        + & 0 & 1 & 1 \\
        \hline
         1 & 0 & 0 & 0 \\
    \end{tabular}
\end{center}
qui donne bien 8 (1000).\\
Alors, pour sommer simplement deux nombres binaires, on peut utiliser une porte XOR, qui donne 1 si et seulement
si une et une seule des deux entrées est à 1, et une porte AND, qui donne 1 si et seulement si les deux entrées
sont à 1.
Cela reviendrait schématiquement à :

\begin{figure}[H]
    \centering
    \begin{tikzpicture}[
    %Environment config
        font=\sffamily,
        thick,
    %Environment styles
        GateCfg/.style={
            logic gate inputs={normal,normal,normal},
            draw,
            scale=1
        }
    ]
        \node (x) at (0,0) {$x_0$};
        \node (y) at (1,0) {$y_0$};
        \node[xor gate US, draw, logic gate inputs={normal,normal},scale=1] (xor) at (2,-1) {};
        \node[and gate US, draw, logic gate inputs={normal,normal},scale=1] (and) at (2,-2) {};
        \draw (x) |- (xor.input 1);
        \draw (y) |- (xor.input 2);
        \draw (x) |- (and.input 1);
        \draw (y) |- (and.input 2);
        \draw (xor.output) -- (3,-1) node[above] {$s_0$};
        \draw (and.output) -- (3,-2) node[above] {$s_1$};
    \end{tikzpicture}
    \caption{Schéma d'une addition de deux bits}
    \label{fig:adder-2bits}
\end{figure}

Mais afin ensuite de pouvoir additionner des nombres de plus de 1 bit, il faut prendre en compte la retenue.
De fait on doit construire un circuit pour sommer trois bits, qui est le même sauf que l'on répète l'opération
avec la première sortie et si l'une donne un reste, on le note, comme suit :

\begin{figure}[H]
    \centering
    \begin{tikzpicture}[
    %Environment config
        font=\sffamily,
        thick,
    %Environment styles
        GateCfg/.style={
            logic gate inputs={normal,normal,normal},
            draw,
            scale=1
        }
    ]
        \node (x) at (0,-1) {$x_0$};
        \node (y) at (1,-1) {$y_0$};
        \node (z) at (2.2,-1) {$z_0$};
        \node[xor gate US, draw, logic gate inputs={normal,normal},scale=1] (xor1) at (2,-6) {};
        \node[and gate US, draw, logic gate inputs={normal,normal},scale=1] (and1) at (1.5,-5) {};
        \draw (x) |- (xor1.input 1);
        \draw (y) |- (xor1.input 2);
        \draw (x) |- (and1.input 1);
        \draw (y) |- (and1.input 2);
        \node[xor gate US, draw, logic gate inputs={normal,normal},scale=1] (xor2) at (4,-4) {};
        \node[and gate US, draw, logic gate inputs={normal,normal},scale=1] (and2) at (3.5,-3) {};
        \draw (z) |- (xor2.input 1);
        \draw (xor1.output) |- (xor2.input 2);
        \draw (z) |- (and2.input 1);
        \draw (xor1.output) |- (and2.input 2);
        \draw (xor2.output) -- (7,-3) node[above] {$s_0$};
        \node[or gate US, draw, logic gate inputs={normal,normal},scale=1] (xor3) at (6,-2) {};
        \draw (and1.output) |- (xor3.input 1);
        \draw (and2.output) |- (xor3.input 2);
        \draw (xor3.output) -- (7,-2) node[above] {$s_1$};
    \end{tikzpicture}
    \caption{Schéma d'une addition de trois bits}
    \label{fig:adder-3bits}
\end{figure}

qui permet bien d'obtenir les quatre sorties de l'addition de trois bits, soit 00, 01, 10 et 11.\\ \\
Une fois ces circuits de base construits, il est possible de les combiner pour obtenir un circuit permettant
d'additionner deux nombres, par exemple de 3 bits chacun.
Pour cela on procède exactement comme lorsque l'on a déconstruit l'addition plus haut, en sommant les bits
de même poids, et en prenant en compte la retenue, via notre circuit de somme de trois bits.

\import{images/notions-bases/classique/circ/}{adder-comp.tex}

On voit bien via cet exemple que même les choses les plus simples doivent être repensées pour être adaptées
aux nouvelles technologies, que ce soit les ordinateurs classiques ou, comme on le verra plus tard, les
ordinateurs quantiques.\\ \\
Un dernier mot sur la notion d'universalité, qui se dit d'un ensemble de portes logiques qui permettent
de construire n'importe quel circuit logique.
En effet, cela revient à dire que matériellement, il suffit de pouvoir construire ces portes pour pouvoir
effectuer les opérations que l'on souhaite.
Dans le cadre des ordinateurs classiques, on peut montrer que les portes NAND et NOR sont universelles, mais on
laissera la démonstration de ce fait pour les plus curieux, car elle n'est pas très compliquée à comprendre, mais
assez longue.

\section{Hardware}\label{sec:hardware}

Comme vu juste avant, les ordinateurs classiques sont composés de portes logiques, qui peuvent toutes être
construites à partir de portes NAND ou NOR.
Afin de donner une idée du fonctionnement d'un ordinateur classique, on va détailler le fonctionnement pratique
d'une implémentation de la porte NAND, qui est la plus simple à construire.

\begin{figure}[H]
    \centering
    \import{images/notions-bases/classique/elec/}{elec-nand.tex}
    \caption{Schéma électronique d'une porte NAND}
    \label{fig:elec-nand}
\end{figure}

On voit sur la figure~\ref{fig:elec-nand} que la porte NAND est composée, de haut en bas, d'une résistance parcourue
par un courant électrique de tension $V_{cc}$, qui mène d'un côté à une sortie $Q$ et de l'autre à deux transistors
alignés avec comme base $A$ et $B$, finalement reliés à la mise à terre.
De manière simplifiée, on peut dire que les transistors sont des interrupteurs qui laissent passer le courant
lorsque la tension à leur base est suffisante.
De plus, le fait de relier le circuit à la mise à terre met le circuit en état bas, soit le 0 logique.
Ainsi, tant que les deux transistors ne sont pas activés, le courant ne peut pas passer, et la sortie est à 1,
et dès que cela passe, ce dernier effet s'applique et la sortie passe à 0.\\ \\
Mais cela n'est pas encore complètement satisfaisant, car il faut encore savoir construire le circuit.
Néanmoins, seuls les transistors sont vraiment compliqués à construire, car les résistances sont relativement
aisées à fabriquer étant une propriété naturelle des matériaux, et les mises à terre sont simplement des
connexions à la terre, tout comme les connexions et la génération de tension qui sont les bases de l'électricité.\\
Pour décrire le fonctionnement d'un transistor, on va utiliser un modèle simplifié, qui est le modèle de
transistor à effet de champ, ou FET (Field effect transistor)~\cite{wiki:mosfet}.

\import{images/notions-bases/classique/elec/}{transistor.tex}

Le principe de fonctionnement est le suivant : lorsque la tension à la base de la grille est suffisante, le courant peut
passer entre le drain et la source, et lorsque ce n'est pas le cas, le courant ne peut pas passer.
La base, ou grille, est composée d'un métal qui est séparé du reste du transistor par une couche d'oxyde,
qui est un isolant.
Ainsi, lorsque la tension dans le métal est suffisante, cela va attirer ou repousser les électrons du substrat
et permettre le passage du courant entre le drain et la source.
Cela est possible car le substrat est un matériau semi-conducteur, qui est un matériau qui peut être isolant
ou conducteur selon la tension qui lui est appliquée, généralement du silicium.
Cette propriété découle de la structure de ce matériau, qui est composé d'atomes de silicium, qui ont 4 électrons
de valence, et qui vont donc former une structure cristalline, où chaque atome est lié à 4 autres.
Cependant, il manque un électron à chaque atome pour former une liaison covalente, et c'est ce qu'on appelle
un trou.
Ainsi, lorsqu'on applique une tension à la base, on va pouvoir déplacer les électrons et les trous, et ainsi
modifier la conductivité du substrat.
C'est une partie de la physique du solide nommée théorie des bandes~\cite{wiki:theo-bandes}, qui ne sera pas détaillée ici, mais mentionnée,
car elle est aussi basée sur la mécanique quantique, ce qui explique l'utilisation de matériaux semi-conducteurs
dans les ordinateurs quantiques.\\ \\
Ainsi, on vient d'avoir un aperçu du fonctionnement d'un ordinateur classique, afin de pouvoir mieux comprendre
les différences avec les ordinateurs quantiques, mais aussi les similarités entre les deux concepts.
Un ordinateur classique est concrètement construit sur un circuit électronique dont les composants de base sont
les transistors, qui sont construits à partir de matériaux semi-conducteurs, propriété qui est basée sur la
mécanique quantique.
Néanmoins, la logique utilisée en est indépendante, et est basée sur la logique booléenne, qui est une logique
binaire, et donc purement déterministe.
