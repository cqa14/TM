\chapter*{Remerciements}

Tout ce travail n'aurait pas été possible sans l'aide de nombreuses personnes que je tiens à remercier ici.
Un grand merci à mon professeur, M. De Montmollin, pour m'avoir donné l'opportunité de réaliser ce travail
de maturité sur un sujet qui m'intéresse et pour m'avoir soutenu tout au long de ce travail, ainsi que de
m'avoir aidé à trouver des contacts dans le domaine.
Je remercie également M. Lévêque, maître d'enseignement et de recherche à l'EPFL, pour la relecture de ce travail.
Je remercie aussi M. Charbon, professeur à l'EPFL et directeur du laboratoire d'architecture quantique, pour l'entretien
extrêmement intéressant que j'ai eu avec lui et pour m'avoir apporté des réponses sur de nombreux points.\\ \\
Je tiens à remercier M. Lerch, doctorant à l'EPFL, qui m'a accompagné pour améliorer mon travail pour le
concours \textit{Science et jeunesse} et qui m'a donné de précieux conseils et de nombreuses idées pour
compléter mon travail, ainsi qu'au reste de l'équipe du laboratoire de l'information et du calcul quantiques dirigé par
Mme. Holmes de l'EPFL pour m'avoir invité à visiter leurs bureaux et pour m'avoir donné des informations sur le sujet.\\ \\
Je désire également mentionner diverses conférences et séminaires auxquels j'ai assisté et qui m'ont permis d'approfondir
mes connaissances sur le sujet.
En particulier, la lecture de M. Aspect à l'occasion du Physics Day 2023 à l'EPFL, qui était très intéressante et qui
m'a permis de mieux comprendre certains enjeux du domaine des innovations quantiques, ainsi que les différents
intervenants lors du Quantum Industry Day in Switzerland 2023, qui m'ont offert de nombreuses perspectives sur le sujet.
Dans ce contexte, je tiens aussi à remercier IBM pour les ordinateurs mis à disposition qui m'ont permis d'exécuter
les algorithmes quantiques présentés dans ce travail, ainsi que pour leurs ressources éducatives en ligne qui m'ont
permis d'apprendre une grande partie de ce que je sais actuellement sur le sujet.\\ \\
Enfin, je remercie mes parents pour leur soutien et leur relecture de ce travail, ainsi que mon grand-père pour
sa relecture attentive des formules physiques et ses conseils avisés, tout particulièrement sur tout ce qui touche
à la physique quantique.
Finalement, je remercie pêle-mêle tout le reste de ma famille et mes amis pour m'avoir supporté tout au long de ce
travail.