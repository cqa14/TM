\chapter{Notions théoriques}\label{ch:notions-theoriques}

Avant d'aborder le vif du sujet, il est nécessaire de définir quelques notions théoriques.
En premier lieu, nous allons définir une méthode pour comparer les performances de différents algorithmes,
car c'est un point central qui justifie la recherche en informatique quantique.
Dans un second temps, nous allons définir les concepts de la mécanique quantique, sur lesquels
s'appuient les algorithmes quantiques.

\section{Informatique}\label{sec:informatique}

La différence de technologie interroge sur la possibilité de comparer les performances de celles-ci.
On voit bien que quelque chose est plus rapide qu'une autre en mesurant le temps, mais si la technologie
est encore à bâtir, il demeure intéressant de comparer les différents algorithmes.
On peut se dire qu'en les exécutant sur un ordinateur, on peut comparer les temps d'exécution.
Mais cela n'est pas fixe et reproductible, car cela dépend de tout un tas de facteurs.\\ \\
Pour illustrer cela, comparons sur une même machine le même algorithme, mais avec deux langages différents.
Premièrement, on va utiliser le Python, et deuxièmement, on va utiliser le C\@.
Le principe de l'algorithme est de calculer la factorielle d'un nombre.
\[ n! = \prod_{i=1}^n i = 1 \times 2 \times 3 \times
\ldots \times (n-1) \times n, n \in \mathbb{N} \]
Pour cela, on va utiliser l'algorithme suivant:

\import{images/preambules/algo/}{fact.tex}
Ce qui s'implémente comme suit dans les différents langages:

\begin{minipage}{0.45\textwidth}
\import{images/preambules/code/}{fun-fact-py.tex}
\end{minipage}
\hfill
\begin{minipage}{0.5\textwidth}
\import{images/preambules/code/}{fun-fact-c.tex}
\end{minipage}

Afin de comparer les temps d'exécution, on va appeler la fonction 10000 fois, et on va calculer la moyenne.
\import{images/preambules/code/}{python-fact.tex}
Dans ce premier programme, on peut noter utilisation du package \lstinline{timeit} qui permet de mesurer
le temps d'exécution.
De plus, le temps d'exécution est très court, de l'ordre de $10^{-5}$ secondes.
\import{images/preambules/code/}{c-fact.tex}
Dans ce second programme, la mesure du temps est plus compliquée, car il faut utiliser des fonctions
spécifiques à la librairie \lstinline{sys/time.h}, sans disposer d'outils dédiés.
De plus, la structure du C est plus complexe que celle du Python.
Néanmoins, le temps d'exécution est toujours très court, de l'ordre de $10^{-8}$ secondes, soit plus de 1000 fois
plus rapide.\\ \\
De plus, on peut faire une constatation similaire sur le temps d'exécution de la fonction selon la taille de
l'entrée.
\begin{figure}[H]
\centering
\import{images/preambules/graph/}{factorial-time-plot.tex}
\caption{Temps d'exécution de la fonction factorielle en Python selon la taille de l'entrée}
\label{fig:fact-plot-time}
\end{figure}
En premier abord, on peut penser que le temps d'exécution est proportionnel à la taille de l'entrée.
En effet, on fait une boucle de $1$ à $n$, et on fait une multiplication à chaque itération.
Or, on constate que le temps d'exécution est plus proche d'une fonction cubique que linéaire.
Cela est dû à la multiplication qui est plus longue pour un grand nombre.\\ \\
On constate donc que sur une même machine, deux programmes donnent des temps très différents pour un même algorithme.
À noter malgré tout que le C est un langage compilé alors que le Python est un langage interprété, ce qui
explique en partie la différence de temps d'exécution, mais pose malgré tout un problème de comparaison.
De plus, leurs temps d'exécution varient selon la taille de l'entrée.
Dès lors, comment comparer les temps d'exécution de deux algorithmes différents ?\\ \\
Les algorithmes sont alors classés selon leur complexité~\cite{wiki:complexity-gen}, soit leur ``temps d'exécution'' selon la taille de
l'entrée, indépendamment de la machine utilisée.
Ce n'est néanmoins pas la définition exacte de la complexité, car c'est plutôt l'ordre de grandeur du nombre d'opérations
à effectuer qui est important, mais cela donne une  excellente base de comparaison pour de grandes valeurs.
Symboliquement, on utilise la notation $\order{}$~\cite{wiki:big-o} pour exprimer la complexité d'un algorithme (à noter que d'autres
notations existent pour des usages similaires, comme $\Theta()$ ou $\Omega()$).\\
La définition de la complexité que l'on va utiliser est la suivante :
\[ f(x) = \order{g(x)} \text{ quand } x \rightarrow \infty \]
si la valeur absolue de $f(x)$ est à un multiple constant de $g(x)$ près, pour des valeurs de $x$ suffisamment
grandes.
Pour simplifier l'écriture, on écrira juste $f(x) = \order{g(x)}$.\\ \\
Cela permet d'expliquer la croissance de la factorielle, car on estime que la multiplication est en $\order{n^2}$ (pour simplifier les calculs, mais sur les grands nombres, c'est un peu moins~\cite{python-mult-comp}),
donc en la faisant $n$ fois, on obtient une complexité de $\order{n^3}$.\\
Prenons quelques autres exemples d'algorithmes afin de mieux comprendre ce concept :
\begin{enumerate}[wide, labelwidth=!, labelindent=0pt]
\item Calculer $(-1)^n$ : le temps ne dépend pas de l'entrée
donc la complexité est $\order{1}$ \\
\import{images/preambules/algo/}{o1.tex}
\item Trouver le maximum d'une liste de nombres : le temps dépend uniquement de la taille de la liste
donc la complexité est $\order{n}$ \\
\import{images/preambules/algo/}{on.tex}
\item Calculer naïvement le produit matriciel de deux matrices carrées de taille $n$ : il faudra calculer
n-fois la somme des n-lignes multipliée par les n-colonnes, donc $n \times n \times n = n^3$, soit $\order{n^3}$\\
\import{images/preambules/fig/}{matmultp.tex}
\import{images/preambules/algo/}{on3.tex}
\end{enumerate}
Sur le dernier exemple, le terme naïf est utilisé, car il existe des algorithmes plus efficaces pour calculer le
produit matriciel, comme par exemple l'algorithme de Strassen~\cite{wiki:matmult}, qui a une complexité de $\order{n^{2.807}}$, ou
celui de Coppersmith-Winograd avec $\order{n^{2.376}}$~\cite{wiki:matmult-copper} (Notons qu'il faudrait pour uniquement lire les entrées
des matrices une complexité $\order{n^2}$).\\ \\
Pour illustrer cette différence de complexité entre deux algorithmes, on va utiliser la recherche d'un nombre
dans une liste ordonnée.
Pour rechercher un nombre dans une liste, l'algorithme le plus simple est de parcourir la liste
jusqu'à trouver le nombre, ou jusqu'à la fin de la liste, avec une complexité de $\order{n}$
(voir Algorithme~\ref{alg:maximum}).\\
Mais si la liste est ordonnée, on peut optimiser la recherche selon le principe de recherche dichotomique.
On va comparer le nombre recherché avec le nombre au milieu de la liste, et si le nombre recherché est plus
grand, on va refaire la même opération sur la moitié supérieure de la liste, sinon sur la moitié inférieure.

\import{images/preambules/algo/}{bin-search.tex}
On voit donc que l'on divise notre liste par deux à chaque itération.
Si on considère que la taille de la liste est $n$ et $x$ le nombre d'opérations nécessaires pour trouver le
nombre, on a donc $n \times 2^{-x} = 1$, soit $x = \log_2(n)$, d'où une complexité de $\order{\log(n)}$
(pour la complexité, la base n'est pas précisée, car par la formule du changement de base du
logarithme, l'un ne varie de l'autre que par un scalaire).\\
On peut donc voir que la recherche dichotomique est beaucoup plus efficace que la recherche naïve, et que
lorsque la taille de la liste augmente, la différence de temps d'exécution entre les deux algorithmes
augmente de manière exponentielle.\\ \\
Ce genre d'amélioration est très important en informatique, car il permet de résoudre des problèmes
qui seraient autrement insolubles.
Ces optimisations sont presque toutes aussi importantes qu'une bonne implémentation, et il est donc
utile de les connaître.
La complexité des algorithmes quantiques vis-à-vis des algorithmes classiques est un des points
centraux justifiant la recherche dans ce domaine, parce que cela permettrait des accélérations
considérables dans de nombreux domaines.\\ \\
Il faut quand même prendre en compte que lorsque l'on implémente un algorithme, la complexité
réelle est souvent plus grande que la complexité théorique, car il faut prendre en compte
les opérations de base, comme les additions, les multiplications, les comparaisons, etc.
Ces opérations sont souvent considérées comme ayant une complexité $\order{1}$, mais en réalité, elles
dépendent de la taille des nombres manipulés, et donc de la taille de l'entrée.\\
Si on reprend l'Algorithme~\ref{alg:neg-one-power}, pour faire $(-1)^n$, on regarde la parité de $n$,
or pour cela, il y a plusieurs façons de le réaliser.
Premièrement, on peut regarder le dernier bit de $n$, et si il est à 1, alors $n$ est impair, sinon
il est pair, ou en base 10 si le dernier chiffre est dans \{0,2,4,6,8\}, alors c'est pair et sinon
impair.
Ceci est une opération en $\order{1}$, car on ne fait qu'une comparaison.
Mais si on n'y pense pas, on peut également faire $n \mod 2$ (donc si le résultat est 0, c'est pair
et sinon impair), qui est une opération beaucoup plus coûteuse, parce qu'elle nécessite une division.
\begin{figure}[H]
\centering
\import{images/preambules/graph/}{principal-complexity.tex}
\caption{Comparaison de quelques complexités courantes}
\label{fig:complexity-comparison}
\end{figure}
Sur le graphe ci-dessus, on peut voir clairement l'enjeu d'optimiser la complexité d'un algorithme,
d'autant plus que la taille de l'entrée augmente.
Cela permet de surpasser des obstacles qui seraient potentiellement insurmontables avec des
améliorations de performances matérielles uniquement, sur lesquelles on compte principalement dans notre
société actuelle en vertu de la loi de Moore~\cite{wiki:moore}.\\ \\
Cette question de la complexité des algorithmes peut être étendue à la théorie de la complexité, visant
à étudier la difficulté des problèmes, et à les classer selon leur complexité et également étudier
les relations entre les classes de complexité~\cite{wiki:complexity-theo}.
Par exemple, on va dénommer $P$ la classe des problèmes qui peuvent être résolus en temps polynomial,
où nous pouvons citer le problème de la multiplication de deux matrices.
Une autre classe célèbre est $NP$, qui contient les problèmes qui peuvent être vérifiés en temps polynomial,
comme par exemple un sudoku : il est facile de vérifier si une grille est correcte, mais il est difficile
de la résoudre de manière générale.\\
Cette question des classes de complexité est très importante, à tel point qu'elle est considérée comme
l'un des sept problèmes du prix du millénaire, celui nommé vulgairement $P=NP$ et visant à déterminer
si tous les problèmes vérifiables en temps polynomial sont également résolubles en temps polynomial.
\href{https://www.claymath.org/millennium-problems/}{Ce sont sept problèmes mathématiques qui ont été définis par le Clay Mathematics Institute en 2000}, et pour
lesquels une récompense d'un million de dollars est promise à celui qui les résoudra.
Les algorithmes quantiques ne remettent pas directement en cause cette question, mais ils en ouvrent de nouvelles,
dans ce domaine, car ils permettent de résoudre des problèmes qui sont dans $NP$ en temps polynomial avec
une probabilité assez élevée~\cite{wiki:bqp}.\\ \\
En résumé, la complexité d'un algorithme est une mesure de sa difficulté, et permet de comparer les
performances de différents algorithmes indépendamment de la machine utilisée.
Cette mesure peut être tout autant dans le temps d'exécution que dans l'espace mémoire utilisé.
Néanmoins, de par sa définition et sa nature asymptotique, la mesure de la complexité est vraiment
pertinente quand les entrées sont grandes, car sinon il faut prendre en compte les machines utilisées.

\section{Physique}\label{sec:physique}

Derrière le terme d'algorithme quantique se trouve les concepts de la mécanique quantique.
Celle-ci vise à décrire le comportement de la matière à l'échelle atomique et subatomique.
Elle est basée sur le principe de dualité onde-corpuscule, qui dit que selon l’expérience envisagée une particule
(objet localisé dans l’espace) se comporte comme une onde (phénomène étendu) et inversement une onde se
comporte comme une particule.\\
Plus concrètement, cette idée peut déjà se pressentir avec les études sur la lumière dès le 19\up{ème} siècle.
En effet, c'est en 1801 que Thomas Young réalise l'expérience qui porte son nom, les fentes de Young~\cite{wiki:young},
qui met en évidence l’aspect ondulatoire de la lumière par les franges d’interférence qui résultent de la superposition
des ondes émises par chacun des deux trous.
Avec des particules classiques on verrait uniquement deux taches suivant que la particule est passée par l’un ou l’autre trou .
\import{images/preambules/fig/}{fente-young.tex}
Le principe de cette expérience est de faire passer un faisceau de lumière à travers deux fentes
parallèles, et d'observer le résultat sur un écran.
Là où l'on s'attendrait à voir deux taches lumineuses, on observe en réalité une multitude de
taches, propres aux franges d'interférences entre deux ondes.
Cette conception de la lumière comme une onde semble être alors la plus probable, car la théorie de
Maxwell, qui décrit les ondes électromagnétiques, permet de décrire la lumière comme tel.\\ \\
%C'est au début du 20\up{ème} siècle que la physique quantique va émerger~\cite{wiki:mech-quantique}, avec les travaux
%de Max Planck sur le rayonnement du corps noir, où il va introduire le concept de quantification,
%soit que l'énergie est émise par paquets, ou quanta, proportionnellement à la fréquence de l'onde.
%C'est également à cette époque que l'on va découvrir l'effet photoélectrique~\cite{wiki:photoelectrique}, duquel Albert Einstein
%va déduire que la lumière est composée de particules, les photons, qui ont une énergie proportionnelle
%à la fréquence de l'onde.
%La relation qu'il en ressort est la suivante :
%\[
%E = h \nu
%\]
%où $E$ est l'énergie du photon, $h$ est la constante de Planck et $\nu$ est la fréquence de l'onde.\\ \\
%Voyant que la lumière, jusque-là considérée comme une onde, peut se comporter comme une particule, où plutôt un paquet dit quanta de lumière ou photon,
%on va se demander si les particules, jusque-là considérées comme des corpuscules, peuvent se comporter
%comme des ondes.
%Le mathématicien français Louis De Broglie va alors proposer que toute particule a une longueur d'onde
%associée~\cite{myst-monde-quantique}, et donc que toute particule peut se comporter comme une onde.
%Là où l'énergie est proportionnelle à la fréquence pour une onde, De Broglie va proposer que la quantité
%de mouvement est proportionnelle à la longueur d'onde pour une particule :
%\[
%p = \frac{h}{\lambda}
%\]
%où $p$ est la quantité de mouvement de la particule, $h$ est la constante de Planck et $\lambda$ est
%la longueur d'onde de la particule.\\ \\
C’est en 1900 que l’idée de quantification va émerger~\cite{wiki:mech-quantique} avec les travaux de Max Planck sur le problème du
rayonnement du corps noir.
Il obtient de manière empirique la formule pour ce rayonnement.
Pour interpréter cette équation, il introduit l’hypothèse que la matière ne peut rayonner de l’énergie que
par des quantités finies - ou quanta - proportionnelles à la fréquence de vibration des atomes.
Toutefois, il n’envisage pas que la lumière puisse avoir une structure discontinue.
Au contraire pour élaborer sa théorie de l’effet photoélectrique (découvert en 1887)~\cite{wiki:photoelectrique} Einstein introduit
en 1905 l’hypothèse que lors de l’émission ou l’absorption d’un rayonnement lumineux par la matière, celui-ci doit
être considéré comme formé d’un nombre fini de petits grains indivisibles, ou ``quantas'', parfaitement localisés dans
l’espace, dont l’énergie $E$ et la quantité de mouvement $p$ sont proportionnelles à la fréquence de l’onde, soit
\[
E = h \nu \qquad \text{et} \qquad p =  h \frac{\nu}{c}
\]
où $h$ est la constante de Planck, $\nu$ la fréquence de l’onde et $c$ la vitesse de la lumière.\\ \\
Au contraire lors de sa propagation la lumière doit être considérée comme une onde obéissant aux équations de
Maxwell.
Mais les physiciens ont du mal à accepter cette hypothèse, en particulier Max Planck qui dans un rapport
élogieux sur Einstein écrit en 1913 ``il est vrai qu’il a parfois manqué le but lors de ses spéculations par
exemple avec son hypothèse sur les quantas lumineux.''\\
L’hypothèse d’Einstein sera confirmée en 1923 par les expériences de Compton et en 1926 on donnera le nom de photon
à cette particule lumineuse.\\ \\
En 1923 le physicien français Louis de Broglie suggère d’associer à toute particule massive d’énergie $E$ et
de quantité de mouvement $p = mv$ une onde de fréquence $\nu$ et de longueur d’onde $\lambda$ avec
\[
    \nu = \frac{E}{h} \qquad \text{et} \qquad \lambda = \frac{h}{p}
\]
Cette hypothèse lui permet d’expliquer les règle de quantification des orbites électroniques dans l’atome introduite
par Bohr en 1913 par décrire les spectres atomiques~\cite{myst-monde-quantique}.\\ \\
De ces théories, le physicien allemand Erwin Schrödinger va se poser la question de connaître l'équation
d'onde qui décrit le comportement d'une particule.
Dans ce qui suit, on va décrire un chemin afin de la comprendre avec une dimension spatiale et une
dimension temporelle, et la généralisation va être mentionnée sans démonstration.\\ \\
Mais avant de commencer, il faut définir ce qu'est une onde.
Une onde est une perturbation qui se propage dans un milieu, et qui transporte de l'énergie sans
transporter de matière.
La perturbation peut être de différentes natures, comme une perturbation mécanique, électromagnétique,
etc.
On peut la décrire de manière classique par une fonction $y(x,t)$, où $x$ est la position dans l'espace et
$t$ est le temps.
Cette fonction est solution de l'équation d'onde classique à une dimension :
\[
    \pdv[2]{y}{x} = \frac{1}{v^2} \pdv[2]{y}{t}
\]
où $v$ est la vitesse de propagation de l'onde.
\begin{proof}
Pour montrer que cette équation fonctionne bel et bien, on va considérer comme onde une corde vibrante en deux dimensions~\cite{wiki:onde-corde}.
Prenons un tronçon de corde infinitésimal $AB$ avec $A(x;y)$ et $B(x+\dd{x};y+\dd{y})$, et l'angle entre la tangente
à la corde et l'horizontal en ce point au temps $t$ donné par $\alpha (x,t)$ (respectivement $\alpha (x+\dd{x},t)$).
En négligeant la force pesante, on a uniquement la force de tension qui agit sur la corde, $\vec{T_A}$ en $A$
et $\vec{T_B}$ en $B$.
\import{images/preambules/fig/}{onde-corde.tex}
Connaissant la masse linéique de la corde $\mu$, on peut écrire la relation suivante:
\[
    \vec{T_A} + \vec{T_B} = \mu \dd{x} \vec{a}
\]
où $\vec{a} = \mqty(\pdv[2]{x}{t} \\ \pdv[2]{y}{t}) = \mqty(0 \\ \pdv[2]{y}{t})$ est l'accélération de la corde
(nulle horizontalement, car il n'y a qu'un déplacement vertical).\\ \\
On peut décomposer l'équation en deux équations, une selon l'axe horizontal $Ox$ et une selon l'axe verticale $Oy$.
\[
    \begin{cases}
        Ox \ : \ - T_A \cos \alpha (x,t) + T_B \cos \alpha (x+\dd{x},t) = 0 \\
        Oy \ : \ - T_A \sin \alpha (x,t) + T_B \sin \alpha (x+\dd{x},t) = \mu \dd{x} \pdv[2]{y}{t}
    \end{cases}
\]
En utilisant les approximations du premier ordre $\sin \alpha \approx \alpha$ et $\cos \alpha \approx 1$, si l'on
considère la variation crée par l'onde et donc les angles comme faibles, on obtient:
\[
    \begin{cases}
        Ox \ : \ - T_A + T_B = 0 \\
        Oy \ : \ - T_A \alpha (x,t) + T_B \alpha (x+\dd{x},t) = \mu \dd{x} \pdv[2]{y}{t}
    \end{cases}
    \Rightarrow
    \begin{cases}
        T_A = T_B = T \\
        - T \alpha (x,t) + T \alpha (x+\dd{x},t) = \mu \dd{x} \pdv[2]{y}{t}
    \end{cases}
\]
De plus, nous pouvons considérer que $\alpha = \tan \alpha = \pdv{y}{x}$, d'où:
\[
    -T \pdv{y}{x} \left( x,t \right) + T \pdv{y}{x} \left( x+\dd{x},t \right) = \mu \dd{x} \pdv[2]{y}{t} \left( x,t \right)
\]
En notant que
\[
    \pdv{y}{x} \left( x+\dd{x},t \right) - \pdv{y}{x} \left( x,t \right) =
    \pdv{x} \left( y \left( x+\dd{x},t \right) - y(x,t) \right) = \pdv{x} \left(  \pdv{y}{x} \left( x,t \right) \dd{x} \right) =
    \pdv[2]{y}{x} \dd{x}
\]
on obtient
\[
    T \pdv[2]{y}{x} \dd{x} = \mu \dd{x} \pdv[2]{y}{t} \Leftrightarrow \pdv[2]{y}{x} = \frac{\mu}{T} \pdv[2]{y}{t}
\]
qui est une équation d'onde, ou équation de d'Alembert, de forme
\[
    \pdv[2]{y}{x} = \frac{1}{v^2} \pdv[2]{y}{t}
\]
où $v = \sqrt{\frac{T}{\mu}}$ est la célérité de l'onde sur la corde dans ce cas.
\end{proof}
On peut noter que la célérité est reliée à la fréquence $\nu$ et à la longueur d'onde $\lambda$ par la relation
$v = \lambda \nu$.
De plus, on peut remarquer que l'équation d'onde est une équation linéaire, c'est-à-dire que si $y_1$ et $y_2$
sont deux solutions de l'équation, alors $y = y_1 + y_2$ est aussi une solution, ce qui va induire le principe dit de superposition.\\ \\
Sachant cela, essayons de construire une équation d'onde pour une particule~\cite{schrodinger-eq}.
Dans le cas classique, une solution possible est une onde plane, de la forme
\[
    y(x,t) = A \cos \left( kx - \omega t \right)
\]
Avec $k = \frac{2 \pi}{\lambda}$ le nombre d'onde et $\omega = 2 \pi \nu$ la pulsation.
Or, des formules $E = h \nu$ et $p = \frac{h}{\lambda}$, on peut déduire que tant qu'il n'y a pas de forces
extérieures, la quantité de mouvement $p$ ne varie pas et donc la longueur d'onde $\lambda$ non plus.
Mais pour pouvoir construire une équation générale, il faut savoir comment cela se passe aussi dans le cas
où cela n'est pas vrai.
L'équation doit donc correspondre à plusieurs postulats:
\begin{enumerate}
    \item Les postulats de De Broglie-Einstein: $E = h \nu$ et $p = \frac{h}{\lambda}$
    \item L'équation d'énergie $E = K + P$ avec $E$ l'énergie totale, $K$ l'énergie cinétique et $P$ l'énergie
        potentielle ; de plus, $K = \frac{1}{2}mv^2 = \frac{p^2}{2m}$ (avec $m$ la masse de la particule, $v$ sa
        vitesse et $p$ sa quantité de mouvement) et $P = V(x,t)$ avec $V$ le potentiel d'où $F = - \pdv{V}{x}$ ;
        ainsi, $E = \frac{p^2}{2m} + V$
    \item L'équation doit être linéaire (comme l'équation d'onde)
    \item Pour un potentiel $V$ constant, l'équation différentielle doit avoir une onde sinusoïdale comme solution
        de longueur d'onde et de fréquence constantes (car $V(x,t) = constante \Rightarrow F = - \pdv{V}{x} = 0$
        et donc $p = constante$ car $F = \dv{p}{t}$)
\end{enumerate}
De ces postulats, on déduit que
\[
    E = \frac{p^2}{2m} + V(x,t) = \frac{h^2}{2m \lambda^2} + V(x,t) = h \nu
\]
puis en remplaçant $\nu$ par $\frac{\omega}{2 \pi}$ et $\lambda$ par $\frac{2 \pi}{k}$, ainsi qu'utilisant la constante
de Planck réduite $\hbar = \frac{h}{2 \pi}$ afin de simplifier la notation, on obtient
\[
    \frac{\hbar^2 k^2}{2m} + V(x,t) = \hbar \omega
\]
sur laquelle on note un facteur $k^2$ à gauche et un facteur $\omega$ à droite.
De la 4\ieme{} condition, commençons par considérer que la fonction recherchée est de la forme
\[
    \Psi (x,t) = \cos \left( kx - \omega t \right)
\]
et de là notons que
\[
    \pdv[2]{\Psi (x,t)}{x} = - k^2 \Psi (x,t) \qquad \text{et} \qquad \pdv{\Psi (x,t)}{t} = \omega \sin \left( kx - \omega t \right)
\]
d'où la conclusion que l'équation doit contenir une dérivée seconde par rapport à $x$ et une dérivée première par
rapport à $t$, ainsi qu'un facteur avec le potentiel $V(x,t)$, multiplié par $\Psi (x,t)$ afin d'assurer
la linéarité de l'équation.\\ \\
Posons donc une équation de la forme
\[
    \alpha \pdv[2]{\Psi (x,t)}{x} + V(x,t) \Psi (x,t) = \beta \pdv{\Psi (x,t)}{t}
\]
où $\alpha$ et $\beta$ sont des constantes à déterminer.\\
En essayant de vérifier la condition 4, on pose $\Psi (x,t) = \cos \left( kx - \omega t \right)$ et $V(x,t)=V_0$,
et on regarde ce que cela donne:
\begin{gather*}
    \alpha \pdv[2]{\Psi (x,t)}{x} + V_0 \Psi (x,t) = \beta \pdv{\Psi (x,t)}{t} \\
    \Leftrightarrow - \alpha k^2 \cos \left( kx - \omega t \right) + V_0 \cos \left( kx - \omega t \right) =
    \beta \omega \sin \left( kx - \omega t \right)\\
    \Leftrightarrow \left( \alpha k^2  - V_0 \right) \cos \left( kx - \omega t \right) + \beta \omega \sin \left( kx - \omega t \right) = 0\\
\end{gather*}
Cette égalité ne se vérifie que si $k^2 = \frac{V_0}{\alpha}$ et $\beta = 0$, or sous ces conditions, on ne satisfait pas
l'équation d'énergie $\frac{\hbar^2 k^2}{2m} + V(x,t) = \hbar \omega$.\\
Nous devons donc considérer une autre forme pour $\Psi (x,t)$, et nous allons essayer avec
\[
    \Psi (x,t) = \cos \left( kx - \omega t \right) + \gamma \sin \left( kx - \omega t \right)
\]
où $\gamma$ est une constante à déterminer.
On a alors
\begin{gather*}
    \pdv{\Psi (x,t)}{t} = \omega \left( \sin \left( kx - \omega t \right) - \gamma \cos \left( kx - \omega t \right) \right)\\
    \pdv[2]{\Psi (x,t)}{x} = - k^2 \cos \left( kx - \omega t \right) - \gamma k^2 \sin \left( kx - \omega t \right)
\end{gather*}
et donc
\begin{gather*}
    \alpha \pdv[2]{\Psi (x,t)}{x} + V(x,t) \Psi (x,t) = \beta \pdv{\Psi (x,t)}{t}\\
    \Leftrightarrow - \alpha k^2 \cos \left( kx - \omega t \right) - \alpha \gamma k^2 \sin \left( kx - \omega t \right) + V_0 \left( \cos \left( kx - \omega t \right) + \gamma \sin \left( kx - \omega t \right) \right) = \\ = \beta \omega \left( \sin \left( kx - \omega t \right) - \gamma \cos \left( kx - \omega t \right) \right)\\
    \Leftrightarrow - \alpha k^2 \left[ \cos \left( kx - \omega t \right) + \gamma \sin \left( kx - \omega t \right) \right] + V_0 \left[ \cos \left( kx - \omega t \right) + \gamma \sin \left( kx - \omega t \right) \right] = \\ = \frac{\beta \omega}{\gamma} \left[- \gamma^2 \cos \left( kx - \omega t \right) + \gamma \sin \left( kx - \omega t \right) \right]\\
\end{gather*}
qui contient dans les crochets à gauche du signe égal la fonction $\Psi (x,t)$, et dans les crochet à droite du signe
égal une fonction qui y ressemble, mais qui n'est pas exactement la même.\\
Nous pourrions alors essayer de trouver une valeur de $\gamma$ qui permet de faire correspondre les deux fonctions,
soit satisfaisant l'équation
\[
- \gamma^2 \cos \left( kx - \omega t \right) + \gamma \sin \left( kx - \omega t \right) = \cos \left( kx - \omega t \right) + \gamma \sin \left( kx - \omega t \right)
\]
ce qui nécessite que $\gamma^2 = -1$, donc $\gamma = \pm \sqrt{-1} = \pm i$, où $i$ est l'unité imaginaire.
En posant $\gamma = i$, on peut simplifier l'équation précédente en
\[
    - \alpha k^2 + V_0 = \frac{\beta \omega}{i}
\]
et en comparant avec l'équation de De Broglie-Einstein $\frac{\hbar^2 k^2}{2m} + V(x,t) = \hbar \omega$,
on voit que l'équation précédente est satisfaite si $\alpha = - \frac{\hbar^2}{2m}$ et $\beta = i \hbar$.\\
Nous avons donc trouvé une équation qui satisfait les conditions 1, 2, 3 et 4, et qui est donc une équation
d'onde quantique pour une particule de masse $m$ dans un potentiel $V(x,t)$, à savoir
\[
    - \frac{\hbar^2}{2m} \pdv[2]{\Psi (x,t)}{x} + V(x,t) \Psi (x,t) = i \hbar \pdv{\Psi (x,t)}{t}
\]
dénommée équation de Schrödinger (à noter que tout comme il existe une équation d'onde de dimension plus grande,
il y a aussi une équation de Schrödinger de dimension supérieure).\\
On admet que pour un instant $t_0$ donné la fonction d’onde $\Psi (x, t_0)$ décrit l’état de la particule
à cet instant.
Comme l’équation de Schrödinger est linéaire, on remarque que connaissant deux états, définis par deux solutions
de cette équation, on obtient par superposition linéaire de ces deux solutions une nouvelle solution, c'est-à-dire
un nouvel état.
C’est ce que l’on appelle ``principe de superposition des états'' : à partir de deux états de la particule, on
peut définir de nouveaux états par combinaisons linéaires.
C’est un principe entièrement nouveau qui n’a pas de correspondance en physique classique.
En utilisant la notation introduite par Dirac, on représente cet état sous la forme d’un vecteur $\ket{\Psi_t}$,
appelé ``ket''.\\ \\
Cette équation prédit excellemment bien les résultats expérimentaux de la mécanique quantique, mais son
interprétation est plus difficile, à commencer par que représente $\Psi (x,t)$, pourquoi l'unité imaginaire
est-elle présente dans l'équation, etc.\\
Cela peut s'expliquer par la différence majeure entre la mécanique classique et la mécanique quantique:
en mécanique classique, la mesure d'une grandeur physique ne change pas la valeur de n'importe quelle autre
grandeur physique, alors qu'en mécanique quantique, la mesure d'une grandeur physique change la valeur
de certaines autres grandeurs physiques.
Cela s'exprime par les relations d'incertitude d'Heisenberg, qui sont des relations entre les incertitudes
de deux grandeurs physiques, et qui sont données par
\[
    \Delta x \Delta p_x \geq \frac{\hbar}{2} \qquad \text{et} \qquad \Delta E \Delta t \geq \frac{\hbar}{2}
\]
qui implique que plus on connaît précisément la position d'une particule, moins on connaît précisément
son impulsion, et vice-versa, et plus on connaît précisément l'énergie d'une particule, moins on connaît
précisément le temps qu'elle mettra à parcourir une certaine distance, et réciproquement.\\ \\
L'équation de Schrödinger est donc une équation qui décrit l'évolution de la fonction d'onde d'une particule
dans le temps, mais comme elle est complexe, elle ne peut pas être interprétée comme une mesure directe.
Or, on sait que le module carré d'un nombre complexe est un nombre réel positif, et donc on peut interpréter
cela comme une mesure.
L'interprétation de la fonction d'onde donnée par Max Born est que son module au carré est la probabilité de trouver la
particule dans un certain état, et donc l'équation de Schrödinger est une équation qui décrit l'évolution de
la probabilité de trouver une particule dans un certain état dans le temps.\\
Dans le cas de la position, qui est une variable continue, la probabilité de trouver la particule dans l'intervalle
$[x_1,x_2]$ est donnée par
\[
    P(x_1 \leq x \leq x_2) = \int_{x_1}^{x_2} \abs{\Psi (x,t)}^2 \dd{x}
\]
ce qui nous indique également, comme la somme des probabilités doit être égale à 1, que la fonction d'onde
doit être normalisée, c'est-à-dire que
\[
    \int_{- \infty}^{+ \infty} \abs{\Psi (x,t)}^2 \dd{x} = 1
\]
Cependant, dans la suite, on va considérer des grandeurs qui ne prennent que des valeurs discrètes, car l'informatique quantique fonctionne à l'image de
celle classique, c'est-à-dire avec des bits, qui ne peuvent prendre que deux valeurs, 0 ou 1.
De plus on va utiliser la notation de Dirac afin de représenter les états quantiques, qui est une notation
qui permet de représenter les états quantiques de manière plus simple et plus compacte.\\
On va donner un exemple de cette notation, et on en posera les concepts plus rigoureusement par la suite.
La grandeur spin a été mise en évidence en 1922 par O. Stern et W. Gerlach qui voulaient mesurer le moment
angulaire d’un atome pour tester l’hypothèse de quantification du plan des orbites des électrons introduite
par Sommerfeld.
Le moment angulaire étant parallèle au moment magnétique, ils envoient des atomes à travers un champ magnétique
et ils s'attendent à trouver sur un écran, situé après l’aimant, un nombre impair de taches correspondant aux
différentes directions possibles du moment angulaire.
Ils observent deux taches ce qui est un résultat incompréhensible.
L’explication sera donnée en 1925 par G. Uhlenbeck et S. Goudsmit qui introduisent l'idée véritablement
révolutionnaire que l’électron possède un moment angulaire intrinsèque, grandeur vectorielle sans équivalent
classique, qui ne peut prendre deux valeurs $+ \frac{1}{2}$ ou $- \frac{1}{2}$ quand mesurée dans une direction donnée.
Pauli donnera le nom de spin à cette nouvelle grandeur.
Comme elle ne peut prendre que deux valeurs, on peut représenter ces deux états par deux vecteurs, qui sont
appelés vecteurs d'état, par exemple $\ket{+}$ et $\ket{-}$ pour les deux états de spin.\\
On représente alors l'état de spin de la particule par un vecteur qui est une combinaison linéaire des deux vecteurs
d'état, par exemple $\ket{\psi} = c_1 \ket{+} + c_2 \ket{-}$, comme les deux états sont ``solutions'' de notre
problème, et donc on peut représenter l'état de la particule par un vecteur qui est une combinaison linéaire
de ceux-ci en vertu du 3\ieme{} postulat évoqué précédemment.
Selon l'interprétation de Born, la probabilité de trouver la particule dans l'état $\ket{+}$ est donnée par
$\abs{c_1}^2$, et la probabilité de trouver la particule dans l'état $\ket{-}$ est donnée par $\abs{c_2}^2$,
et donc la somme des probabilités est égale à 1, ce qui implique que $\abs{c_1}^2 + \abs{c_2}^2 = 1$.\\ \\
Ce formalisme peut paraître étrange, car il affirme que la particule n’est ni dans l’état $\ket{+}$ ni dans l’état
$\ket{-}$, mais que si on fait une mesure, on l’observera soit dans l’état $\ket{+}$, soit dans l’état $\ket{-}$ avec un résultat
aléatoire dont la probabilité est connue et après la mesure la particule sera dans l’état observé avec probabilité 1.
Or ce phénomène est réel, et il est appelé superposition quantique~\cite{wiki:superposition}.
C'est similaire au comportement lorsque deux ondes classiques se rencontrent, elles vont en quelque sorte s'additionner,
comme à l'image de deux vagues qui se rencontrent en créant des vaguelettes l'une sur l'autre par endroit, augmentant
leur taille à d'autre ou encore s'annulant en certains points.\\
Le fameux chat de Schrödinger~\cite{wiki:chat-schrodinger} est une expérience de pensée visant à illustrer ce phénomène.
Un dispositif diabolique est mis en place, qui consiste en une boîte hermétique contenant un chat, un flacon
de poison, un détecteur de radioactivité et une source radioactive, et si le détecteur détecte une particule
radioactive, alors le flacon de poison est cassé, et le chat meurt.
Or le fait que la particule radioactive se désintègre ou non est un phénomène quantique, et donc tant que l'on ne
regarde pas dans la boîte, on ne sait pas si la particule s'est désintégrée ou non, et donc si le chat est mort
ou vivant.
Cela parait impossible et montre les lacunes de l'interprétation de la mécanique quantique, mais des phénomènes
semblables peuvent être réalisés en laboratoire.\\ \\
Citons par exemple l'interféromètre de Mach-Zehnder~\cite{wiki:exp-superposition}, qui est un dispositif permettant de mettre en évidence
le phénomène d'interférence quantique, qui est un phénomène qui n'a pas d'équivalent en mécanique classique.
C'est un dispositif qui consiste en un laser, qui est une source de photons, qui sont des particules de lumière,
qui est envoyé sur un miroir semi-réfléchissant, qui va diviser le faisceau en deux faisceaux, qui vont suivre
des chemins différents, et qui vont être réfléchis par deux miroirs, et sont redirigés vers un autre miroir
semi-réfléchissant, qui va recombiner les deux faisceaux, et les envoyer vers deux détecteurs.
La lumière agit comme une onde, et le montage est conçu de manière que les deux faisceaux se recombinent
avec une différence de phase telle que les deux ondes s'annulent sur une des sorties, et se renforcent sur
l'autre.
\import{images/preambules/fig/}{mach-zender.tex}
L'expérience devient intéressante lorsque l'on envoie les photons un par un, car on s'attendrait à ce qu'ils
arrivent sur les deux détecteurs comme ils le feraient s'ils étaient des particules classiques, parce qu'ils ne devraient,
selon notre intuition, parcourir qu'un seul des deux chemins et donc pouvoir atteindre les deux détecteurs, car
ils ne devraient plus subir d'interférences.
Or la réalité est différente, et on observe que les photons arrivent toujours sur le même détecteur, et donc
que les photons interfèrent avec eux-mêmes, et qu'ils parcourent en quelque sorte les deux chemins en même temps.\\
Pour en revenir aux concepts évoqués précédemment, le premier miroir va créer une superposition quantique des
deux états, qui sont les deux chemins possibles, et le deuxième miroir va impacter à nouveau cette superposition,
et crée cette interférence quantique, qui mène à cette mesure étrange.\\ \\
Le dernier concept que l'on va évoquer est celui de l'intrication quantique~\cite{wiki:entenglement}, qui se produit lorsque deux
particules interagissent, et que l'on ne peut plus décrire l'état de chacune des particules indépendamment,
mais que l'on doit décrire l'état du système formé par les deux particules.
Si on reprend l'expérience de pensée du chat de Schrödinger, on peut imaginer que la vie du chat est liée à
l'état de la particule radioactive, et donc que tant que l'on ne regarde pas dans la boîte, le chat est à la
fois mort et vivant, et que l'on ne peut pas décrire l'état du chat sans décrire l'état de la particule radioactive.
Cela peut paraître étrange, mais c'est un phénomène réel, et il a été mis en évidence par l'expérience d'Aspect,
qui lui a valu le prix Nobel de physique en 2022.\\
Ce concept d'intrication fut mis en évidence par Schrödinger, puis Einstein, Podolsky et Rosen, ont émis l'hypothèse
que la mécanique quantique était incomplète, et que les particules possédaient des propriétés cachées, car sinon
cela impliquerait que les particules communiquent instantanément, ce qui est impossible selon la relativité
restreinte, car cela impliquerait que l'information se déplace plus vite que la lumière.
Néanmoins, cette simultanéité ne va pas à l'encontre de la relativité restreinte, via le théorème de non-communication,
car pour constater cette causalité, il faudrait pouvoir communiquer de l'information, ce qui se fait à la vitesse
de la lumière au maximum.\\
Néanmoins, l'hypothèse EPR (Einstein, Podolsky, Rosen) suggérait qu'il y avait des propriétés cachées, et donc
que la mécanique quantique était incomplète, et donc qu'il existait une théorie plus fondamentale.
C'est dans ce contexte que John Bell a énoncé son théorème, qui permet de tester expérimentalement l'hypothèse EPR,
et qui consiste en une inégalité, qui si elle est violée, implique qu'il y a une influence instantanée entre les
particules et non un paramètre autre, et donc contredit l'hypothèse EPR.
Alain Aspect a réalisé une expérience qui a violé cette inégalité de manière indiscutable~\cite{wiki:aspect}, et qui de plus était
en parfait accord avec la mécanique quantique.
\import{images/preambules/fig/}{aspect.tex}
Cette expérience consiste en un laser, qui envoie des photons sur un cristal, qui va créer une paire de photons
intriqués, qui vont être envoyés sur deux détecteurs, qui vont mesurer la polarisation des photons.
Les deux détecteurs mesurent la polarisation des photons selon deux axes différents, et on peut choisir l'angle
de ces axes, et donc on peut choisir la polarisation que l'on veut mesurer.
On va changer cet angle après que les photons soient partis du cristal, et donc après qu'ils soient intriqués,
et on va mesurer la polarisation des photons, et c'est cette polarisation qui va être corrélée, et qui va nous
permettre de tester l'inégalité de Bell.\\ \\
En résumé, la mécanique quantique est une théorie qui décrit le monde à l'échelle microscopique et qui fait
apparaître des phénomènes qui n'ont pas d'équivalent en mécanique classique, et qui sont contre-intuitifs.
Ceux-ci font intervenir des concepts probabilistes, de superposition quantique, d'intrication quantique, et
d'interférence quantique, mais qui sont néanmoins prédictifs et vérifiables expérimentalement.\\
Cela fait de la physique quantique une théorie déterministe des systèmes considérés pour ce qui est de l’évolution
temporelle, c'est-à-dire qui permet de décrire l’évolution d’un système à partir des conditions initiales,
ce qui rend ses concepts applicables à l'informatique.
Néanmoins, la mesure d’une observable va modifier l’état de façon aléatoire.
On va donc perdre une partie de l'information lors de celle-ci, ce qui fait que les résultats sont
probabilistes, et donc non purement déterministes à cet égard.
De fait, on peut utiliser les propriétés de superposition des états quantiques afin de réaliser des calculs
sur plusieurs états en même temps, et donc de manière parallèle, avec l'intrication qui permet de lier différents
états entre eux et d'augmenter encore les possibilités de calculs.
Tout cela permet de réaliser des opérations complexes, mais il faut garder en tête la perte d'information lors
de la mesure, et les problèmes qui en découlent, car une interaction avec l'environnement peut détruire la
fonction d'onde, tout comme une mesure, et donc détruire les calculs en cours.