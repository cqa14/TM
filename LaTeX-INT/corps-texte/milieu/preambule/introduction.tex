\chapter{Introduction}\label{ch:introduction}

``C'est quoi un projet que tu trouves particulièrement intéressant que t'as vu ?'' est la question posée
à Octave Klaba \footnote{De \href{https://www.youtube.com/watch?v=GRnWBIJb_Oo}{\textbf{Underscore\textunderscore} :
\textit{On a recu le milliardaire qui fait trembler amazon}}}, président d'\href{https://www.ovhcloud.com/fr/}{OVH}, l'un des plus gros hébergeurs
de sites web d'Europe.
Il y répond ``[\ldots] Mais s'il y a un sujet, c'est vraiment quantique.
En fait, il faut bien voir, c'est que tout le siècle dernier était basé sur l'atome [\ldots] Celui-là,
c'est le quantique.'' \\
Cette citation est un bon exemple de l'importance de la physique quantique dans le domaine
de l'informatique.
En effet, parmi tous les domaines de recherches technologiques, comme l'amélioration des machines
existantes, les recherches en informatique biologique avec par exemple le stockage de l'information
dans l'ADN~\cite{wiki:natural-computing}, ou encore les recherches en informatique quantique, c'est cette dernière qui est la plus
mise en avant par un homme aussi bien placé dans le domaine de l'innovation technologique. \\ \\
L'idée d'utiliser la physique quantique pour améliorer l'informatique est apparue dans les années 1980,
proposée indépendamment par Richard Feynman, prix Nobel de physique en 1965, et Yuri Manin.
Ce fut par la demande exponentielle de ressources informatiques, pour faire de la simulation de systèmes quantiques
dynamiques, qu'émergea l'idée d'utiliser des ordinateurs basés sur des phénomènes quantiques.\\
Entre 1980 et 2000, le domaine de l'informatique et plus précisément de l'algorithmie quantique
voit apparaître de nombreux résultats théoriques très prometteurs.
Le premier cas d'un avantage théorique est montré par David Deutsch en 1985, via un cas de problème
dit de ``boite noire'', puis ces problèmes vont être très étudié, comme la généralisation de celui étudié
par Deutsch avec l'aide de Richard Jozsa.\\
Les idées d'applications concrètes apparaissent tout d'abord en 1984, avec la proposition d'utiliser les
propriétés quantiques afin d'améliorer la sécurité des communications, via de potentiels nouveaux protocoles
de cryptographie qui utiliserait des clés dites ``quantiques''.\\
D'un autre côté, en 1994, Peter Shor crée, en s'appuyant sur les résultats de ses prédécesseurs, un algorithme
de factorisation des nombres entiers en facteurs premiers qui pourrait casser le système de sécurité alors très répandu
\textit{RSA}.
Cela inquiéta suffisamment les spécialistes de la sécurité informatique pour que l'on crée des protocoles
de cryptographie plus sûrs, et surtout résistant ces potentielles nouvelles technologies.\\
Les dernières avancées majeures de cette période sont dues à Lov Grover, proposant un algorithme
quantique plus rapide que celui classique dans la résolution d'une catégorie de problèmes assez répandu.
Dans cette même année 1996, Seth Lloyd prouve que les ordinateurs quantiques permettent de simuler
des systèmes quantiques avec un gros avantage sur les simulations classiques~\cite{wiki:hisotry-qc}.\\ \\
En pratique, la première démonstration que la technologie est réalisable est la construction d'un ordinateur
quantique à deux qubits - unité en informatique quantique, équivalent au bit classique - en 1998.
Au fur et à mesure, le nombre de qubits a augmenté et l'erreur devient de plus en plus réduite, à tel
point que Google et la Nasa annoncent en 2019 qu'ils auraient atteint la suprématie quantique.
Cette suprématie désigne le moment où un ordinateur quantique peut effectuer une tâche bien choisie en moins de temps
qu'un ordinateur classique.\\
Néanmoins, cette affirmation est très controversée et il est actuellement admis que l'affirmation
était prématurée.
Depuis cet ordinateur à 54 qubits, la recherche continue et aujourd'hui, en 2023, il existe certains
ordinateurs à plus de 400 qubits, et l'objectif d'IBM, par exemple, est d'atteindre 4000 qubits d'ici
à 2025~\cite{ibm-plans} (un processeur actuel possède des milliards de transistors - autour de 2 milliards - néanmoins
cela correspondrait à une trentaine de qubits parfaits, mais ce n'est pas du tout réaliste, car en
pratique, il faudrait peut-être plusieurs milliers de qubits réels pour un seul qubit logique~\cite{qubit-error-correction,qubit-logique}).\\ \\
On voit donc de nos jours que la recherche en informatique quantique est très active et que les
plus grands acteurs de l'informatique, comme Google, IBM, Microsoft, Intel, ou encore la Nasa, sont
très impliqués dans ce domaine.
Cette recherche mobilise de nombreuses compétences, des ingénieurs qui créent les machines,
des physiciens qui étudient les phénomènes quantiques, des mathématiciens qui développent les algorithmes,
des informaticiens qui créent les logiciels, et des chercheurs qui étudient les applications possibles.
Ces grandes entreprises cherchent aussi à attirer des jeunes dans ce domaine, en mettant à
disposition des outils de simulation, des tutoriels, des formations, et en organisant des concours.
L'accès à ces outils est aussi une manière pour les entreprises d'essayer de se garantir une place
dans ce secteur si prometteur.
Citons par exemple la création de langage de programmation spécifique comme le Q\#~\cite{microsoft-qsharp} de Microsoft~\cite{microsoft-quantum} ou le
module Qiskit~\cite{qiskit-web} d'IBM~\cite{ibm-quantum-web}, en plus des solutions open sources existantes, ou la mise à disposition de
simulateurs ou d'ordinateur quantique réel à des gens qui ne sont pas dans la recherche.\\ \\
Ce travail vise à poser les bases des ordinateurs quantiques, ainsi que d'en présenter
certains exemples qui montrent l'origine de l'intérêt pour cette technologie.
Il faut malgré tout garder à l'esprit que cela demeure un domaine très jeune et qu'il demeure principalement
théorique, et que les machines existantes actuellement sont encore soumis à des erreurs importantes.
