\chapter{Algorithme de Deutsch-Jozsa}\label{ch:algorithme-de-deutsch-jozsa}

Premièrement introduit par David Deutsch et Richard Jozsa~\cite{deutsch-josza-ori}, c'est un des premiers
algorithmes quantiques qui montre un avantage exponentiel sur un algorithme classique.
Il en demeure un faible intérêt pratique, mais il est important pour comprendre
comment exploiter la superposition et l'interférence quantique.\\ \\
Le problème de Deutsch-Jozsa est une histoire de boites noires.
On a une boite noire qui prend en entrée un certain nombre de bits et qui renvoie
un bit.
On sait que cette boite noire est soit constante, soit équilibrée, i.e.\ soit elle
renvoie toujours 0, soit elle renvoie 0 pour la moitié des entrées et 1 pour l'autre
moitié.
Analytiquement, on peut dénoter la boite noire par une fonction $f:\{0,1\}^n \rightarrow \{0,1\}$.
Le but de l'algorithme est de déterminer si la boite noire est constante ou équilibrée.\\
Commençons par l'algorithme classique.
Afin d'être sûr de notre réponse, il faut tester la boite noire pour la moitié des
entrées possibles plus une.
En effet, si on a une boite noire qui renvoie 0 pour la moitié des entrées et 1 pour
l'autre moitié, il est possible que les entrées testées soient toutes dans la moitié
qui renvoie 0.
De fait, tant que cela n'est pas fait, on ne peut pas conclure.
Le nombre d'entrées à tester est donc $2^{n-1}+1$ pour $n$ le nombre d'entrées, ce qui
est donc une complexité exponentielle, $\order{2^n}$.
\begin{figure}[H]
    \centering
    \begin{tikzpicture}[
    %Environment config
        font=\sffamily,
        thick,
    %Environment styles
        GateCfg/.style={
            logic gate inputs={normal,normal,normal},
            draw,
            scale=1
        }
    ]
        \node (x) at (0,0) {$1$};
        \node (y) at (0,1) {$0$};
        \draw (x) -- (1,0);
        \draw (y) -- (1,1);
        \node (f1) at (1.2,0) {$f$};
        \node (f2) at (1.2,1) {$f$};
        \draw[draw=black] (1,-0.3) rectangle (1.4,0.3);
        \draw[draw=black] (1,0.7) rectangle (1.4,1.3);
        \node[xor gate US, draw, logic gate inputs={normal,normal},scale=1] (xor) at (3,0.5) {};
        \draw (f1) -- (xor.input 2);
        \draw (f2) -- (xor.input 1);
        \draw (xor.output) -- (4,0.5) node[right] {$s$};
    \end{tikzpicture}
    \caption{Résolution classique du problème de Deutsch-Jozsa, $n=2$ (si $f(0) = f(1)$, alors $s = 0$, sinon $s = 1$)}
    \label{fig:clas-dj}
\end{figure}
La version quantique de l'algorithme que nous allons présenter n'est pas la version
originale de Deutsch et Jozsa, mais une version améliorée par Cleve, Ekert, Macchiavello
et Mosca~\cite{deutsch-jozsa-discu}.\\
L'idée de l'algorithme~\cite{wiki:deutsch-josza} est de mettre tous les qubits en superposition, en notant que
celui qui percevra la réponse de la boite noire doit être initialement à 1.
Ensuite, on applique la boite noire sur tous les qubits.
Cela va avoir pour effet de modifier tous les qubits, comme la base n'est plus la base
computationnelle, mais la base d'Hadamard.
En revenant dans la base computationnelle, via une autre transformation d'Hadamard, car
elle est sa propre inverse, la mesure des qubits d'entrée nous donnera la réponse quant
à la nature de la boite noire.
\begin{figure}[H]
    \[\shorthandoff{!}
    \scalebox{1.0}{
        \Qcircuit @C=1.0em @R=0.2em @!R { \\
        \nghost{{entrée} :  } & \lstick{{entrée} :  } & \qw \barrier[0em]{1} & \qw & \gate{\mathrm{H}} & \multigate{1}{\mathrm{Oracle}}_<<<{0} & \gate{\mathrm{H}} & \meter & \qw & \qw\\
        \nghost{{sortie} :  } & \lstick{{sortie} :  } & \gate{\mathrm{X}} & \qw & \gate{\mathrm{H}} & \ghost{\mathrm{Oracle}}_<<<{1} & \qw & \qw & \qw & \qw\\
        \nghost{\mathrm{{mesure} :  }} & \lstick{\mathrm{{mesure} :  }} & \lstick{/_{_{1}}} \cw & \cw & \cw & \cw & \cw & \dstick{_{_{\hspace{0.0em}0}}} \cw \ar @{<=} [-2,0] & \cw & \cw\\
        \\ }}
    \]
    \caption{Algorithme de Deutsch}
    \label{fig:alg-deutsch}
\end{figure}
La figure~\ref{fig:alg-deutsch} montre l'algorithme de Deutsch, qui est la version
à une entrée de l'algorithme de Deutsch-Jozsa.
On considère que l'oracle applique la fonction $f$ sur les qubits d'entrée et renvoie
le résultat sur le qubit de sortie sans créer de décohérence.\\
Pour résoudre le problème de manière classique, on voit qu'il faut appeler la fonction
deux fois, une fois avec 0 en entrée et une fois avec 1 en entrée.
Dans ce schéma, on voit que l'on appelle la fonction une seule fois, qui est déjà un
avantage.\\
On commence donc après la barrière avec les qubits dans l'état $\ket{0}\ket{1}$, qui
deviennent $\frac{1}{2}(\ket{0}+\ket{1})(\ket{0}-\ket{1})$ après l'application de la
première porte d'Hadamard sur chacun des qubits.
Ensuite, on applique l'oracle, qui va modifier les qubits d'entrée en fonction de la
fonction $f$ comme suit :
\[
    \ket{x}\ket{y} \rightarrow \ket{x}\ket{f(x) \oplus y}
\]
En l'appliquant sur le circuit, on obtient :
\[
    \frac{1}{2}(\ket{0} (\ket{f(0) \oplus 0}-\ket{f(0) \oplus 1}) + \ket{1}(\ket{f(1) \oplus 0}-\ket{f(1) \oplus 1}))
\]
En séparant les cas, on observe que
\[
    \begin{cases}
        f(0) = 0 \Rightarrow f(0) \oplus 0 = 0 \text{ et } f(0) \oplus 1 = 1 \Rightarrow (\ket{f(0) \oplus 0}-\ket{f(0) \oplus 1}) = (\ket{0}-\ket{1})\\
        f(0) = 1 \Rightarrow f(0) \oplus 0 = 1 \text{ et } f(0) \oplus 1 = 0 \Rightarrow (\ket{f(0) \oplus 0}-\ket{f(0) \oplus 1}) = (\ket{1}-\ket{0}) = -(\ket{0}-\ket{1})
    \end{cases}
\]
et similairement pour $f(1)$.\\
On peut donc réécrire $\ket{f(0) \oplus 0}-\ket{f(0) \oplus 1}$ comme $(-1)^{f(0)}(\ket{0}-\ket{1})$
et $\ket{f(1) \oplus 0}-\ket{f(1) \oplus 1}$ comme $(-1)^{f(1)}(\ket{0}-\ket{1})$,
ce qui nous donne :
\[
    \frac{1}{2}((-1)^{f(0)} \ket{0} (\ket{0}-\ket{1}) + (-1)^{f(1)} \ket{1}(\ket{0}-\ket{1}))
\]
Mettons ensuite $(-1)^{f(0)}$ en évidence, et comme $(-1)^n$ alterne, on peut écrire
$(-1)^{f(1) - f(0)} = (-1)^{f(0) \oplus f(1)}$, ainsi que $(\ket{0}-\ket{1})$ mis en évidence :
\[
    (-1)^{f(0)} \frac{1}{2} (\ket{0} + (-1)^{f(0) \oplus f(1)} \ket{1})(\ket{0}-\ket{1})
\]
L'opérateur $\oplus$, qui est l'addition modulo 2, est équivalent à l'opérateur XOR, comme évoqué dans la section~\ref{ch:un-ordinateur-classique}.
Or, si $f(0) = f(1)$, l'oracle est constant, ce qui est équivalent à dire que $f(0) \oplus f(1) = 0$.
Essayons donc d'extraire cette information du circuit.\\
Pour la suite on ne se soucie plus du qubit de sortie, car on le voit dans l'équation,
ce n'est pas lui qui va nous donner l'information.
De plus, la phase globale $(-1)^{f(0)}$ n'est pas observable, car elle ne change pas
la probabilité de mesure, on peut donc l'ignorer également.
Nous ne soucions donc plus que des qubits d'entrée, et on peut réécrire l'équation
comme suit :
\[
    \frac{1}{\sqrt{2}} (\ket{0} + (-1)^{f(0) \oplus f(1)} \ket{1})
\]
qui devient ensuite par la porte d'Hadamard :
\[
    \frac{1}{2} (\ket{0} + \ket{1} + (-1)^{f(0) \oplus f(1)} \ket{0} - (-1)^{f(0) \oplus f(1)} \ket{1}) = \frac{1}{2} ((1 + (-1)^{f(0) \oplus f(1)}) \ket{0} + (1 - (-1)^{f(0) \oplus f(1)}) \ket{1})
\]
On constate alors que si $f(0) \oplus f(1) = 0$, alors la probabilité de mesure de
$\ket{0}$ est de 1, et si $f(0) \oplus f(1) = 1$, alors la probabilité de mesure de
$\ket{1}$ est de 1, ainsi, en mesurant le qubit d'entrée, on peut déterminer si
l'oracle est constant ou équilibré.\\
On peut démontrer le même résultat en prenant $n$ entrées, et en mesurant les $n$
qubits d'entrée, on peut déterminer si l'oracle est constant ou équilibré.
Néanmoins, on ne le fera pas ici, car c'est sensiblement la même démonstration à part
que l'on traite des sommes sur toutes les entrées plutôt que des valeurs plus concrètes
comme dans ce cas-ci.\\ \\
Il est amusant de l'essayer avec un circuit réel, comme celui de la figure~\ref{fig:alg-deutsch},
en utilisant comme oracle constant aucune porte, et comme oracle équilibré une porte
$CX$ avec comme contrôle l'entrée et comme cible le qubit de sortie.
On obtient après exécution de l'algorithme 1024 fois des résultats satisfaisants,
soit 51 mesures de 0 pour 973 mesures de 1 pour l'oracle équilibré, et 1017 mesures
de 0 pour 7 mesures de 1 pour l'oracle constant.
Cela correspond bien à ce que l'on attendait, néanmoins les calculs prédisaient un résultat
parfait, ce qui n'est pas le cas ici, mais cela est dû à d'autres facteurs propres
aux ordinateurs quantiques, comme les erreurs de portes, les erreurs de mesure, etc.\\ \\
Ce circuit présente une première idée de ce que permettent la superposition et l'interférence
quantique.
On voit dans cet algorithme que l'on peut traiter plusieurs cas en même temps, et
que l'on peut ensuite extraire l'information de manière constructive.
Cela permet aussi de voir que par rapport à un ordinateur classique, on peut avoir
un temps constant pour résoudre un problème, alors que l'ordinateur classique a besoin
d'un temps exponentiel, comme c'est déjà visible pour le cas de deux entrées, où on
voit que sur la figure~\ref{fig:clas-dj} on a besoin de deux appels de la fonction
pour résoudre le problème, alors que l'algorithme quantique n'en a besoin que d'un
(cf. figure~\ref{fig:alg-deutsch}).
Cela peut être significatif si la fonction est coûteuse à évaluer, et quand on a plus
de deux entrées, cela devient un appel pour le quantique et $2^{n-1}+1$ appels pour
le classique.
De fait, cela peut apparaître comme un avantage non-négligeable sur des problèmes
plus complexes.
