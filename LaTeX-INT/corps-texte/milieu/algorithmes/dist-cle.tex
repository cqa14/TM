\chapter{Cryptographie : distribution de clés}\label{ch:cryptographie-:-distribution-de-cles}

On a vu dans le chapitre précédent que certains systèmes à clé publique seraient mis en danger
par l'existence d'un ordinateur quantique.
Cependant, il existe aussi des avantages à utiliser les propriétés quantiques de la matière
pour sécuriser des communications.\\
Par exemple, on peut utiliser la polarisation de photons pour envoyer des qubits d'information
via des fibres optiques, ce qui ferait un canal de communication quantique.
Cela peut être utilisé pour distribuer des clés de façon sécurisée, c'est-à-dire sans que
quelqu'un puisse écouter la communication et que cela soit détecté, ce qui est impossible
avec des moyens classiques.
Ces clés seront ensuite secrètes et pourront être utilisées pour chiffrer des messages
avec un algorithme symétrique classique.\\ \\
On va prendre nos acteurs habituels pour expliquer le principe de la distribution de clés quantique.
Donc Alice et Bob veulent se partager une clé secréte, et Eve veut écouter la communication.
Le protocole compte sur le fait que la mesure d'un qubit modifie son état, et donc que
si Eve écoute la communication, Alice et Bob le sauront.\\
Avec le circuit ci-dessous, on peut voir que sans interception, Alice met son qubit dans
un état superposé, et Bob le remet dans la base standard avant de le mesurer.
\begin{figure}[H]
    \centering
    \[\shorthandoff{!}
    \scalebox{1.0}{
        \Qcircuit @C=1.0em @R=0.2em @!R { \\
        \nghost{{q} :  } & \lstick{{q} :  } & \gate{\mathrm{H}} \barrier[0em]{0} & \qw & \gate{\mathrm{H}} & \meter & \qw & \qw\\
        \nghost{\mathrm{{c} :  }} & \lstick{\mathrm{{c} :  }} & \lstick{/_{_{1}}} \cw & \cw & \cw & \dstick{_{_{\hspace{0.0em}0}}} \cw \ar @{<=} [-1,0] & \cw & \cw\\
        \\ }}
    \]
    \caption{Sans interception (Alice | Bob)}
    \label{fig:wo-interception}
\end{figure}
Mais si Eve intercepte le qubit, elle va effectuer une mesure sur le qubit, ce qui va
le faire passer dans la base standard, et donc Bob va le remettre en superposition avant
la mesure et le résultat sera aléatoire.
\begin{figure}[H]
    \centering
    \[\shorthandoff{!}
    \scalebox{1.0}{
        \Qcircuit @C=1.0em @R=0.2em @!R { \\
        \nghost{{q} :  } & \lstick{{q} :  } & \gate{\mathrm{H}} \barrier[0em]{0} & \qw & \meter \barrier[0em]{0} & \qw & \gate{\mathrm{H}} & \meter & \qw & \qw\\
        \nghost{\mathrm{{c} :  }} & \lstick{\mathrm{{c} :  }} & \lstick{/_{_{1}}} \cw & \cw & \dstick{_{_{\hspace{0.0em}0}}} \cw \ar @{<=} [-1,0] & \cw & \cw & \dstick{_{_{\hspace{0.0em}0}}} \cw \ar @{<=} [-1,0] & \cw & \cw\\
        \\ }}
    \]
    \caption{Avec interception (Alice | Eve | Bob)}
    \label{fig:w-interception}
\end{figure}
Sachant cela, on peut établir un protocole de distribution de clés quantique comme suit :
\begin{enumerate}
    \item Alice choisit une liste de bits aléatoires (par exemple : 01001), et pour chaque bit,
        elle choisit une base de mesure aléatoire (standard ou Hadamard, p.ex. : ZZXXZ).
        Elle garde ces informations pour elle.
    \item Alice encode ensuite chaque bit de sa liste dans un qubit, en utilisant la base
        correspondante (ce qui donne avec les exemples précédant : $\ket{0}\ket{1}\ket{+}\ket{+}\ket{1}$).
        Elle envoie ensuite les qubits à Bob.
    \item Bob mesure alors chaque qubit aléatoirement dans une base standard ou Hadamard (p.ex. : ZXZXZ).
        Il garde les résultats pour lui.
    \item Bob et Alice partagent alors les bases utilisées pour chaque qubit, et jettent les
        qubits pour lesquels ils n'ont pas utilisé la même base.
        Ceux pour lesquels ils ont utilisé la même base sont alors utilisés pour générer la clé (dans notre exemple : 001).
    \item Pour vérifier que personne n'a écouté la communication, Alice et Bob prennent une
        partie de la clé, et la comparent.
        Si elles sont identiques, alors personne n'a écouté la communication.
        Sinon, ils recommencent le protocole.
        En pratique, c'est un poil plus complexe, car il faut prendre en compte le fait que
        les qubits peuvent être différents simplement à cause du bruit, et en pratique, on
        doit le prendre en compte dans le partage de la clé et dans la vérification.
\end{enumerate}
On voit donc par cela que l'utilisation de canaux de communication quantique permettrait
de distribuer des clés de façon sécurisée, et donc de chiffrer des messages de manière
plus sûre.
Les technologies quantiques ne se limitent donc pas à la puissance de calcul, mais peuvent
aussi être utilisées dans de nombreux autres domaines, comme les communications.
