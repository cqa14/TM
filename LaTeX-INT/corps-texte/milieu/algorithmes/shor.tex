\chapter{Algorithme de Shor}\label{ch:algorithme-de-shor}

L'algorithme de Shor fut développé en 1994 par Peter Shor~\cite{shor-article}.
Il permet de factoriser un nombre entier $N$ en un produit de deux nombres premiers avec
une complexité temporelle en $\order{(\log N)^3}$ et spatiale en $\order{\log N}$.
En comparaison, de manière classique, on est supérieur à $\order{N^k}$, plus autour
de $\order{e^{N}}$~\cite{wiki:int-fact}.
L'intérêt de cet algorithme est donc évident pour la cryptographie, qui s'appuie sur cette
complexité afin de garantir la sécurité des communications, via des codes comme RSA\@.
Néanmoins, son impact est quand même à modérer, car même s'il revenait à être implémenté,
il existe d'autres méthodes de cryptographie dite post-quantiques, qui ne sont pas sensibles
à ces découvertes.

\section{Principe}\label{sec:principe}

L'algorithme de Shor~\cite{wiki:shor} se base sur une partie classique, et une partie quantique.
La partie classique est la génération d'un nombre aléatoire $a$ inférieur à $N$.
Ensuite calculer le plus grand diviseur commun de $a$ et $N$, si celui-ci est différent de 1,
alors on a trouvé un facteur de $N$.
Sinon, on passe à la partie quantique.
Le calcul du PGDC peut se faire via l'algorithme d'Euclide par exemple.

\begin{algorithm}[H]
    \caption{Algorithm d'Euclide}
    \label{alg:euclide}

    \Entree{Deux entiers $a$ et $b$}
    \Sortie{Le PGDC de $a$ et $b$}
    \Tq{$a \neq b$}{
        \eSi{$a > b$}{
            $a \gets a-b$ \;
        }{
            $b \gets b-a$ \;
        }
    }
    \Retour $a$\;
\end{algorithm}

La partie quantique vise à calculer la période d'une fonction $f(x) = a^x \mod N$,
autrement dit, le plus petit entier $r$ tel que $f(x+r) = f(x)$.
Si $r$ est impair, ou que $a^{r/2} \mod N = -1$, alors on recommence avec un autre $a$.
Sinon, on peut tirer les conclusions suivantes :
\begin{enumerate}
    \item Comme pour $x=0$, $f(0) = a^0 \mod N = 1$ donc $f(0 + r)=f(r)=1$, on obtient $a^r \mod N = 1$
    \item Cela implique $a^r - 1 \mod N = 0$, ainsi $N$ divise $a^r - 1$.
    \item Comme $r$ est pair, on peut écrire $a^r - 1 = (a^{r/2} - 1)(a^{r/2} + 1)$.
    \item Comme $r$ est la plus petite période, $a^{r/2} - 1$ n'est pas divisible par $N$.
    On peut donc calculer le PGDC de $a^{r/2} - 1$ et $N$ afin de trouver un facteur de $N$.
    Si celui-ci vaut 1, alors on recommence avec un autre $a$, et sinon, on a réussi.
\end{enumerate}
Il a été prouvé que cette méthode devrait fonctionner après quelques essais.\\ \\
La partie quantique est principalement une estimation de phase, voir~\ref{sec:estimation-de-phase-quantique}.
L'astuce consiste à utiliser un opérateur $U$ tel que $U\ket{x} = \ket{ax \mod N}$.
Dans ce cas, nous avons $U^n\ket{x} = \ket{a^n x \mod N}$.\\
Deux problèmes se posent alors, premièrement, il faut un vecteur propre de $U$ afin de pouvoir
utiliser l'algorithme d'estimation de phase, et deuxièmement, il faut pouvoir implémenter $U$.
Le premier problème peut être résolu en définissant des états comme suit :
\[
    \ket{u_s} = \frac{1}{\sqrt{r}} \sum_{k=0}^{r-1} e^{-\frac{2 \pi i s k}{r}} \ket{a^k \mod N}
\]
qui sont des vecteurs propres de $U$ associés à la valeur propre $e^{\frac{2 \pi i s}{r}}$.
On peut s'en convaincre car dans l'anneau $\mathbb{Z}_N$, on a $a^r = a^0 = 1$ et appliquer
$U \ket{u_s}$ revient globalement à faire la somme de $k=1$ à $r$ mais par la remarque précédente,
c'est équivalent à faire la somme de $k=0$ à $r-1$.
De plus, la somme de vecteurs propres de $U$ est aussi un vecteur propre de $U$.
Si l'on choisit de faire
\[
    \frac{1}{\sqrt{r}} \sum_{s=0}^{r-1} \ket{u_s} = \ket{1}
\]
car les différentes phases s'annulent.
De fait, nous avons un vecteur propre de $U$.\\
Notons via cela que la phase obtenue sera $\theta = \frac{s}{r}$, et que pour déduire $r$ il faudra
transformer la phase en une fraction avec le dénominateur inférieur à $N$, et celui-ci sera $r$.\\
Le second problème est de construire $U$.
Pour ce faire on peut implémenter l'algorithme classique pour l'exponentiation modulaire,
de complexité polynomiale, néanmoins comme la construction concrète n'est pas évidente et
demeure la raison principale de ralentissement de l'algorithme, nous ne la détaillerons pas ici.
Nous utiliserons celle construite dans le module \texttt{Qiskit}.
Notons que nous utilisons à chaque fois des portes $U^{2^j}$, sachant que si $a \mod N = b$, alors
$a^c \mod N = b^c$.

\begin{algorithm}[H]
    \caption{Exponentiation modulaire}
    \label{alg:exp-mod}

    \Entree{Trois entiers $a$, $j$ et $N$}
    \Sortie{$a^{2^j} \mod N$}
    \Pour{$i \gets 1$ à $j$}{
        $a \gets a^2 \mod N$ \;
    }
    \Retour $a$\;
\end{algorithm}

\section{Implémentation simple}\label{sec:implementation-simple}

Pour montrer le fonctionnement de l'algorithme, nous allons implémenter une version simple
de celui-ci, afin de factoriser $N=15$ à partir de $a=7$.
\begin{figure}[H]
    \centering
    \[\shorthandoff{!}
    \scalebox{0.45}{
        \Qcircuit @C=1.0em @R=0.2em @!R { \\
        \nghost{{q}_{0} :  } & \lstick{{q}_{0} :  } & \gate{\mathrm{H}} & \ctrl{8} & \qw & \qw & \qw & \qw & \qw & \qw & \qw & \multigate{7}{\mathrm{QFT^\dagger}}_<<<{0} & \meter & \qw & \qw & \qw & \qw & \qw & \qw & \qw & \qw & \qw\\
        \nghost{{q}_{1} :  } & \lstick{{q}_{1} :  } & \gate{\mathrm{H}} & \qw & \ctrl{7} & \qw & \qw & \qw & \qw & \qw & \qw & \ghost{\mathrm{QFT^\dagger}}_<<<{1} & \qw & \meter & \qw & \qw & \qw & \qw & \qw & \qw & \qw & \qw\\
        \nghost{{q}_{2} :  } & \lstick{{q}_{2} :  } & \gate{\mathrm{H}} & \qw & \qw & \ctrl{6} & \qw & \qw & \qw & \qw & \qw & \ghost{\mathrm{QFT^\dagger}}_<<<{2} & \qw & \qw & \meter & \qw & \qw & \qw & \qw & \qw & \qw & \qw\\
        \nghost{{q}_{3} :  } & \lstick{{q}_{3} :  } & \gate{\mathrm{H}} & \qw & \qw & \qw & \ctrl{5} & \qw & \qw & \qw & \qw & \ghost{\mathrm{QFT^\dagger}}_<<<{3} & \qw & \qw & \qw & \meter & \qw & \qw & \qw & \qw & \qw & \qw\\
        \nghost{{q}_{4} :  } & \lstick{{q}_{4} :  } & \gate{\mathrm{H}} & \qw & \qw & \qw & \qw & \ctrl{4} & \qw & \qw & \qw & \ghost{\mathrm{QFT^\dagger}}_<<<{4} & \qw & \qw & \qw & \qw & \meter & \qw & \qw & \qw & \qw & \qw\\
        \nghost{{q}_{5} :  } & \lstick{{q}_{5} :  } & \gate{\mathrm{H}} & \qw & \qw & \qw & \qw & \qw & \ctrl{3} & \qw & \qw & \ghost{\mathrm{QFT^\dagger}}_<<<{5} & \qw & \qw & \qw & \qw & \qw & \meter & \qw & \qw & \qw & \qw\\
        \nghost{{q}_{6} :  } & \lstick{{q}_{6} :  } & \gate{\mathrm{H}} & \qw & \qw & \qw & \qw & \qw & \qw & \ctrl{2} & \qw & \ghost{\mathrm{QFT^\dagger}}_<<<{6} & \qw & \qw & \qw & \qw & \qw & \qw & \meter & \qw & \qw & \qw\\
        \nghost{{q}_{7} :  } & \lstick{{q}_{7} :  } & \gate{\mathrm{H}} & \qw & \qw & \qw & \qw & \qw & \qw & \qw & \ctrl{1} & \ghost{\mathrm{QFT^\dagger}}_<<<{7} & \qw & \qw & \qw & \qw & \qw & \qw & \qw & \meter & \qw & \qw\\
        \nghost{{q}_{8} :  } & \lstick{{q}_{8} :  } & \gate{\mathrm{X}} & \multigate{3}{\mathrm{7\string^1\,mod\,15}}_<<<{0} & \multigate{3}{\mathrm{7\string^2\,mod\,15}}_<<<{0} & \multigate{3}{\mathrm{7\string^4\,mod\,15}}_<<<{0} & \multigate{3}{\mathrm{7\string^8\,mod\,15}}_<<<{0} & \multigate{3}{\mathrm{7\string^16\,mod\,15}}_<<<{0} & \multigate{3}{\mathrm{7\string^32\,mod\,15}}_<<<{0} & \multigate{3}{\mathrm{7\string^64\,mod\,15}}_<<<{0} & \multigate{3}{\mathrm{7\string^128\,mod\,15}}_<<<{0} & \qw & \qw & \qw & \qw & \qw & \qw & \qw & \qw & \qw & \qw & \qw\\
        \nghost{{q}_{9} :  } & \lstick{{q}_{9} :  } & \qw & \ghost{\mathrm{7\string^1\,mod\,15}}_<<<{1} & \ghost{\mathrm{7\string^2\,mod\,15}}_<<<{1} & \ghost{\mathrm{7\string^4\,mod\,15}}_<<<{1} & \ghost{\mathrm{7\string^8\,mod\,15}}_<<<{1} & \ghost{\mathrm{7\string^16\,mod\,15}}_<<<{1} & \ghost{\mathrm{7\string^32\,mod\,15}}_<<<{1} & \ghost{\mathrm{7\string^64\,mod\,15}}_<<<{1} & \ghost{\mathrm{7\string^128\,mod\,15}}_<<<{1} & \qw & \qw & \qw & \qw & \qw & \qw & \qw & \qw & \qw & \qw & \qw\\
        \nghost{{q}_{10} :  } & \lstick{{q}_{10} :  } & \qw & \ghost{\mathrm{7\string^1\,mod\,15}}_<<<{2} & \ghost{\mathrm{7\string^2\,mod\,15}}_<<<{2} & \ghost{\mathrm{7\string^4\,mod\,15}}_<<<{2} & \ghost{\mathrm{7\string^8\,mod\,15}}_<<<{2} & \ghost{\mathrm{7\string^16\,mod\,15}}_<<<{2} & \ghost{\mathrm{7\string^32\,mod\,15}}_<<<{2} & \ghost{\mathrm{7\string^64\,mod\,15}}_<<<{2} & \ghost{\mathrm{7\string^128\,mod\,15}}_<<<{2} & \qw & \qw & \qw & \qw & \qw & \qw & \qw & \qw & \qw & \qw & \qw\\
        \nghost{{q}_{11} :  } & \lstick{{q}_{11} :  } & \qw & \ghost{\mathrm{7\string^1\,mod\,15}}_<<<{3} & \ghost{\mathrm{7\string^2\,mod\,15}}_<<<{3} & \ghost{\mathrm{7\string^4\,mod\,15}}_<<<{3} & \ghost{\mathrm{7\string^8\,mod\,15}}_<<<{3} & \ghost{\mathrm{7\string^16\,mod\,15}}_<<<{3} & \ghost{\mathrm{7\string^32\,mod\,15}}_<<<{3} & \ghost{\mathrm{7\string^64\,mod\,15}}_<<<{3} & \ghost{\mathrm{7\string^128\,mod\,15}}_<<<{3} & \qw & \qw & \qw & \qw & \qw & \qw & \qw & \qw & \qw & \qw & \qw\\
        \nghost{\mathrm{{c} :  }} & \lstick{\mathrm{{c} :  }} & \lstick{/_{_{8}}} \cw & \cw & \cw & \cw & \cw & \cw & \cw & \cw & \cw & \cw & \dstick{_{_{\hspace{0.0em}0}}} \cw \ar @{<=} [-12,0] & \dstick{_{_{\hspace{0.0em}1}}} \cw \ar @{<=} [-11,0] & \dstick{_{_{\hspace{0.0em}2}}} \cw \ar @{<=} [-10,0] & \dstick{_{_{\hspace{0.0em}3}}} \cw \ar @{<=} [-9,0] & \dstick{_{_{\hspace{0.0em}4}}} \cw \ar @{<=} [-8,0] & \dstick{_{_{\hspace{0.0em}5}}} \cw \ar @{<=} [-7,0] & \dstick{_{_{\hspace{0.0em}6}}} \cw \ar @{<=} [-6,0] & \dstick{_{_{\hspace{0.0em}7}}} \cw \ar @{<=} [-5,0] & \cw & \cw\\
        \\ }}
    \]
    \caption{Algorithmes de Shor pour $N=15$ et $a=7$}
    \label{fig:shor-15-7}
\end{figure}
Nous distinguons deux parties au circuit.
La première partie, en haut, est le même que pour l'estimation de phase, la partie avec la mesure.
La seconde partie, en bas, est initialisée dans l'état $\ket{1}$, et est composée de l'application
successive des portes $U$, et notons également que nous utilisons dans ce cas une porte $U$
spécialisée pour $a=7$ et $N=15$, néanmoins, il est possible de construire une porte $U$ pour
n'importe quel $a$ et $N$ en prenant quelques considérations supplémentaires (principalement
la taille des deux nombres).

\subsection{Simulation}\label{subsec:simulation}

La simulation du circuit fait apparaître, dans ce cas, quatre valeurs possibles pour la mesure
de la partie supérieure du circuit.
\begin{figure}[H]
    \centering
    \import{images/algo/shor/}{shor_plot_sim.tex}
    \caption{Simulation du circuit de Shor pour $N=15$ et $a=7$}
    \label{fig:shor-15-7-sim}
\end{figure}
En convertissant en décimal, nous obtenons les valeurs 0, 64, 128 et 192.
En divisant par $2^8 = 256$, on obtient des phases de 0, 0.25, 0.5 et 0.75 qui correspondent
aux fractions $\frac{0}{1}$, $\frac{1}{4}$, $\frac{1}{2}$ et $\frac{3}{4}$.
De fait, deux mesures nous donnent la bonne valeur de $r = 4$, car l'algorithme de Shor
peut échouer si $s=0$ ou $s$ et $r$ ne sont pas premiers entre eux, où il faut alors essayer d'amplifier
les fractions également.\\ \\
En effet, $a^{r/2} - 1 = 7^{4/2} - 1 = 49 - 1 = 48$ et ensuite le PGDC de $48$ et $15$ est $3$,
qui est un facteur de $15$.

\subsection{Hardware réel}\label{subsec:hardware-reel}

Nous avons également exécuté le circuit sur un hardware réel, en prenant , et nous avons obtenu les
résultats suivants.
\begin{figure}[H]
    \centering
    \[\shorthandoff{!}
        \scalebox{1.0}{
            \Qcircuit @C=1.0em @R=0.2em @!R { \\
            \nghost{{q}_{0} :  } & \lstick{{q}_{0} :  } & \gate{\mathrm{H}} & \ctrl{3} & \qw & \qw & \multigate{2}{\mathrm{QFT^\dagger}}_<<<{0} & \meter & \qw & \qw & \qw & \qw\\
            \nghost{{q}_{1} :  } & \lstick{{q}_{1} :  } & \gate{\mathrm{H}} & \qw & \ctrl{2} & \qw & \ghost{\mathrm{QFT^\dagger}}_<<<{1} & \qw & \meter & \qw & \qw & \qw\\
            \nghost{{q}_{2} :  } & \lstick{{q}_{2} :  } & \gate{\mathrm{H}} & \qw & \qw & \ctrl{1} & \ghost{\mathrm{QFT^\dagger}}_<<<{2} & \qw & \qw & \meter & \qw & \qw\\
            \nghost{{q}_{3} :  } & \lstick{{q}_{3} :  } & \gate{\mathrm{X}} & \multigate{3}{\mathrm{7\string^1\,mod\,15}}_<<<{0} & \multigate{3}{\mathrm{7\string^2\,mod\,15}}_<<<{0} & \multigate{3}{\mathrm{7\string^4\,mod\,15}}_<<<{0} & \qw & \qw & \qw & \qw & \qw & \qw\\
            \nghost{{q}_{4} :  } & \lstick{{q}_{4} :  } & \qw & \ghost{\mathrm{7\string^1\,mod\,15}}_<<<{1} & \ghost{\mathrm{7\string^2\,mod\,15}}_<<<{1} & \ghost{\mathrm{7\string^4\,mod\,15}}_<<<{1} & \qw & \qw & \qw & \qw & \qw & \qw\\
            \nghost{{q}_{5} :  } & \lstick{{q}_{5} :  } & \qw & \ghost{\mathrm{7\string^1\,mod\,15}}_<<<{2} & \ghost{\mathrm{7\string^2\,mod\,15}}_<<<{2} & \ghost{\mathrm{7\string^4\,mod\,15}}_<<<{2} & \qw & \qw & \qw & \qw & \qw & \qw\\
            \nghost{{q}_{6} :  } & \lstick{{q}_{6} :  } & \qw & \ghost{\mathrm{7\string^1\,mod\,15}}_<<<{3} & \ghost{\mathrm{7\string^2\,mod\,15}}_<<<{3} & \ghost{\mathrm{7\string^4\,mod\,15}}_<<<{3} & \qw & \qw & \qw & \qw & \qw & \qw\\
            \nghost{\mathrm{{c5} :  }} & \lstick{\mathrm{{c5} :  }} & \lstick{/_{_{3}}} \cw & \cw & \cw & \cw & \cw & \dstick{_{_{\hspace{0.0em}0}}} \cw \ar @{<=} [-7,0] & \dstick{_{_{\hspace{0.0em}1}}} \cw \ar @{<=} [-6,0] & \dstick{_{_{\hspace{0.0em}2}}} \cw \ar @{<=} [-5,0] & \cw & \cw\\
            \\ }}
    \]
    \caption{Algorithmes de Shor pour $N=15$ et $a=7$, version à 3 qubits pour encoder l'angle}
    \label{fig:shor-15-7-reel}
\end{figure}
Afin de donner une idée de la qualité des résultats, nous avons également appliqué une
réduction d'erreur automatique, calibrée pour le hardware utilisé.
\begin{figure}[H]
    \centering
    \import{images/algo/shor/}{shor_plot_mitigated.tex}
    \caption{Execution du circuit de Shor pour $N=15$ et $a=7$ sur un hardware réel, avec en rouge les valeurs brutes, et en bleu après une atténuation d'erreur \protect\footnotemark}
    \label{fig:shor-15-7-re}
\end{figure}
\footnotetext{Executé le 31.08.2023 sur la machine 'ibm\_nairobi', \textit{job id : cjo1svhpthn588jkkphg} ; avec pour la réduction d'erreur automatique, \textit{job id : cjo1ngpdll3gjfa1lne0}}
Sur les machines à dispositions, nous voyons que les résultats sont peu encourageants, car outre la solution
triviale 000, celle qui ressort le plus est 100, qui correspond à $r=2$, parce que cela revient à $\frac{4}{2^3} = \frac{1}{2}$, hors comme dit plus haut ce n'est pas
la bonne solution, car $7^2 \mod 15 = 4 $ alors que l'on recherche $7^r \mod 15 = 1$.
Les mesures pour $r=4$ ne ressortent pas à cause du bruit, néanmoins la solution $\theta = \frac{1}{2}$ est
liée au fait que $s=2$, et donc que cela se simplifie en $\frac{2}{4} = \frac{1}{2}$.\\ \\
Une explication à l'apparition des mesures de type $b_1 b_2 1$, qui sont forcéments fausses, car cela donnerait
une fraction au dénominateur de $2^3$, qui est supérieur au $r$ recherché.
Cela est dû à l'erreur sur la porte $U^4$, qui est la porte la plus grande et donc avec le plus d'erreur.
Or dans les bons résultats, ce bit est nul, ce qui nous indique qu'en simulant le circuit cette porte devrait
avoir aucun effet, mais en l'exécutant sur un hardware réel, cela n'est pas le cas, et donc cela crée de l'erreur
de manière non négligeable, comme on le voit en comparant les résultats~\ref{fig:shor-15-7-re} et~\ref{fig:shor-15-7-re-sim}.
\begin{figure}[H]
    \centering
    \import{images/algo/shor/}{shor_plot_mitigated_sim.tex}
    \caption{Simulation du circuit de Shor pour $N=15$ et $a=7$ utilisé sur un vrai ordinateur}
    \label{fig:shor-15-7-re-sim}
\end{figure}
Sur ce dernier diagramme, on observe néanmoins effectivement les mesures 110 et 010 qui donnent après traitement $r=4$,
et donc qui sont les bonnes mesures.

\section{Application à un problème concret}\label{sec:application-a-un-probleme-concret}

La factorisation d'un nombre en un produit de deux nombres premiers est d'une importance
majeure en cryptographie.
La plupart des systèmes de cryptographie actuels reposent sur la difficulté de factoriser
les grands nombres, tel que le célèbre RSA.\\ \\
Le système RSA fonctionne comme suit~\cite{wiki:rsa}.
Prenons deux personnes Alice et Bob.
Alice crée les clés de chiffrement en choisissant deux nombres premiers $p$ et $q$ distincts.
Alors, elle calcule le nombre $n = pq$, puis $\phi (n) = (p-1)(q-1)$, où $\phi$ est la fonction
qui retourne la valeur de l'indicatrice d'Euler en $n$, soit le nombre de nombres premiers
avec $n$ inférieurs à $n$.
Cette fonction est multiplicative, c'est-à-dire que $\phi (uv) = \phi (u) \phi (v)$, et si $p$ est
premier, alors $\phi (p) = p-1$.
Après cela, Alice choisit un nombre $e$ premier avec $\phi (n)$, et calcule $d$ tel que
$ed \mod \phi (n) = 1$.\\
Alors la clé publique est le couple $(n, e)$, et la clé privée est $d$.
Pour chiffrer un message $M$ inférieur à $n$, Bob calcule $C = M^e \mod n$, et pour déchiffrer,
Alice calcule $M = C^d \mod n$.\\
Cela se démontre en utilisant le petit théorème de Fermat, qui dit que si $p$ est premier,
$M^{p-1} \mod p = 1$.
Alors, on a $C^d \mod n = (M^e)^d \mod n = M^{ed} \mod n$.
Or $ed \mod (p-1)(q-1) = 1$, ce qui est équivalent à $ed = 1 + k(p-1)(q-1)$ pour un certain
$k \in \mathbb{N}$.
De fait, d'après le petit théorème de Fermat, $M^{ed} \mod p = M^{1+k(p-1)(q-1)} \mod p = M (M^{p-1})^{k(q-1)} \mod p = M \mod p$
et de même $M^{ed} \mod q = M \mod q$.
De fait, $M^{ed} - M$ est congru à 0 modulo $p$ et $q$, et donc divisible par $p$ et $q$, et
comme $p$ et $q$ sont premiers entre eux, $M^{ed} - M$ est divisible par $pq = n$,
ce qui implique que $M^{ed} \mod n = M$.\\
Le système repose donc sur la difficulté de factoriser $n$ en $p$ et $q$, et donc de  calculer
$\phi (n)$.\\ \\
Prenons un exemple concret.
Issu du cours de mathématiques appliquées de deuxième année de monsieur Klose, gymnase Auguste
Piccard.\\ \\
\textit{On donne le couple $(E;n) = (17;143)$, ainsi que le message chiffré (en blocs)
$$14;112;49;117;17;81;53;37;49$$ Déchiffrez ce message.}\\ \\
Ce problème peut être pris comme un exemple réduit de quelqu'un voulant craquer un message
en connaissant la clé publique.\\ \\
On va donc construire un circuit qui va calculer la factorisation de $N=n=143$  via l'algorithme
de Shor.
En générant $a=12$, via le constructeur de circuits de Shor proposé par Qiskit~\cite{Qiskit,shor-imp}, on obtient
le circuit suivant.
\begin{figure}[H]
    \centering
    \[\shorthandoff{!}
    \scalebox{0.6}{
        \Qcircuit @C=1.0em @R=0.2em @!R { \\
        \nghost{{up}_{0} :  } & \lstick{{up}_{0} :  } & \gate{\mathrm{H}} & \multigate{33}{\mathrm{17\string^x\,mod\,143}}_<<<{0} & \multigate{15}{\mathrm{QFT^\dagger}}_<<<{0} & \meter & \qw & \qw & \qw & \qw & \qw & \qw & \qw & \qw & \qw & \qw & \qw & \qw & \qw & \qw & \qw & \qw & \qw\\
        \nghost{{up}_{1} :  } & \lstick{{up}_{1} :  } & \gate{\mathrm{H}} & \ghost{\mathrm{17\string^x\,mod\,143}}_<<<{1} & \ghost{\mathrm{QFT^\dagger}}_<<<{1} & \qw & \meter & \qw & \qw & \qw & \qw & \qw & \qw & \qw & \qw & \qw & \qw & \qw & \qw & \qw & \qw & \qw & \qw\\
        \nghost{{up}_{2} :  } & \lstick{{up}_{2} :  } & \gate{\mathrm{H}} & \ghost{\mathrm{17\string^x\,mod\,143}}_<<<{2} & \ghost{\mathrm{QFT^\dagger}}_<<<{2} & \qw & \qw & \meter & \qw & \qw & \qw & \qw & \qw & \qw & \qw & \qw & \qw & \qw & \qw & \qw & \qw & \qw & \qw\\
        \nghost{{up}_{3} :  } & \lstick{{up}_{3} :  } & \gate{\mathrm{H}} & \ghost{\mathrm{17\string^x\,mod\,143}}_<<<{3} & \ghost{\mathrm{QFT^\dagger}}_<<<{3} & \qw & \qw & \qw & \meter & \qw & \qw & \qw & \qw & \qw & \qw & \qw & \qw & \qw & \qw & \qw & \qw & \qw & \qw\\
        \nghost{{up}_{4} :  } & \lstick{{up}_{4} :  } & \gate{\mathrm{H}} & \ghost{\mathrm{17\string^x\,mod\,143}}_<<<{4} & \ghost{\mathrm{QFT^\dagger}}_<<<{4} & \qw & \qw & \qw & \qw & \meter & \qw & \qw & \qw & \qw & \qw & \qw & \qw & \qw & \qw & \qw & \qw & \qw & \qw\\
        \nghost{{up}_{5} :  } & \lstick{{up}_{5} :  } & \gate{\mathrm{H}} & \ghost{\mathrm{17\string^x\,mod\,143}}_<<<{5} & \ghost{\mathrm{QFT^\dagger}}_<<<{5} & \qw & \qw & \qw & \qw & \qw & \meter & \qw & \qw & \qw & \qw & \qw & \qw & \qw & \qw & \qw & \qw & \qw & \qw\\
        \nghost{{up}_{6} :  } & \lstick{{up}_{6} :  } & \gate{\mathrm{H}} & \ghost{\mathrm{17\string^x\,mod\,143}}_<<<{6} & \ghost{\mathrm{QFT^\dagger}}_<<<{6} & \qw & \qw & \qw & \qw & \qw & \qw & \meter & \qw & \qw & \qw & \qw & \qw & \qw & \qw & \qw & \qw & \qw & \qw\\
        \nghost{{up}_{7} :  } & \lstick{{up}_{7} :  } & \gate{\mathrm{H}} & \ghost{\mathrm{17\string^x\,mod\,143}}_<<<{7} & \ghost{\mathrm{QFT^\dagger}}_<<<{7} & \qw & \qw & \qw & \qw & \qw & \qw & \qw & \meter & \qw & \qw & \qw & \qw & \qw & \qw & \qw & \qw & \qw & \qw\\
        \nghost{{up}_{8} :  } & \lstick{{up}_{8} :  } & \gate{\mathrm{H}} & \ghost{\mathrm{17\string^x\,mod\,143}}_<<<{8} & \ghost{\mathrm{QFT^\dagger}}_<<<{8} & \qw & \qw & \qw & \qw & \qw & \qw & \qw & \qw & \meter & \qw & \qw & \qw & \qw & \qw & \qw & \qw & \qw & \qw\\
        \nghost{{up}_{9} :  } & \lstick{{up}_{9} :  } & \gate{\mathrm{H}} & \ghost{\mathrm{17\string^x\,mod\,143}}_<<<{9} & \ghost{\mathrm{QFT^\dagger}}_<<<{9} & \qw & \qw & \qw & \qw & \qw & \qw & \qw & \qw & \qw & \meter & \qw & \qw & \qw & \qw & \qw & \qw & \qw & \qw\\
        \nghost{{up}_{10} :  } & \lstick{{up}_{10} :  } & \gate{\mathrm{H}} & \ghost{\mathrm{17\string^x\,mod\,143}}_<<<<{10} & \ghost{\mathrm{QFT^\dagger}}_<<<<{10} & \qw & \qw & \qw & \qw & \qw & \qw & \qw & \qw & \qw & \qw & \meter & \qw & \qw & \qw & \qw & \qw & \qw & \qw\\
        \nghost{{up}_{11} :  } & \lstick{{up}_{11} :  } & \gate{\mathrm{H}} & \ghost{\mathrm{17\string^x\,mod\,143}}_<<<<{11} & \ghost{\mathrm{QFT^\dagger}}_<<<<{11} & \qw & \qw & \qw & \qw & \qw & \qw & \qw & \qw & \qw & \qw & \qw & \meter & \qw & \qw & \qw & \qw & \qw & \qw\\
        \nghost{{up}_{12} :  } & \lstick{{up}_{12} :  } & \gate{\mathrm{H}} & \ghost{\mathrm{17\string^x\,mod\,143}}_<<<<{12} & \ghost{\mathrm{QFT^\dagger}}_<<<<{12} & \qw & \qw & \qw & \qw & \qw & \qw & \qw & \qw & \qw & \qw & \qw & \qw & \meter & \qw & \qw & \qw & \qw & \qw\\
        \nghost{{up}_{13} :  } & \lstick{{up}_{13} :  } & \gate{\mathrm{H}} & \ghost{\mathrm{17\string^x\,mod\,143}}_<<<<{13} & \ghost{\mathrm{QFT^\dagger}}_<<<<{13} & \qw & \qw & \qw & \qw & \qw & \qw & \qw & \qw & \qw & \qw & \qw & \qw & \qw & \meter & \qw & \qw & \qw & \qw\\
        \nghost{{up}_{14} :  } & \lstick{{up}_{14} :  } & \gate{\mathrm{H}} & \ghost{\mathrm{17\string^x\,mod\,143}}_<<<<{14} & \ghost{\mathrm{QFT^\dagger}}_<<<<{14} & \qw & \qw & \qw & \qw & \qw & \qw & \qw & \qw & \qw & \qw & \qw & \qw & \qw & \qw & \meter & \qw & \qw & \qw\\
        \nghost{{up}_{15} :  } & \lstick{{up}_{15} :  } & \gate{\mathrm{H}} & \ghost{\mathrm{17\string^x\,mod\,143}}_<<<<{15} & \ghost{\mathrm{QFT^\dagger}}_<<<<{15} & \qw & \qw & \qw & \qw & \qw & \qw & \qw & \qw & \qw & \qw & \qw & \qw & \qw & \qw & \qw & \meter & \qw & \qw\\
        \nghost{{down}_{0} :  } & \lstick{{down}_{0} :  } & \gate{\mathrm{X}} & \ghost{\mathrm{17\string^x\,mod\,143}}_<<<<{16} & \qw & \qw & \qw & \qw & \qw & \qw & \qw & \qw & \qw & \qw & \qw & \qw & \qw & \qw & \qw & \qw & \qw & \qw & \qw\\
        \nghost{{down}_{1} :  } & \lstick{{down}_{1} :  } & \qw & \ghost{\mathrm{17\string^x\,mod\,143}}_<<<<{17} & \qw & \qw & \qw & \qw & \qw & \qw & \qw & \qw & \qw & \qw & \qw & \qw & \qw & \qw & \qw & \qw & \qw & \qw & \qw\\
        \nghost{{down}_{2} :  } & \lstick{{down}_{2} :  } & \qw & \ghost{\mathrm{17\string^x\,mod\,143}}_<<<<{18} & \qw & \qw & \qw & \qw & \qw & \qw & \qw & \qw & \qw & \qw & \qw & \qw & \qw & \qw & \qw & \qw & \qw & \qw & \qw\\
        \nghost{{down}_{3} :  } & \lstick{{down}_{3} :  } & \qw & \ghost{\mathrm{17\string^x\,mod\,143}}_<<<<{19} & \qw & \qw & \qw & \qw & \qw & \qw & \qw & \qw & \qw & \qw & \qw & \qw & \qw & \qw & \qw & \qw & \qw & \qw & \qw\\
        \nghost{{down}_{4} :  } & \lstick{{down}_{4} :  } & \qw & \ghost{\mathrm{17\string^x\,mod\,143}}_<<<<{20} & \qw & \qw & \qw & \qw & \qw & \qw & \qw & \qw & \qw & \qw & \qw & \qw & \qw & \qw & \qw & \qw & \qw & \qw & \qw\\
        \nghost{{down}_{5} :  } & \lstick{{down}_{5} :  } & \qw & \ghost{\mathrm{17\string^x\,mod\,143}}_<<<<{21} & \qw & \qw & \qw & \qw & \qw & \qw & \qw & \qw & \qw & \qw & \qw & \qw & \qw & \qw & \qw & \qw & \qw & \qw & \qw\\
        \nghost{{down}_{6} :  } & \lstick{{down}_{6} :  } & \qw & \ghost{\mathrm{17\string^x\,mod\,143}}_<<<<{22} & \qw & \qw & \qw & \qw & \qw & \qw & \qw & \qw & \qw & \qw & \qw & \qw & \qw & \qw & \qw & \qw & \qw & \qw & \qw\\
        \nghost{{down}_{7} :  } & \lstick{{down}_{7} :  } & \qw & \ghost{\mathrm{17\string^x\,mod\,143}}_<<<<{23} & \qw & \qw & \qw & \qw & \qw & \qw & \qw & \qw & \qw & \qw & \qw & \qw & \qw & \qw & \qw & \qw & \qw & \qw & \qw\\
        \nghost{{aux}_{0} :  } & \lstick{{aux}_{0} :  } & \qw & \ghost{\mathrm{17\string^x\,mod\,143}}_<<<<{24} & \qw & \qw & \qw & \qw & \qw & \qw & \qw & \qw & \qw & \qw & \qw & \qw & \qw & \qw & \qw & \qw & \qw & \qw & \qw\\
        \nghost{{aux}_{1} :  } & \lstick{{aux}_{1} :  } & \qw & \ghost{\mathrm{17\string^x\,mod\,143}}_<<<<{25} & \qw & \qw & \qw & \qw & \qw & \qw & \qw & \qw & \qw & \qw & \qw & \qw & \qw & \qw & \qw & \qw & \qw & \qw & \qw\\
        \nghost{{aux}_{2} :  } & \lstick{{aux}_{2} :  } & \qw & \ghost{\mathrm{17\string^x\,mod\,143}}_<<<<{26} & \qw & \qw & \qw & \qw & \qw & \qw & \qw & \qw & \qw & \qw & \qw & \qw & \qw & \qw & \qw & \qw & \qw & \qw & \qw\\
        \nghost{{aux}_{3} :  } & \lstick{{aux}_{3} :  } & \qw & \ghost{\mathrm{17\string^x\,mod\,143}}_<<<<{27} & \qw & \qw & \qw & \qw & \qw & \qw & \qw & \qw & \qw & \qw & \qw & \qw & \qw & \qw & \qw & \qw & \qw & \qw & \qw\\
        \nghost{{aux}_{4} :  } & \lstick{{aux}_{4} :  } & \qw & \ghost{\mathrm{17\string^x\,mod\,143}}_<<<<{28} & \qw & \qw & \qw & \qw & \qw & \qw & \qw & \qw & \qw & \qw & \qw & \qw & \qw & \qw & \qw & \qw & \qw & \qw & \qw\\
        \nghost{{aux}_{5} :  } & \lstick{{aux}_{5} :  } & \qw & \ghost{\mathrm{17\string^x\,mod\,143}}_<<<<{29} & \qw & \qw & \qw & \qw & \qw & \qw & \qw & \qw & \qw & \qw & \qw & \qw & \qw & \qw & \qw & \qw & \qw & \qw & \qw\\
        \nghost{{aux}_{6} :  } & \lstick{{aux}_{6} :  } & \qw & \ghost{\mathrm{17\string^x\,mod\,143}}_<<<<{30} & \qw & \qw & \qw & \qw & \qw & \qw & \qw & \qw & \qw & \qw & \qw & \qw & \qw & \qw & \qw & \qw & \qw & \qw & \qw\\
        \nghost{{aux}_{7} :  } & \lstick{{aux}_{7} :  } & \qw & \ghost{\mathrm{17\string^x\,mod\,143}}_<<<<{31} & \qw & \qw & \qw & \qw & \qw & \qw & \qw & \qw & \qw & \qw & \qw & \qw & \qw & \qw & \qw & \qw & \qw & \qw & \qw\\
        \nghost{{aux}_{8} :  } & \lstick{{aux}_{8} :  } & \qw & \ghost{\mathrm{17\string^x\,mod\,143}}_<<<<{32} & \qw & \qw & \qw & \qw & \qw & \qw & \qw & \qw & \qw & \qw & \qw & \qw & \qw & \qw & \qw & \qw & \qw & \qw & \qw\\
        \nghost{{aux}_{9} :  } & \lstick{{aux}_{9} :  } & \qw & \ghost{\mathrm{17\string^x\,mod\,143}}_<<<<{33} & \qw & \qw & \qw & \qw & \qw & \qw & \qw & \qw & \qw & \qw & \qw & \qw & \qw & \qw & \qw & \qw & \qw & \qw & \qw\\
        \nghost{\mathrm{{m} :  }} & \lstick{\mathrm{{m} :  }} & \lstick{/_{_{16}}} \cw & \cw & \cw & \dstick{_{_{\hspace{0.0em}0}}} \cw \ar @{<=} [-34,0] & \dstick{_{_{\hspace{0.0em}1}}} \cw \ar @{<=} [-33,0] & \dstick{_{_{\hspace{0.0em}2}}} \cw \ar @{<=} [-32,0] & \dstick{_{_{\hspace{0.0em}3}}} \cw \ar @{<=} [-31,0] & \dstick{_{_{\hspace{0.0em}4}}} \cw \ar @{<=} [-30,0] & \dstick{_{_{\hspace{0.0em}5}}} \cw \ar @{<=} [-29,0] & \dstick{_{_{\hspace{0.0em}6}}} \cw \ar @{<=} [-28,0] & \dstick{_{_{\hspace{0.0em}7}}} \cw \ar @{<=} [-27,0] & \dstick{_{_{\hspace{0.0em}8}}} \cw \ar @{<=} [-26,0] & \dstick{_{_{\hspace{0.0em}9}}} \cw \ar @{<=} [-25,0] & \dstick{_{_{\hspace{0.0em}10}}} \cw \ar @{<=} [-24,0] & \dstick{_{_{\hspace{0.0em}11}}} \cw \ar @{<=} [-23,0] & \dstick{_{_{\hspace{0.0em}12}}} \cw \ar @{<=} [-22,0] & \dstick{_{_{\hspace{0.0em}13}}} \cw \ar @{<=} [-21,0] & \dstick{_{_{\hspace{0.0em}14}}} \cw \ar @{<=} [-20,0] & \dstick{_{_{\hspace{0.0em}15}}} \cw \ar @{<=} [-19,0] & \cw & \cw\\
        \\ }}
    \]
    \caption{Algorithmes de Shor pour $N=143$ et $a=12$}
    \label{fig:shor-143-12}
\end{figure}
Un tel circuit étant trop gros pour être simulé par un ordinateur privé, on compte sur le simulateur
par produit matriciel d'IBM.
On obtient alors la sortie suivante :\\ \\
\textit{The list of factors of 143 as computed by the Shor's algorithm is [11, 13]}\\ \\
qui est bien la factorisation de 143.\\
Notons donc plusieurs choses.
Tout d'abord la difficulté de simulation de tels circuits.
En effet, on considère ici un circuit de taille 34, or le maximum envisageable pour un ordinateur
de tout un chacun est de l'ordre de 32, et la ressource nécessaire pour le simuler est exponentielle
en la taille du circuit.
De plus, l'exécution de ce circuit sur un ordinateur quantique est difficilement envisageable
car il nécessite des ordinateurs reservés à la recherche.
Ensuite, on peut noter l'utilisation de $a=12$ qui est un choix pratique, car d'autres essais
avec par exemple $a=2$ ou $a=17$ ne permettent pas d'obtenir la factorisation de $N$.\\ \\
Après avoir obtenu la factorisation de 143, il est aisé de finir le problème proposé plus haut.
On commence par calculer $\phi(143)$, qui vaut $(11-1)(13-1)=120$.
Puis on cherche $17 \cdot d \mod 120 = 1$, qui peut se calculer par exemple en utilisant
le fait que $a^{\phi (n)} \mod n = 1$ et donc l'inverse est $a^{-1} \mod n = a^{\phi (n)-1} \mod n$,
ce qui donne $17^{119} \mod 120 = 113 = d$.
En utilisant la méthode de décodage, nous obtenons alors le message suivant :
$$27; 8; 4; 13; 62; 9; 14; 20; 4$$
qui correspond selon la table alphabétique du cours en question à :\\ \\
\textit{Bien joue}\\ \\
et donc le message est bien décodé.\\ \\
Par là, on voit un avantage possible de l'ordinateur quantique, déjà dans la réalisation de
calculs avec une complexité moindre qu'un ordinateur classique, mais aussi dans la possibilité
de travailler en complémentarité avec un ordinateur classique, en utilisant l'ordinateur quantique
pour des calculs spécifiques, et l'ordinateur classique pour des calculs plus simples, dans ce cas
toute la préparation de l'algorithme par ordinateur classique, et la recherche de la période
par ordinateur quantique.
De plus, on entre dans un des domaines dans lequel l'ordinateur quantique est le plus étudié,
la cryptographie, et dans ce cas la remise en question de nos méthodes de chiffrement actuelles,
qui se base sur la complexité nécessaire à factoriser un nombre de manière classique, et qui
serait mis en danger par un ordinateur quantique.
