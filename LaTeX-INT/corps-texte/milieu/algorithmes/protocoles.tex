\chapter{Quelques protocoles}\label{ch:quelques-protocoles}

Certains concepts ont déjà été vu plus tôt, néanmoins un certain formalisme est
très utile à la bonne compréhension des capacités plus avancées de l'informatique
quantique.
Ce ne sont pas des algorithmes à proprement parler, mais plutôt des protocoles
qui servent de base à la construction d'algorithmes.

\section{Retour de phase}\label{sec:retour-de-phase}

Ce protocole est très perturbant au premier abord, mais il est en fait très
simple.
Rappelons l'algorithme de Deutsch-Jozsa vu dans le chapitre du même nom.
Lors de l'application de l'oracle, l'opération qui devait être effectuée sur la
sortie s'est retrouvée, de par la configuration du circuit, effectuée sur
l'entrée.\\
On définit un \textit{retour de phase} quand le vecteur propre d'une porte sur un qubit
est ``retourné'' sur un autre qubit via une opération de contrôle.\\ \\
Prenons l'exemple le plus simple, celui d'une porte de $CX$.
Pour rappel, la porte $CX$ est une porte de contrôle qui effectue une porte $X$
sur le qubit cible si le qubit de contrôle est à l'état $\ket{1}$, donc va inverser
le qubit cible.\\
Les figures suivantes indiquent en rouge l'état initial et en vert l'état
final.
De plus, les qubits de contrôle sont donnés avec un $_c$ et ceux cibles avec un
$_t$.

\begin{figure}[H]
    \centering
    \subfloat[Contrôle]{
        \begin{blochsphere}[radius=1.5 cm,tilt=15,rotation=-30,opacity=0]
            \drawBallGrid[style={opacity=0.1}]{30}{180}

            \drawStatePolar[axisarrow = true, statewidth = 0.3]{x-Achse}{90}{90}
            \drawStatePolar[axisarrow=true, statewidth = 0.3]{y-Achse}{90}{0}
            \drawStatePolar[axisarrow = true, statewidth = 0.3]{z-Achse}{0}{0}

            \node[below] at (x-Achse) {\fontsize{0.15cm}{1em} \large $x$};
            \node[below] at (y-Achse) {\fontsize{0.15cm}{1em} \large $y$};
            \node[below] at (z-Achse) {\fontsize{0.15cm}{1em} \large $z$};

            \drawStatePolar[statecolor = red]{State}{180}{0}
            \drawStatePolar[statecolor = green]{State}{180}{0}

            \labelLatLon{up}{90}{0};
            \labelLatLon{down}{-90}{90};
            \node[above] at (up) {$\ket{0}$};
            \node[below] at (down) {$\ket{1}$};
        \end{blochsphere}
    }
    \subfloat[Cible]{
        \begin{blochsphere}[radius=1.5 cm,tilt=15,rotation=-30,opacity=0]
            \drawBallGrid[style={opacity=0.1}]{30}{180}

            \drawStatePolar[axisarrow = true, statewidth = 0.3]{x-Achse}{90}{90}
            \drawStatePolar[axisarrow=true, statewidth = 0.3]{y-Achse}{90}{0}
            \drawStatePolar[axisarrow = true, statewidth = 0.3]{z-Achse}{0}{0}

            \node[below] at (x-Achse) {\fontsize{0.15cm}{1em} \large $x$};
            \node[below] at (y-Achse) {\fontsize{0.15cm}{1em} \large $y$};
            \node[below] at (z-Achse) {\fontsize{0.15cm}{1em} \large $z$};

            \drawStatePolar[statecolor = red]{State}{0}{0}
            \drawStatePolar[statecolor = green]{State}{180}{0}

            \labelLatLon{up}{90}{0};
            \labelLatLon{down}{-90}{90};
            \node[above] at (up) {$\ket{0}$};
            \node[below] at (down) {$\ket{1}$};
        \end{blochsphere}
    }
    \caption{Application d'une porte $CX$ sur un qubit à l'état $\ket{1}_c \otimes \ket{0}_t$}
    \label{fig:retour-de-phase-1}
\end{figure}

Maintenant, remarquons que $X$ aura le même effet sur la base $\{\ket{+}, \ket{-}\}$ que
$Z$ sur la base $\{\ket{0}, \ket{1}\}$.
De fait, $X \ket{-} = -\ket{-}$, et sachant cela, on peut experimenter avec une $CX$
appliquée sur les pairs de qubits $\ket{0}_c \otimes \ket{-}_t$ et $\ket{1}_c \otimes \ket{-}_t$ :
\begin{align*}
    \ket{0}_c \otimes \ket{-}_t &\xrightarrow{CX} \ket{0}_c \otimes \ket{-}_t \\
    \ket{1}_c \otimes \ket{-}_t &\xrightarrow{CX} \ket{1}_c \otimes X \ket{-}_t = -\ket{1}_c \otimes \ket{-}_t
\end{align*}
Dans ces cas-là, seule la phase globale est affectée, et ce n'est donc pas observable,
donc peu intéressant.
Néanmoins, si le qubit de contrôle est en superposition, alors on peut observer
des phénomènes intéressants, comme avec l'application d'une porte $CX$ sur l'état
$\ket{+}_c \otimes \ket{-}_t$ :
\[
    \ket{+}_c \otimes \ket{-}_t \xrightarrow{CX} \frac{1}{\sqrt{2}} \left( \ket{0}_c \otimes \ket{-}_t - \ket{1}_c \otimes \ket{-}_t \right)
\]
que l'on peut séparer de manière équivalente en :
\[
    \frac{1}{\sqrt{2}} \left( \ket{0}_c \otimes \ket{-}_t - \ket{1}_c \otimes \ket{-}_t \right) = \frac{1}{\sqrt{2}} \left( \ket{0}_c - \ket{1}_c \right) \otimes \ket{-}_t = \ket{-}_c \otimes \ket{-}_t
\]
La vision de cette opération sur la sphère de Bloch est très instructive.

\begin{figure}[H]
    \centering
    \subfloat[Contrôle]{
        \begin{blochsphere}[radius=1.5 cm,tilt=15,rotation=-30,opacity=0]
            \drawBallGrid[style={opacity=0.1}]{30}{180}

            \drawStatePolar[axisarrow = true, statewidth = 0.3]{x-Achse}{90}{90}
            \drawStatePolar[axisarrow=true, statewidth = 0.3]{y-Achse}{90}{0}
            \drawStatePolar[axisarrow = true, statewidth = 0.3]{z-Achse}{0}{0}

            \node[below] at (x-Achse) {\fontsize{0.15cm}{1em} \large $x$};
            \node[below] at (y-Achse) {\fontsize{0.15cm}{1em} \large $y$};
            \node[below] at (z-Achse) {\fontsize{0.15cm}{1em} \large $z$};

            \drawStatePolar[statecolor = red]{State}{90}{90}
            \drawStatePolar[statecolor = green]{State}{-90}{90}

            \labelLatLon{up}{90}{0};
            \labelLatLon{down}{-90}{90};
            \node[above] at (up) {$\ket{0}$};
            \node[below] at (down) {$\ket{1}$};
        \end{blochsphere}
    }
    \subfloat[Cible]{
        \begin{blochsphere}[radius=1.5 cm,tilt=15,rotation=-30,opacity=0]
            \drawBallGrid[style={opacity=0.1}]{30}{180}

            \drawStatePolar[axisarrow = true, statewidth = 0.3]{x-Achse}{90}{90}
            \drawStatePolar[axisarrow=true, statewidth = 0.3]{y-Achse}{90}{0}
            \drawStatePolar[axisarrow = true, statewidth = 0.3]{z-Achse}{0}{0}

            \node[below] at (x-Achse) {\fontsize{0.15cm}{1em} \large $x$};
            \node[below] at (y-Achse) {\fontsize{0.15cm}{1em} \large $y$};
            \node[below] at (z-Achse) {\fontsize{0.15cm}{1em} \large $z$};

            \drawStatePolar[statecolor = red]{State}{-90}{90}
            \drawStatePolar[statecolor = green]{State}{-90}{90}

            \labelLatLon{up}{90}{0};
            \labelLatLon{down}{-90}{90};
            \node[above] at (up) {$\ket{0}$};
            \node[below] at (down) {$\ket{1}$};
        \end{blochsphere}
    }
    \caption{Application d'une porte $CX$ sur un qubit à l'état $\ket{+}_c \otimes \ket{-}_t$}
    \label{fig:retour-de-phase-2}
\end{figure}
En comparant la figure~\ref{fig:retour-de-phase-2} avec la figure~\ref{fig:retour-de-phase-1},
on remarque que l'application d'une porte $CX$ sur un qubit à l'état $\ket{+}_c \otimes \ket{-}_t$
va modifier l'état du qubit de contrôle, à l'inverse de ce qui se passe avec dans la base
standard $\{\ket{0}, \ket{1}\}$.\\
Cet exemple répond complètement à la définition : via la porte de contrôle $CX$,
l'opération, on arrive en changeant de base à modifier l'état du qubit de contrôle.
De plus, ce changement de base est aisé, se faisant simplement en appliquant une porte
$H$ sur les qubits désirés, ainsi que réversible.
Il en découle l'identité décrite dans la figure~\ref{fig:id-cx}.
Celle-ci permet de choisir librement le contrôle et la cible d'une porte $CX$,
dans le cas où l'architecture ne le permet pas directement et cela n'est qu'un
exemple parmi tant d'autres.
\begin{figure}[H]
    \centering
    \[\shorthandoff{!}
    \scalebox{1.0}{
        \Qcircuit @C=1.0em @R=0.8em @!R { \\
        & \ctrl{1} & \qw\\
        & \targ & \qw\\
        \\ }}
    \]
    =
    \[\shorthandoff{!}
    \scalebox{1.0}{
        \Qcircuit @C=1.0em @R=0.2em @!R { \\
        & \gate{\mathrm{H}} & \targ & \gate{\mathrm{H}} & \qw \\
        & \gate{\mathrm{H}} & \ctrl{-1} & \gate{\mathrm{H}} & \qw \\
        \\ }}
    \]
    \caption{Changement de sens d'une porte $CX$ en changeant de base}
    \label{fig:id-cx}
\end{figure}
Le retour de phase est donc un protocole très utile, car il implique que selon l'état des qubits,
une opération contrôlée peut être effectuée dans un sens ou dans l'autre, et cela
peut être très utile dans la construction de circuits.

\section{Transformée de Fourier quantique}\label{sec:transformee-de-fourier-quantique}

Les séries de Fourier~\cite{wiki:serie-fourier} sont des outils mathématiques très puissants, permettant
de décomposer une fonction périodique en une somme de fonctions sinusoïdales.
Ces séries sont très utilisées en traitement du signal, en analyse de données,
et de nombreuses autres applications.\\
En effet, elles permettent de décomposer un signal en une somme de signaux plus
simples, et donc de simplifier l'analyse de celui-ci.
L'idée des séries de Fourier est de décomposer une fonction périodique $f$ en une
somme de fonctions sinusoïdales de fréquences différentes, et en généralisant
aux nombres complexes, elle s'exprime via la formule d'Euler :
\[
    e^{i\theta} = \cos(\theta) + i\sin(\theta)
\]
On appelle transformée de Fourier la transformation qui permet de passer d'une
fonction $f$ à sa décomposition en série de Fourier, généralisée à toute fonction,
même non périodique.
Dans le cas qui nous intéresse, seule la version discrète~\cite{wiki:transformee-fourier-disc} de la transformée de
Fourier nous intéresse, et elle associe un vecteur $(x_0, \ldots, x_{N-1})$ à un
vecteur $(y_0, \ldots, y_{N-1})$ tel que :
\[
    y_k = \frac{1}{\sqrt{N}} \sum_{j=0}^{N-1} x_j e^{2\pi i \frac{jk}{N}}
\]
et similairement de manière quantique~\cite{wiki:qft}, on associe un état $\ket{X} = \sum_{j=0}^{N-1} x_j \ket{j}$
à un état $\ket{Y} = \sum_{k=0}^{N-1} y_k \ket{k}$ selon la même formule.\\ \\
Elle fait ainsi la transformation entre deux bases, celle $Z$ et celle de Fourier.
La visualisation intuitive de celle-ci est suffisante pour comprendre son fonctionnement
et son implémentation.

\begin{figure}[H]
    \centering
    \subfloat[qubit 0]{
        \begin{blochsphere}[radius=1.5 cm,tilt=15,rotation=-30,opacity=0]
            \drawBallGrid[style={opacity=0.1}]{30}{180}

            \drawStatePolar[axisarrow = true, statewidth = 0.3]{x-Achse}{90}{90}
            \drawStatePolar[axisarrow=true, statewidth = 0.3]{y-Achse}{90}{0}
            \drawStatePolar[axisarrow = true, statewidth = 0.3]{z-Achse}{0}{0}

            \node[below] at (x-Achse) {\fontsize{0.15cm}{1em} \large $x$};
            \node[below] at (y-Achse) {\fontsize{0.15cm}{1em} \large $y$};
            \node[below] at (z-Achse) {\fontsize{0.15cm}{1em} \large $z$};

            \drawStatePolar[statecolor = red]{State}{180}{0}
            \drawStatePolar[statecolor = green]{State}{0}{0}

            \labelLatLon{up}{90}{0};
            \labelLatLon{down}{-90}{90};
            \node[above] at (up) {$\ket{0}$};
            \node[below] at (down) {$\ket{1}$};
        \end{blochsphere}
    }
    \subfloat[qubit 1]{
        \begin{blochsphere}[radius=1.5 cm,tilt=15,rotation=-30,opacity=0]
            \drawBallGrid[style={opacity=0.1}]{30}{180}

            \drawStatePolar[axisarrow = true, statewidth = 0.3]{x-Achse}{90}{90}
            \drawStatePolar[axisarrow=true, statewidth = 0.3]{y-Achse}{90}{0}
            \drawStatePolar[axisarrow = true, statewidth = 0.3]{z-Achse}{0}{0}

            \node[below] at (x-Achse) {\fontsize{0.15cm}{1em} \large $x$};
            \node[below] at (y-Achse) {\fontsize{0.15cm}{1em} \large $y$};
            \node[below] at (z-Achse) {\fontsize{0.15cm}{1em} \large $z$};

            \drawStatePolar[statecolor = red]{State}{0}{0}
            \drawStatePolar[statecolor = green]{State}{180}{0}

            \labelLatLon{up}{90}{0};
            \labelLatLon{down}{-90}{90};
            \node[above] at (up) {$\ket{0}$};
            \node[below] at (down) {$\ket{1}$};
        \end{blochsphere}
    }
    \subfloat[qubit 2]{
        \begin{blochsphere}[radius=1.5 cm,tilt=15,rotation=-30,opacity=0]
            \drawBallGrid[style={opacity=0.1}]{30}{180}

            \drawStatePolar[axisarrow = true, statewidth = 0.3]{x-Achse}{90}{90}
            \drawStatePolar[axisarrow=true, statewidth = 0.3]{y-Achse}{90}{0}
            \drawStatePolar[axisarrow = true, statewidth = 0.3]{z-Achse}{0}{0}

            \node[below] at (x-Achse) {\fontsize{0.15cm}{1em} \large $x$};
            \node[below] at (y-Achse) {\fontsize{0.15cm}{1em} \large $y$};
            \node[below] at (z-Achse) {\fontsize{0.15cm}{1em} \large $z$};

            \drawStatePolar[statecolor = red]{State}{0}{0}
            \drawStatePolar[statecolor = green]{State}{180}{0}

            \labelLatLon{up}{90}{0};
            \labelLatLon{down}{-90}{90};
            \node[above] at (up) {$\ket{0}$};
            \node[below] at (down) {$\ket{1}$};
        \end{blochsphere}
    }
    \caption{Nombres dans la base $Z$ (1 en en rouge, 6 en vert)}
    \label{fig:qft-z}
\end{figure}
\begin{figure}[H]
    \centering
    \subfloat[qubit 0]{
        \begin{blochsphere}[radius=1.5 cm,tilt=15,rotation=-30,opacity=0]
            \drawBallGrid[style={opacity=0.1}]{30}{180}

            \drawStatePolar[axisarrow = true, statewidth = 0.3]{x-Achse}{90}{90}
            \drawStatePolar[axisarrow=true, statewidth = 0.3]{y-Achse}{90}{0}
            \drawStatePolar[axisarrow = true, statewidth = 0.3]{z-Achse}{0}{0}

            \node[below] at (x-Achse) {\fontsize{0.15cm}{1em} \large $x$};
            \node[below] at (y-Achse) {\fontsize{0.15cm}{1em} \large $y$};
            \node[below] at (z-Achse) {\fontsize{0.15cm}{1em} \large $z$};

            \drawStatePolar[statecolor = red]{State}{90}{45}
            \drawStatePolar[statecolor = green]{State}{90}{180}

            \labelLatLon{up}{90}{0};
            \labelLatLon{down}{-90}{90};
            \node[above] at (up) {$\ket{0}$};
            \node[below] at (down) {$\ket{1}$};
        \end{blochsphere}
    }
    \subfloat[qubit 1]{
        \begin{blochsphere}[radius=1.5 cm,tilt=15,rotation=-30,opacity=0]
            \drawBallGrid[style={opacity=0.1}]{30}{180}

            \drawStatePolar[axisarrow = true, statewidth = 0.3]{x-Achse}{90}{90}
            \drawStatePolar[axisarrow=true, statewidth = 0.3]{y-Achse}{90}{0}
            \drawStatePolar[axisarrow = true, statewidth = 0.3]{z-Achse}{0}{0}

            \node[below] at (x-Achse) {\fontsize{0.15cm}{1em} \large $x$};
            \node[below] at (y-Achse) {\fontsize{0.15cm}{1em} \large $y$};
            \node[below] at (z-Achse) {\fontsize{0.15cm}{1em} \large $z$};

            \drawStatePolar[statecolor = red]{State}{90}{0}
            \drawStatePolar[statecolor = green]{State}{90}{-90}

            \labelLatLon{up}{90}{0};
            \labelLatLon{down}{-90}{90};
            \node[above] at (up) {$\ket{0}$};
            \node[below] at (down) {$\ket{1}$};
        \end{blochsphere}
    }
    \subfloat[qubit 2]{
        \begin{blochsphere}[radius=1.5 cm,tilt=15,rotation=-30,opacity=0]
            \drawBallGrid[style={opacity=0.1}]{30}{180}

            \drawStatePolar[axisarrow = true, statewidth = 0.3]{x-Achse}{90}{90}
            \drawStatePolar[axisarrow=true, statewidth = 0.3]{y-Achse}{90}{0}
            \drawStatePolar[axisarrow = true, statewidth = 0.3]{z-Achse}{0}{0}

            \node[below] at (x-Achse) {\fontsize{0.15cm}{1em} \large $x$};
            \node[below] at (y-Achse) {\fontsize{0.15cm}{1em} \large $y$};
            \node[below] at (z-Achse) {\fontsize{0.15cm}{1em} \large $z$};

            \drawStatePolar[statecolor = red]{State}{90}{-90}
            \drawStatePolar[statecolor = green]{State}{90}{90}

            \labelLatLon{up}{90}{0};
            \labelLatLon{down}{-90}{90};
            \node[above] at (up) {$\ket{0}$};
            \node[below] at (down) {$\ket{1}$};
        \end{blochsphere}
    }
    \caption{Nombres dans la base de Fourier (1 en rouge, 6 en vert)}
    \label{fig:qft-f}
\end{figure}
Dans la figure~\ref{fig:qft-z}, les nombres sont en binaire (i.e. 1 correspond à 100 et 6 à 011) en utilisant la base $\ket{0}$
et $\ket{1}$, tandis que dans la figure~\ref{fig:qft-f}, on observe que les nombres sont
encodés dans la rotation autour de l'axe $Z$.
On note ce nouvel état $QFT\ket{x} = \ket{\tilde{x}}$.\\
Comme on voit sur la figure~\ref{fig:qft-f}, le nombre 6 est encodé sur trois qubits
via une rotation de $\frac{6}{2^n} = \frac{6}{2^3} = \frac{3}{4}$ d'un tour complet
sur le qubit 0, $\frac{3}{2}$ d'un tour complet sur le qubit 1 et une fois encore,
on double cela pour le qubit 2.\\ \\
Nous ne nous attarderons pas sur l'implémentation plus que cela, car elle fait
exactement ce que l'on vient de décrire, soit l'application d'une porte d'Hadamard,
puis la rotation autour de l'axe $Z$ selon la valeur des autres qubits via des portes
de phase contrôlées, ensuite l'opération se répète sur les qubits suivants.\\
Néanmoins, on se rend compte que le circuit grossit assez vite, et à un certain
stade, pour économiser des ressources, on peut ignorer certaines rotations tout en gardant
une précision suffisante.\\
Dernier point, comme toutes les opérations, celle-ci est réversible, et on peut définir
l'opération inverse comme $QFT^{\dagger}$.\\ \\
La transformée de Fourier quantique permet donc de profiter des différentes bases
offertes par les qubits, par exemple dans ce cas en encodant les nombres différemment.
Dans ce cas-là, cela va même plus loin, car cela n'est pas comme en informatique
classique où l'on se base sur deux états, mais on exploite une propriété quantique
qui est la phase.

\section{Estimation de phase quantique}\label{sec:estimation-de-phase-quantique}

Tout comme la section précédente, pour ce protocole, nous laisserons quelque peu de
côté la partie mathématique, car l'aspect intuitif est bien plus intéressant et
pertinent.\\
Soit un opérateur $U$ et un état $\ket{\psi}$ tel que $U\ket{\psi} = e^{2\pi i \theta}\ket{\psi}$,
l'algorithme d'estimation de phase quantique~\cite{wiki:qpe} permet d'estimer $\theta$.
On note donc que $\ket{\psi}$ est un vecteur propre de $U$ avec la valeur propre
$e^{2\pi i \theta}$.\\ \\
Pour ce faire, on initialise un registre de $t$ qubits à $\ket{0}$, et un autre
registre d'un qubit à $\ket{\psi}$.
Ensuite, on fait passer le registre de $t$ qubits à travers une porte d'Hadamard,
puis le but est d'encoder la phase de $U$ dans la base de Fourier sur le registre.
Pour ce faire, on utilise le retour de phase d'une porte $CU$, avec un des qubits
du registre de $t$ qubits comme contrôle et $\ket{\psi}$ en cible.
Sur le premier qubit du registre de $t$ qubits, la porte $CU$ $2^{t-1}$ fois, le second
$2^{t-2}$ fois, et ainsi de suite jusqu'au dernier qubit où la porte $CU$ est appliquée
une fois.\\
De là, on applique la transformée de Fourier inverse sur le registre de $t$ qubits,
puis on mesure le registre.
La mesure de celle-ci correspond à $2^t \theta$, et dans ce cas si c'est un nombre
entier, on aura la solution exacte, sinon, on aura une probabilité bien plus élevée pour
les valeurs autour de la solution.\\ \\
Si l'on construit le circuit avec un angle $\theta = \frac{1}{3}$, on utilisera pour
l'exemple la porte $P(\frac{2 \pi}{3})$, et on note que $P(\frac{2 \pi}{3})\ket{1} = e^{2\pi i \frac{1}{3}}\ket{1}$,
donc $\ket{1}$ est un vecteur propre de notre opérateur.
\begin{figure}[H]
    \centering
    \[\shorthandoff{!}
    \scalebox{0.8}{
        \Qcircuit @C=1.0em @R=0.2em @!R { \\
        \nghost{{q}_{0} :  } & \lstick{{q}_{0} :  } & \gate{\mathrm{H}} & \ctrl{3} & \qw & \qw & \qw & \qw & \qw & \qw & \qw & \qw & \qw & \qw & \qw & \qw & \qw & \qw & \qw & \qw & \qw & \qw & \qw & \qw & \qw & \qw & \qw & \qw & \qw & \qw & \qw & \multigate{2}{QFT^{\dagger}}_<<<{0} & \meter & \qw & \qw & \qw & \qw\\
        \nghost{{q}_{1} :  } & \lstick{{q}_{1} :  } & \gate{\mathrm{H}} & \qw & \qw & \qw & \qw & \ctrl{2} & \qw & \qw & \qw & \ctrl{2} & \qw & \qw & \qw & \qw & \qw & \qw & \qw & \qw & \qw & \qw & \qw & \qw & \qw & \qw & \qw & \qw & \qw & \qw & \qw & \ghost{QFT^{\dagger}}_<<<{1} & \qw & \meter & \qw & \qw & \qw\\
        \nghost{{q}_{2} :  } & \lstick{{q}_{2} :  } & \gate{\mathrm{H}} & \qw & \dstick{\hspace{2.0em}\mathrm{P}\,(\mathrm{\frac{2\pi}{3}})} \qw & \qw & \qw & \qw & \dstick{\hspace{2.0em}\mathrm{P}\,(\mathrm{\frac{2\pi}{3}})} \qw & \qw & \qw & \qw & \dstick{\hspace{2.0em}\mathrm{P}\,(\mathrm{\frac{2\pi}{3}})} \qw & \qw & \qw & \ctrl{1} & \dstick{\hspace{2.0em}\mathrm{P}\,(\mathrm{\frac{2\pi}{3}})} \qw & \qw & \qw & \ctrl{1} & \dstick{\hspace{2.0em}\mathrm{P}\,(\mathrm{\frac{2\pi}{3}})} \qw & \qw & \qw & \ctrl{1} & \dstick{\hspace{2.0em}\mathrm{P}\,(\mathrm{\frac{2\pi}{3}})} \qw & \qw & \qw & \ctrl{1} & \dstick{\hspace{2.0em}\mathrm{P}\,(\mathrm{\frac{2\pi}{3}})} \qw & \qw & \qw & \ghost{QFT^{\dagger}}_<<<{2} & \qw & \qw & \meter & \qw & \qw\\
        \nghost{{q}_{3} :  } & \lstick{{q}_{3} :  } & \gate{\mathrm{X}} & \control \qw & \qw & \qw & \qw & \control \qw & \qw & \qw & \qw & \control \qw & \qw & \qw & \qw & \control \qw & \qw & \qw & \qw & \control \qw & \qw & \qw & \qw & \control \qw & \qw & \qw & \qw & \control \qw & \qw & \qw & \qw & \qw & \qw & \qw & \qw & \qw & \qw\\
        \nghost{\mathrm{{c} :  }} & \lstick{\mathrm{{c} :  }} & \lstick{/_{_{3}}} \cw & \cw & \cw & \cw & \cw & \cw & \cw & \cw & \cw & \cw & \cw & \cw & \cw & \cw & \cw & \cw & \cw & \cw & \cw & \cw & \cw & \cw & \cw & \cw & \cw & \cw & \cw & \cw & \cw & \cw & \dstick{_{_{\hspace{0.0em}0}}} \cw \ar @{<=} [-4,0] & \dstick{_{_{\hspace{0.0em}1}}} \cw \ar @{<=} [-3,0] & \dstick{_{_{\hspace{0.0em}2}}} \cw \ar @{<=} [-2,0] & \cw & \cw\\
        \\ }}
    \]
    \caption{Estimation de phase pour $\theta = \frac{1}{3}$ avec un registre de $t = 3$ qubits}
    \label{fig:qpes3-circ}
\end{figure}
Les qubits 0 à 2 sont le registre de $t$ qubits, le qubit 3 est le registre de $\ket{\psi}$,
que l'on initialise selon le protocole énoncé précédemment.
Petite subtilité, par rapport à ce qui a été dit auparavant, on applique la porte $CU$
le plus de fois sur le dernier qubit du registre de $t$ qubits, et non le premier.
Si l'on ne le faisait pas, les résultats obtenus seraient inversés, et serait donc moins
lisibles, cela est dû à la manière dont le module utilisé pour la simulation des circuits
quantiques et l'interaction avec les ordinateurs à disposition fonctionne.
\begin{figure}[H]
    \centering
    \import{images/algo/protocoles/}{qpe13s3.tex}
    \caption{Mesures pour $\theta = \frac{1}{3}$ avec un registre de $t = 3$ qubits}
    \label{fig:qpes3-meas}
\end{figure}
Les deux plus grandes valeurs obtenues sont 011 qui correspond à 3 et donc un angle
de $\theta \approx \frac{3}{2^3} = 0.375$ et 010 qui correspond à 2 et donc un angle
$\theta \approx \frac{2}{2^3} = 0.25$.
Elles entourent bien la valeur de $\theta$ que l'on cherchait à estimer, néanmoins
la précision laisse à désirer.\\ \\
Pour l'augmenter, on peut augmenter le nombre de qubits du registre de $t$ qubits.
Regardons par exemple avec 5 qubits.
Outre le gigantisme du circuit~\ref{fig:qpes5-circ}, problème qui sera discuté ultérieurement
avec des pistes d'amélioration, on peut remarquer que les mesures sont bien plus précises
que dans le cas précédent.
On obtient principalement une mesure de 01011, soit 11 en décimal, ce qui donne une estimation
de $\frac{11}{32} \approx 0.344$.
\begin{sidewaysfigure}
    \centering
    \[\shorthandoff{!}
    \scalebox{0.4}{
        \Qcircuit @C=1.0em @R=0.2em @!R { \\
        \nghost{{q}_{0} :  } & \lstick{{q}_{0} :  } & \gate{\mathrm{H}} & \ctrl{5} & \qw & \qw & \qw & \qw & \qw & \qw & \qw & \qw & \qw & \qw & \qw & \qw & \qw & \qw & \qw & \qw & \qw & \qw & \qw & \qw & \qw & \qw & \qw & \qw & \qw & \qw & \qw & \qw & \qw & \qw & \qw & \qw & \qw & \qw & \qw & \qw & \qw & \qw & \qw & \qw & \qw & \qw & \qw & \qw & \qw & \qw & \qw & \qw & \qw & \qw & \qw & \qw & \qw & \qw & \qw & \qw & \qw & \qw & \qw & \qw & \qw & \qw & \qw & \qw & \qw & \qw & \qw & \qw & \qw & \qw & \qw & \qw & \qw & \qw & \qw & \qw & \qw & \qw & \qw & \qw & \qw & \qw & \qw & \qw & \qw & \qw & \qw & \qw & \qw & \qw & \qw & \qw & \qw & \qw & \qw & \qw & \qw & \qw & \qw & \qw & \qw & \qw & \qw & \qw & \qw & \qw & \qw & \qw & \qw & \qw & \qw & \qw & \qw & \qw & \qw & \qw & \qw & \qw & \qw & \qw & \qw & \qw & \qw & \multigate{4}{QFT^{\dagger}}_<<<{0} \barrier[0em]{5} & \qw & \meter & \qw & \qw & \qw & \qw & \qw & \qw\\
        \nghost{{q}_{1} :  } & \lstick{{q}_{1} :  } & \gate{\mathrm{H}} & \qw & \qw & \qw & \qw & \ctrl{4} & \qw & \qw & \qw & \ctrl{4} & \qw & \qw & \qw & \qw & \qw & \qw & \qw & \qw & \qw & \qw & \qw & \qw & \qw & \qw & \qw & \qw & \qw & \qw & \qw & \qw & \qw & \qw & \qw & \qw & \qw & \qw & \qw & \qw & \qw & \qw & \qw & \qw & \qw & \qw & \qw & \qw & \qw & \qw & \qw & \qw & \qw & \qw & \qw & \qw & \qw & \qw & \qw & \qw & \qw & \qw & \qw & \qw & \qw & \qw & \qw & \qw & \qw & \qw & \qw & \qw & \qw & \qw & \qw & \qw & \qw & \qw & \qw & \qw & \qw & \qw & \qw & \qw & \qw & \qw & \qw & \qw & \qw & \qw & \qw & \qw & \qw & \qw & \qw & \qw & \qw & \qw & \qw & \qw & \qw & \qw & \qw & \qw & \qw & \qw & \qw & \qw & \qw & \qw & \qw & \qw & \qw & \qw & \qw & \qw & \qw & \qw & \qw & \qw & \qw & \qw & \qw & \qw & \qw & \qw & \qw & \ghost{QFT^{\dagger}}_<<<{1} & \qw & \qw & \meter & \qw & \qw & \qw & \qw & \qw\\
        \nghost{{q}_{2} :  } & \lstick{{q}_{2} :  } & \gate{\mathrm{H}} & \qw & \qw & \qw & \qw & \qw & \qw & \qw & \qw & \qw & \qw & \qw & \qw & \ctrl{3} & \qw & \qw & \qw & \ctrl{3} & \qw & \qw & \qw & \ctrl{3} & \qw & \qw & \qw & \ctrl{3} & \qw & \qw & \qw & \qw & \qw & \qw & \qw & \qw & \qw & \qw & \qw & \qw & \qw & \qw & \qw & \qw & \qw & \qw & \qw & \qw & \qw & \qw & \qw & \qw & \qw & \qw & \qw & \qw & \qw & \qw & \qw & \qw & \qw & \qw & \qw & \qw & \qw & \qw & \qw & \qw & \qw & \qw & \qw & \qw & \qw & \qw & \qw & \qw & \qw & \qw & \qw & \qw & \qw & \qw & \qw & \qw & \qw & \qw & \qw & \qw & \qw & \qw & \qw & \qw & \qw & \qw & \qw & \qw & \qw & \qw & \qw & \qw & \qw & \qw & \qw & \qw & \qw & \qw & \qw & \qw & \qw & \qw & \qw & \qw & \qw & \qw & \qw & \qw & \qw & \qw & \qw & \qw & \qw & \qw & \qw & \qw & \qw & \qw & \qw & \ghost{QFT^{\dagger}}_<<<{2} & \qw & \qw & \qw & \meter & \qw & \qw & \qw & \qw\\
        \nghost{{q}_{3} :  } & \lstick{{q}_{3} :  } & \gate{\mathrm{H}} & \qw & \qw & \qw & \qw & \qw & \qw & \qw & \qw & \qw & \qw & \qw & \qw & \qw & \qw & \qw & \qw & \qw & \qw & \qw & \qw & \qw & \qw & \qw & \qw & \qw & \qw & \qw & \qw & \ctrl{2} & \qw & \qw & \qw & \ctrl{2} & \qw & \qw & \qw & \ctrl{2} & \qw & \qw & \qw & \ctrl{2} & \qw & \qw & \qw & \ctrl{2} & \qw & \qw & \qw & \ctrl{2} & \qw & \qw & \qw & \ctrl{2} & \qw & \qw & \qw & \ctrl{2} & \qw & \qw & \qw & \qw & \qw & \qw & \qw & \qw & \qw & \qw & \qw & \qw & \qw & \qw & \qw & \qw & \qw & \qw & \qw & \qw & \qw & \qw & \qw & \qw & \qw & \qw & \qw & \qw & \qw & \qw & \qw & \qw & \qw & \qw & \qw & \qw & \qw & \qw & \qw & \qw & \qw & \qw & \qw & \qw & \qw & \qw & \qw & \qw & \qw & \qw & \qw & \qw & \qw & \qw & \qw & \qw & \qw & \qw & \qw & \qw & \qw & \qw & \qw & \qw & \qw & \qw & \qw & \ghost{QFT^{\dagger}}_<<<{3} & \qw & \qw & \qw & \qw & \meter & \qw & \qw & \qw\\
        \nghost{{q}_{4} :  } & \lstick{{q}_{4} :  } & \gate{\mathrm{H}} & \qw & \dstick{\hspace{2.0em}\mathrm{P}\,(\mathrm{\frac{2\pi}{3}})} \qw & \qw & \qw & \qw & \dstick{\hspace{2.0em}\mathrm{P}\,(\mathrm{\frac{2\pi}{3}})} \qw & \qw & \qw & \qw & \dstick{\hspace{2.0em}\mathrm{P}\,(\mathrm{\frac{2\pi}{3}})} \qw & \qw & \qw & \qw & \dstick{\hspace{2.0em}\mathrm{P}\,(\mathrm{\frac{2\pi}{3}})} \qw & \qw & \qw & \qw & \dstick{\hspace{2.0em}\mathrm{P}\,(\mathrm{\frac{2\pi}{3}})} \qw & \qw & \qw & \qw & \dstick{\hspace{2.0em}\mathrm{P}\,(\mathrm{\frac{2\pi}{3}})} \qw & \qw & \qw & \qw & \dstick{\hspace{2.0em}\mathrm{P}\,(\mathrm{\frac{2\pi}{3}})} \qw & \qw & \qw & \qw & \dstick{\hspace{2.0em}\mathrm{P}\,(\mathrm{\frac{2\pi}{3}})} \qw & \qw & \qw & \qw & \dstick{\hspace{2.0em}\mathrm{P}\,(\mathrm{\frac{2\pi}{3}})} \qw & \qw & \qw & \qw & \dstick{\hspace{2.0em}\mathrm{P}\,(\mathrm{\frac{2\pi}{3}})} \qw & \qw & \qw & \qw & \dstick{\hspace{2.0em}\mathrm{P}\,(\mathrm{\frac{2\pi}{3}})} \qw & \qw & \qw & \qw & \dstick{\hspace{2.0em}\mathrm{P}\,(\mathrm{\frac{2\pi}{3}})} \qw & \qw & \qw & \qw & \dstick{\hspace{2.0em}\mathrm{P}\,(\mathrm{\frac{2\pi}{3}})} \qw & \qw & \qw & \qw & \dstick{\hspace{2.0em}\mathrm{P}\,(\mathrm{\frac{2\pi}{3}})} \qw & \qw & \qw & \qw & \dstick{\hspace{2.0em}\mathrm{P}\,(\mathrm{\frac{2\pi}{3}})} \qw & \qw & \qw & \ctrl{1} & \dstick{\hspace{2.0em}\mathrm{P}\,(\mathrm{\frac{2\pi}{3}})} \qw & \qw & \qw & \ctrl{1} & \dstick{\hspace{2.0em}\mathrm{P}\,(\mathrm{\frac{2\pi}{3}})} \qw & \qw & \qw & \ctrl{1} & \dstick{\hspace{2.0em}\mathrm{P}\,(\mathrm{\frac{2\pi}{3}})} \qw & \qw & \qw & \ctrl{1} & \dstick{\hspace{2.0em}\mathrm{P}\,(\mathrm{\frac{2\pi}{3}})} \qw & \qw & \qw & \ctrl{1} & \dstick{\hspace{2.0em}\mathrm{P}\,(\mathrm{\frac{2\pi}{3}})} \qw & \qw & \qw & \ctrl{1} & \dstick{\hspace{2.0em}\mathrm{P}\,(\mathrm{\frac{2\pi}{3}})} \qw & \qw & \qw & \ctrl{1} & \dstick{\hspace{2.0em}\mathrm{P}\,(\mathrm{\frac{2\pi}{3}})} \qw & \qw & \qw & \ctrl{1} & \dstick{\hspace{2.0em}\mathrm{P}\,(\mathrm{\frac{2\pi}{3}})} \qw & \qw & \qw & \ctrl{1} & \dstick{\hspace{2.0em}\mathrm{P}\,(\mathrm{\frac{2\pi}{3}})} \qw & \qw & \qw & \ctrl{1} & \dstick{\hspace{2.0em}\mathrm{P}\,(\mathrm{\frac{2\pi}{3}})} \qw & \qw & \qw & \ctrl{1} & \dstick{\hspace{2.0em}\mathrm{P}\,(\mathrm{\frac{2\pi}{3}})} \qw & \qw & \qw & \ctrl{1} & \dstick{\hspace{2.0em}\mathrm{P}\,(\mathrm{\frac{2\pi}{3}})} \qw & \qw & \qw & \ctrl{1} & \dstick{\hspace{2.0em}\mathrm{P}\,(\mathrm{\frac{2\pi}{3}})} \qw & \qw & \qw & \ctrl{1} & \dstick{\hspace{2.0em}\mathrm{P}\,(\mathrm{\frac{2\pi}{3}})} \qw & \qw & \qw & \ctrl{1} & \dstick{\hspace{2.0em}\mathrm{P}\,(\mathrm{\frac{2\pi}{3}})} \qw & \qw & \qw & \ctrl{1} & \dstick{\hspace{2.0em}\mathrm{P}\,(\mathrm{\frac{2\pi}{3}})} \qw & \qw & \qw & \ghost{QFT^{\dagger}}_<<<{4} & \qw & \qw & \qw & \qw & \qw & \meter & \qw & \qw\\
        \nghost{{q}_{5} :  } & \lstick{{q}_{5} :  } & \gate{\mathrm{X}} & \control \qw & \qw & \qw & \qw & \control \qw & \qw & \qw & \qw & \control \qw & \qw & \qw & \qw & \control \qw & \qw & \qw & \qw & \control \qw & \qw & \qw & \qw & \control \qw & \qw & \qw & \qw & \control \qw & \qw & \qw & \qw & \control \qw & \qw & \qw & \qw & \control \qw & \qw & \qw & \qw & \control \qw & \qw & \qw & \qw & \control \qw & \qw & \qw & \qw & \control \qw & \qw & \qw & \qw & \control \qw & \qw & \qw & \qw & \control \qw & \qw & \qw & \qw & \control \qw & \qw & \qw & \qw & \control \qw & \qw & \qw & \qw & \control \qw & \qw & \qw & \qw & \control \qw & \qw & \qw & \qw & \control \qw & \qw & \qw & \qw & \control \qw & \qw & \qw & \qw & \control \qw & \qw & \qw & \qw & \control \qw & \qw & \qw & \qw & \control \qw & \qw & \qw & \qw & \control \qw & \qw & \qw & \qw & \control \qw & \qw & \qw & \qw & \control \qw & \qw & \qw & \qw & \control \qw & \qw & \qw & \qw & \control \qw & \qw & \qw & \qw & \control \qw & \qw & \qw & \qw & \control \qw & \qw & \qw & \qw & \control \qw & \qw & \qw & \qw & \qw & \qw & \qw & \qw & \qw & \qw & \qw & \qw & \qw\\
        \nghost{\mathrm{{c} :  }} & \lstick{\mathrm{{c} :  }} & \lstick{/_{_{5}}} \cw & \cw & \cw & \cw & \cw & \cw & \cw & \cw & \cw & \cw & \cw & \cw & \cw & \cw & \cw & \cw & \cw & \cw & \cw & \cw & \cw & \cw & \cw & \cw & \cw & \cw & \cw & \cw & \cw & \cw & \cw & \cw & \cw & \cw & \cw & \cw & \cw & \cw & \cw & \cw & \cw & \cw & \cw & \cw & \cw & \cw & \cw & \cw & \cw & \cw & \cw & \cw & \cw & \cw & \cw & \cw & \cw & \cw & \cw & \cw & \cw & \cw & \cw & \cw & \cw & \cw & \cw & \cw & \cw & \cw & \cw & \cw & \cw & \cw & \cw & \cw & \cw & \cw & \cw & \cw & \cw & \cw & \cw & \cw & \cw & \cw & \cw & \cw & \cw & \cw & \cw & \cw & \cw & \cw & \cw & \cw & \cw & \cw & \cw & \cw & \cw & \cw & \cw & \cw & \cw & \cw & \cw & \cw & \cw & \cw & \cw & \cw & \cw & \cw & \cw & \cw & \cw & \cw & \cw & \cw & \cw & \cw & \cw & \cw & \cw & \cw & \cw & \dstick{_{_{\hspace{0.0em}0}}} \cw \ar @{<=} [-6,0] & \dstick{_{_{\hspace{0.0em}1}}} \cw \ar @{<=} [-5,0] & \dstick{_{_{\hspace{0.0em}2}}} \cw \ar @{<=} [-4,0] & \dstick{_{_{\hspace{0.0em}3}}} \cw \ar @{<=} [-3,0] & \dstick{_{_{\hspace{0.0em}4}}} \cw \ar @{<=} [-2,0] & \cw & \cw\\
        \\ }}
    \]
    \caption{Estimation de phase pour $\theta = \frac{1}{3}$ avec un registre de $t = 5$ qubits}
    \label{fig:qpes5-circ}
\end{sidewaysfigure}
\begin{figure}[H]
    \centering
    \import{images/algo/protocoles/}{qpe13s5.tex}
    \caption{Mesures pour $\theta = \frac{1}{3}$ avec un registre de $t = 5$ qubits}
    \label{fig:qpes5-meas}
\end{figure}
On a vu avec la transformée de Fourier quantique que la phase offrait une nouvelle
dimension pour l'encodage des nombres.
La phase peut offrir d'autres possibilités, et donc la possibilité de mesurer cette
propriété est intéressante.
Il faut préciser que la transformée de Fourier quantique se base sur la phase des
qubits, alors que l'estimation de phase quantique mesure la phase qu'induit un
opérateur sur un état propre de celui-ci.
Néanmoins, l'un a l'autre peuvent être relié, car on peut encoder dans la base de Fourier
par un opérateur et on peut mesurer la phase de celui-ci.
De fait, les deux notions sont assez lié dans ce sens-là, même si elles peuvent être
utilisées indépendamment l'une de l'autre.
De plus, cela revêt un aspect aussi conceptuel, car par cette mesure, on donne une valeur
concrète à une propriété quantique, basée sur les nombres complexes qui plus est.
