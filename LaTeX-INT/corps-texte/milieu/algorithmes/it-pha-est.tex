\chapter{Estimation de phase itérative}\label{ch:iterative-phase-estimatimation}

Durant les chapitres précédents, l'estimation de phase a eu de nombreux rôles très utiles,
que ce soit dans l'implémentation de l'algorithme de Shor ou dans la recherche du nombre
de solutions de l'opérateur de Grover.
Néanmoins, ce circuit grossit assez vite quand on désire une précision élevée, ce qui
entraîne une augmentation de l'erreur dans ces grands circuits.\\
Nous allons donc voir dans ce chapitre une méthode permettant d'obtenir une estimation
de phase avec une précision arbitraire, tout en gardant un circuit de taille constante.
L'estimation de phase itérative~\cite{ipe-article} est une méthode consistante à faire certaines opérations,
puis à mesurer le circuit, et à recommencer en suivant une certaine procédure jusqu'à
obtenir la précision désirée.\\ \\
Reprenons donc le problème de l'estimation de phase~\ref{sec:estimation-de-phase-quantique},
et considérons que l'angle $\theta$ peut être écrit comme une
$\theta = \frac{\theta_1}{2} + \frac{\theta_2}{4} + \dots + \frac{\theta_m}{2^m}$.
Nous allons déveloper la méthode avec une porte $U$ active sur un qubit, mais il est
aussi valable pour une porte $U$ active sur plusieurs qubits.\\
L'initialisation du circuit se fait avec deux qubits, $q_0 \rightarrow \ket{+}$ et
$q_1 \rightarrow \ket{\Psi}$, avec $\ket{\Psi}$ un vecteur propre de $U$.
L'application d'une porte $CU$ un nombre $2^t$ de fois va, par un retour de phase,
faire passer $q_0 \rightarrow \ket{0} + e^{2i\pi\theta2^t}\ket{1}$.\\ \\
Alors, pour $t= m-1$, nous obtenons un facteur pour $\ket{1}$ de $e^{2i\pi\theta2^{m-1}}
= e^{2i\pi 2^{m-1} \left(\frac{\theta_1}{2} + \frac{\theta_2}{4} + \dots + \frac{\theta_m}{2^m}\right)}
= e^{2i\pi \frac{\theta_m}{2}}$, car les autres facteurs sont des multiples de $2\pi$.
De fait, si $\theta_m = 0$, alors le facteur est égal à $1$, et si $\theta_m = 1$, alors
le facteur est égal à $-1$, ce qui fera basculer le qubit $q_0$ à $\ket{+}$ ou $\ket{-}$.
En appliquant alors une porte de Hadamard sur $q_0$, nous obtenons $\ket{0}$ ou $\ket{1}$,
qui correspond à $\theta_m$.\\
Pour la deuxième itération, nous allons appliquer $CU$ un nombre $2^{m-2}$ de fois, ce qui
va nous donner un facteur pour $\ket{1}$ de $e^{2i\pi\theta2^{m-2}} = e^{2i\pi 2^{m-2}
\left(\frac{\theta_1}{2} + \frac{\theta_2}{4} + \dots + \frac{\theta_m}{2^m}\right)}
= e^{2i\pi \frac{\theta_{m-1}}{2}} e^{2i\pi \frac{\theta_{m}}{4}}$.
Comme nous connaissons déjà $\theta_m$, nous pouvons annuler le facteur le concernant
en appliquant une porte de phase de $-\frac{\pi}{2}$ sur $q_0$ si $\theta_m = 1$, et
sinon ne rien faire.\\
On continue la même procédure jusqu'à la dernière itération, en enlevant à chaque fois
les facteurs déjà connus.\\ \\
Il suffit ensuite pour connaitre la phase de prendre la liste des bits $\theta_1 \theta_2
\dots \theta_m$ et de les convertir en décimal, puis de diviser par $2^m$ pour obtenir
la phase $\theta$.\\ \\
Par exemple, pour la porte $U = \begin{pmatrix} 1 & 0 \\ 0 & e^{i\frac{\pi}{4}} \end{pmatrix}$,
nous pouvons nous attendre à obtenir $\theta = \frac{1}{8} = 0.125$.
\begin{figure}[H]
    \centering
    \[\shorthandoff{!}
    \scalebox{0.5}{
        \Qcircuit @C=1.0em @R=0.2em @!R { \\
        \nghost{{q}_{0} :  } & \lstick{{q}_{0} :  } & \gate{\mathrm{H}} & \control \qw & \dstick{\hspace{2.0em}\mathrm{P}\,(\mathrm{\frac{\pi}{4}})} \qw & \qw & \qw & \control \qw & \dstick{\hspace{2.0em}\mathrm{P}\,(\mathrm{\frac{\pi}{4}})} \qw & \qw & \qw & \control \qw & \dstick{\hspace{2.0em}\mathrm{P}\,(\mathrm{\frac{\pi}{4}})} \qw & \qw & \qw & \control \qw & \dstick{\hspace{2.0em}\mathrm{P}\,(\mathrm{\frac{\pi}{4}})} \qw & \qw & \qw & \gate{\mathrm{H}} & \meter \barrier[0em]{1} & \qw & \gate{\mathrm{\left|0\right\rangle}} & \gate{\mathrm{H}} & \control \qw & \dstick{\hspace{2.0em}\mathrm{P}\,(\mathrm{\frac{\pi}{4}})} \qw & \qw & \qw & \control \qw & \dstick{\hspace{2.0em}\mathrm{P}\,(\mathrm{\frac{\pi}{4}})} \qw & \qw & \qw & \gate{\mathrm{P}\,(\mathrm{\frac{-\pi}{2}})} & \gate{\mathrm{H}} & \meter \barrier[0em]{1} & \qw & \gate{\mathrm{\left|0\right\rangle}} & \gate{\mathrm{H}} & \control \qw & \dstick{\hspace{2.0em}\mathrm{P}\,(\mathrm{\frac{\pi}{4}})} \qw & \qw & \qw & \gate{\mathrm{P}\,(\mathrm{\frac{-\pi}{2}})} & \gate{\mathrm{P}\,(\mathrm{\frac{-\pi}{4}})} & \gate{\mathrm{H}} & \meter & \qw & \qw\\
        \nghost{{q}_{1} :  } & \lstick{{q}_{1} :  } & \gate{\mathrm{X}} & \ctrl{-1} & \qw & \qw & \qw & \ctrl{-1} & \qw & \qw & \qw & \ctrl{-1} & \qw & \qw & \qw & \ctrl{-1} & \qw & \qw & \qw & \qw & \qw & \qw & \qw & \qw & \ctrl{-1} & \qw & \qw & \qw & \ctrl{-1} & \qw & \qw & \qw & \qw & \qw & \qw & \qw & \qw & \qw & \ctrl{-1} & \qw & \qw & \qw & \qw & \qw & \qw & \qw & \qw & \qw\\
        \nghost{\mathrm{{c} :  }} & \lstick{\mathrm{{c} :  }} & \lstick{/_{_{3}}} \cw & \cw & \cw & \cw & \cw & \cw & \cw & \cw & \cw & \cw & \cw & \cw & \cw & \cw & \cw & \cw & \cw & \cw & \dstick{_{_{\hspace{0.0em}0}}} \cw \ar @{<=} [-2,0] & \cw & \cw & \cw & \cw & \cw & \cw & \cw & \cw & \cw & \cw & \cw & \control \cw^(0.0){^{\mathtt{c2_0=0x1}}} \cwx[-2] & \cw & \dstick{_{_{\hspace{0.0em}1}}} \cw \ar @{<=} [-2,0] & \cw & \cw & \cw & \cw & \cw & \cw & \cw & \control \cw^(0.0){^{\mathtt{c2_1=0x1}}} \cwx[-2] & \control \cw^(0.0){^{\mathtt{c2_0=0x1}}} \cwx[-2] & \cw & \dstick{_{_{\hspace{0.0em}2}}} \cw \ar @{<=} [-2,0] & \cw & \cw\\
        \\ }}
    \]
    \caption{Recherche par itération de la phase de $U$}
    \label{fig:ipe-ex}
\end{figure}
Les résultats obtenus montre bien le résultat attendu, et il est à noter la faible
proportion d'erreurs sur un ordinateur quantique réel.
Notons que dans ce cas, nous avons été contraints à utiliser un modèle de bruit plutôt qu'un
vrai ordinateur quantique, car le contrôle d'une porte par la mesure d'un qubit n'est pas
encore implémenté sur les ordinateurs quantiques à notre disposition.
\begin{figure}[H]
    \centering
    \subfloat[Simulation]{
        \import{images/algo/ipe/}{ipe.tex}
    }
    \subfloat[Simulation avec bruit \protect\footnotemark]{
        \import{images/algo/ipe/}{ipe_real.tex}
    }
    \caption{Résultats du circuit (chaîne de bits | valeur décimale)}
    \label{fig:ipe-res}
\end{figure}
\footnotetext{Executé selon le modèle \textit{FakeMelbourne}, à date du 20.08.2023, calqué sur un ordinateur d'IBM hors-ligne au moment de la rédaction}
Cette manière de faire permet donc d'obtenir une estimation de phase avec une précision
arbitraire sans nécessiter un plus gros circuit.
Cela rend donc son utilisation possible sur un ordinateur actuel ou à court terme.
Dans cet objectif, il permet aussi de limiter le temps d'intrication des qubits, car ils
sont mesurés à chaque itération, ce qui permet d'utiliser cet algorithme sans nécessiter un temps
de cohérence trop long.
