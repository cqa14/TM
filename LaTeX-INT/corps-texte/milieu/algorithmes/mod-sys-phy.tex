\chapter{Modélisation d'un système physique}\label{ch:modelisation-d'un-systeme-physique}

\section{Propriété d'un système}\label{sec:propriete-d'un-systeme}
La modélisation de systèmes quantiques fut une des premières applications pour lequel
fut pensé l'ordinateur quantique.
En effet, la simulation d'un système physique est un problème qui peut être résolu par
un ordinateur quantique de manière plus efficace qu'un ordinateur classique, car il
utilise des phénomènes similaires à ceux qui sont étudiés.\\ \\
Dans ce chapitre, nous allons voir comment calculer les niveaux d'énergie hyperfins
d'un atome d'hydrogène à l'état fondamental.
L'atome d'hydrogène est composé d'un proton et d'un électron, et chacun de ces deux
éléments possède un spin, qui prend une valeur positive ou négative.
Il y a donc quatre états de spins de l'atome d'hydrogène, soit les deux spins sont
positifs (état $\ket{++}$), soit les deux spins sont négatifs (état $\ket{--}$), soit
le spin de l'électron est positif et celui du proton est négatif (état $\ket{+-}$), soit
le spin de l'électron est négatif et celui du proton est positif (état $\ket{-+}$).
\import{images/algo/mod}{hype-schm.tex}
De cela, il demeure la question de savoir comment les particules interagissent entre
elles et donc comment le système évolue.
Afin de le faire, nous allons utiliser l'équation de Schrödinger, qui a été présentée
au début afin de justifier les propriétés quantiques de la matière, puis fut quelque
peu oubliée.
C'est donc à ce moment que nous allons la réintroduire :
\[
    - \frac{\hbar^2}{2m} \pdv[2]{\Psi (x,t)}{x} + V(x,t) \Psi (x,t) = i \hbar \pdv{\Psi (x,t)}{t}
\]
qui, formulée comme cela, est assez peu pratique dans notre cas.
En effet, toute la partie gauche de l'équation peut être exprimée comme un opérateur
d'évolution, nommé un hamiltonien, et noté $\hat{H}$.
Cela donne donc :
\[
    \hat{H} \ket{\Psi (t)} = i \hbar \pdv{\ket{\Psi (t)}}{t}
\]
avec l'hamiltonien qui dépend généralement du temps.
Quand ce dernier ne dépend pas du temps, l'équation se simplifie en :
\[
    \hat{H} \ket{\Psi} = E \ket{\Psi}
\]
avec $E$ l'énergie du système.\\ \\
Petite note sur l'hamiltonien en mécanique classique.
Celui-ci est étroitement lié au lagrangien, qui est une manière de décrire un système
physique.
Considérant une position $q$, la variation de celle-ci par rapport au temps est donnée
par :
\[
    \dot{q} = \dv{q}{t}
\]
qui permet de définir la quantité de mouvement $p$ comme étant :
\[
    p = \pdv{\mathcal{L}}{\dot{q}}
\]
avec $\mathcal{L}$ une fonction appelée lagrangien.
De cela, on peut établir une équation différentielle, appelée équation d'Euler-Lagrange,
qui est parfois plus simple à résoudre que l'équation de Newton, qui s'écrit :
\[
    \dv{p}{t} = \pdv{\mathcal{L}}{q}
\]
L'hamiltonien $\mathcal{H}$ est défini quant à lui comme étant la somme de l'énergie cinétique et
de l'énergie potentielle (de manière rigoureuse, il s'agit de la transformée de Legendre
du lagrangien, néanmoins nous nous suffirons de la définition précédente).
Il sert ensuit à définir les équations d'Hamilton :
\[
    \dot{q} = \pdv{\mathcal{H}}{p} \qquad \dot{p} = - \pdv{\mathcal{H}}{q}
\]
et il a un rôle similaire en mécanique quantique.\\ \\
Quand on parle d'un opérateur d'évolution qui influe la fonction d'onde, ça rappellera les
portes des ordinateurs quantiques.
Celles-ci sont en effet des opérateurs d'évolution, et en pratique on calcule l'hamiltonien
de chaque qubit afin de le modifier via les micro-ondes afin que l'utilisateur final ne voie
que les portes quantiques sans devoir faire ces calculs.\\ \\
Le système que l'on étudie est un système à deux particules, et chaque particule a deux états,
ce qui rappelle les qubits.
Or l'hamiltonien du système peut être exprimé selon les bases de Pauli, qui sont les matrices
des portes quantiques $X$, $Y$ et $Z$.
Dans ce cas précis, l'hamiltonien est donné par :
\[
    \hat{H} = A (XX + YY + ZZ)
\]
avec $A$ une constante propre à l'atome d'hydrogène qui peut être calculée via les flux de champs
magnétiques, et qui vaut environ $1.47 \cdot 10^{-6}$ électrons-volts ($eV$).
Comme évoqué pour le cas classique, l'hamiltonien est étroitement lié à l'énergie du système,
et par l'équation de Schrödinger indépendante du temps, on peut déduire que l'énergie du système
correspond aux valeurs propres de l'hamiltonien, qui peut être fait via sa mesure dans les
différentes bases :
\[
    E = \expval{H} = A (\expval{XX} + \expval{YY} + \expval{ZZ})
\]
Cette opération est faisable avec un ordinateur quantique en faisant les changements de base nécessaires.
Voyons donc ce que cela donne déjà avec une porte :
\begin{align*}
    \langle Z \rangle = \langle q| Z |q \rangle &= \langle q|0 \rangle \langle 0|q\rangle - \langle q|1 \rangle \langle 1|q\rangle = |\langle 0|q \rangle|^2-|\langle 1|q \rangle|^2 \\
    \langle X \rangle = \langle q| X |q \rangle &= \langle q|+ \rangle \langle +|q\rangle + \langle q|- \rangle \langle -|q\rangle = |\langle +|q \rangle|^2 + |\langle -|q \rangle|^2 = H (|\langle 0|q \rangle|^2-|\langle 1|q \rangle|^2) \\
    \langle Y \rangle = \langle q| Y |q \rangle &= -i \langle q|0 \rangle \langle 1|q\rangle + i \langle q|1 \rangle \langle 0|q\rangle = -i \langle X \rangle = S^\dag H (|\langle 0|q \rangle|^2-|\langle 1|q \rangle|^2)
\end{align*}
où l'on calcule en prenant un qubit général $\ket{q}$ et l'on cherche à isoler les mesures selon
$\ket{0}$ et $\ket{1}$ en utilisant des portes.
Notons que la porte $S^\dag$ est la porte inverse de $S$, qui est définie comme étant :
\[
    S = \begin{pmatrix}
        1 & 0 \\
        0 & i
    \end{pmatrix}
\]
d'où le fait que $S^\dag$ soit définie comme étant :
\[
    S^\dag = \begin{pmatrix}
        1 & 0 \\
        0 & -i
    \end{pmatrix}
\]
Si l'on fait ensuite de même avec le composante de l'hamiltonien, on obtient :
\begin{align*}
    \langle ZZ \rangle = \langle \psi| ZZ |\psi \rangle &= \langle \psi|(|0 \rangle \langle 0| - |1 \rangle \langle 1|) \otimes (|0 \rangle \langle 0| - |1 \rangle \langle 1|)|\psi \rangle \\
    &= |\langle 00|\psi \rangle|^2-|\langle 01|\psi \rangle|^2-|\langle 10|\psi \rangle|^2+|\langle 11|\psi \rangle|^2 \\
    \langle XX \rangle = \langle \psi| XX |\psi \rangle &= \langle \psi|H(|0 \rangle \langle 0| - |1 \rangle \langle 1|) \otimes H(|0 \rangle \langle 0| - |1 \rangle \langle 1|)|\psi \rangle \\
    &= H^{\otimes 2}(|\langle 00|\psi \rangle|^2-|\langle 01|\psi \rangle|^2-|\langle 10|\psi \rangle|^2+|\langle 11|\psi \rangle|^2) \\
    \langle YY \rangle = \langle \psi| YY |\psi \rangle &= \langle \psi|S^\dag H(|0 \rangle \langle 0| - |1 \rangle \langle 1|) \otimes S^\dag H(|0 \rangle \langle 0| - |1 \rangle \langle 1|)|\psi \rangle \\
    &= {S^{\dag}}^{\otimes 2} H^{\otimes 2}(|\langle 00|\psi \rangle|^2-|\langle 01|\psi \rangle|^2-|\langle 10|\psi \rangle|^2+|\langle 11|\psi \rangle|^2)
\end{align*}
pour un état général $\ket{\psi}$.\\ \\
La méthode ensuite consiste à créer les quatre états de Bell, qui sont les quatre états
de superposition équilibrés pour un système à deux qubits.
Ensuite pour chacun de ces états, on mesure les composantes selon les différentes bases,
puis on calcule l'énergie du système via les proportions de chacune des mesures.
\begin{figure}[H]
    \centering
    \[\shorthandoff{!}
    \scalebox{1.0}{
        \Qcircuit @C=1.0em @R=0.2em @!R { \\
        \nghost{{q}_{0} :  } & \lstick{{q}_{0} :  } & \gate{\mathrm{H}} & \ctrl{1} & \qw & \qw\\
        \nghost{{q}_{1} :  } & \lstick{{q}_{1} :  } & \qw & \targ & \qw & \qw\\
        \\ }}
    \scalebox{1.0}{
        \Qcircuit @C=1.0em @R=0.2em @!R { \\
        \nghost{{q}_{0} :  } & \lstick{{q}_{0} :  } & \gate{\mathrm{X}} & \gate{\mathrm{H}} & \ctrl{1} & \qw & \qw\\
        \nghost{{q}_{1} :  } & \lstick{{q}_{1} :  } & \qw & \qw & \targ & \qw & \qw\\
        \\ }}
    \]
    \[\shorthandoff{!}
    \scalebox{1.0}{
        \Qcircuit @C=1.0em @R=0.2em @!R { \\
        \nghost{{q}_{0} :  } & \lstick{{q}_{0} :  } & \gate{\mathrm{H}} & \ctrl{1} & \qw & \qw\\
        \nghost{{q}_{1} :  } & \lstick{{q}_{1} :  } & \gate{\mathrm{X}} & \targ & \qw & \qw\\
        \\ }}
    \scalebox{1.0}{
        \Qcircuit @C=1.0em @R=0.2em @!R { \\
        \nghost{{q}_{0} :  } & \lstick{{q}_{0} :  } & \gate{\mathrm{H}} & \gate{\mathrm{Z}} & \ctrl{1} & \qw & \qw\\
        \nghost{{q}_{1} :  } & \lstick{{q}_{1} :  } & \gate{\mathrm{X}} & \gate{\mathrm{Z}} & \targ & \qw & \qw\\
        \\ }}
    \]
    \caption{Création des états de Bell $\ket{\Phi^+}$, $\ket{\Phi^-}$, $\ket{\Psi^+}$ et $\ket{\Psi^-}$}
    \label{fig:circ-bell}
\end{figure}
\begin{figure}[H]
    \centering
    \[\shorthandoff{!}
        \scalebox{1.0}{
            \Qcircuit @C=1.0em @R=0.2em @!R { \\
            \nghost{{q}_{0} :  } & \lstick{{q}_{0} :  } \barrier[0em]{1} & \qw & \meter & \qw & \qw & \qw\\
            \nghost{{q}_{1} :  } & \lstick{{q}_{1} :  } & \qw & \qw & \meter & \qw & \qw\\
            \nghost{\mathrm{{meas} :  }} & \lstick{\mathrm{{meas} :  }} & \lstick{/_{_{2}}} \cw & \dstick{_{_{\hspace{0.0em}0}}} \cw \ar @{<=} [-2,0] & \dstick{_{_{\hspace{0.0em}1}}} \cw \ar @{<=} [-1,0] & \cw & \cw\\
            \\ }}
        \scalebox{1.0}{
            \Qcircuit @C=1.0em @R=0.2em @!R { \\
            \nghost{{q}_{0} :  } & \lstick{{q}_{0} :  } & \gate{\mathrm{H}} \barrier[0em]{1} & \qw & \meter & \qw & \qw & \qw\\
            \nghost{{q}_{1} :  } & \lstick{{q}_{1} :  } & \gate{\mathrm{H}} & \qw & \qw & \meter & \qw & \qw\\
            \nghost{\mathrm{{meas} :  }} & \lstick{\mathrm{{meas} :  }} & \lstick{/_{_{2}}} \cw & \cw & \dstick{_{_{\hspace{0.0em}0}}} \cw \ar @{<=} [-2,0] & \dstick{_{_{\hspace{0.0em}1}}} \cw \ar @{<=} [-1,0] & \cw & \cw\\
            \\ }}
    \]
    \[\shorthandoff{!}
        \scalebox{1.0}{
            \Qcircuit @C=1.0em @R=0.2em @!R { \\
            \nghost{{q}_{0} :  } & \lstick{{q}_{0} :  } & \gate{\mathrm{S^\dagger}} & \gate{\mathrm{H}} \barrier[0em]{1} & \qw & \meter & \qw & \qw & \qw\\
            \nghost{{q}_{1} :  } & \lstick{{q}_{1} :  } & \gate{\mathrm{S^\dagger}} & \gate{\mathrm{H}} & \qw & \qw & \meter & \qw & \qw\\
            \nghost{\mathrm{{meas} :  }} & \lstick{\mathrm{{meas} :  }} & \lstick{/_{_{2}}} \cw & \cw & \cw & \dstick{_{_{\hspace{0.0em}0}}} \cw \ar @{<=} [-2,0] & \dstick{_{_{\hspace{0.0em}1}}} \cw \ar @{<=} [-1,0] & \cw & \cw\\
            \\ }}
    \]
    \caption{Circuit de mesure selon la base $ZZ$, $XX$, $YY$}
    \label{fig:circ-meas-double}
\end{figure}
Les états de Bell sont traités ensuite en les mesurant selon les bases $ZZ$, $XX$ et $YY$,
$2^{16}$ fois pour chaque base.
Nous dénommons $p_{i}$ la proportion de mesure de l'état $i$, soit $p_{i} = \frac{\# i}{2^{16}}$
avec $i \in \{00, 01, 10, 11\}$.
Ensuite, on sait que l'énergie est donnée par la formule suivante, déduite des équations
présentées plus haut :
\[
    E = A (p_{00} - p_{01} - p_{10} + p_{11})
\]
ce qui nous donne par simulation que pour les états $\ket{\Phi^+}$, $\ket{\Phi^-}$ et $\ket{\Psi^+}$,
l'énergie est de $E \approx 1.47 \cdot 10^{-6} \ [eV]$ et pour l'état $\ket{\Psi^-}$, l'énergie est
de $E \approx -4.41 \cdot 10^{-6} \ [eV]$.
Cela correspond bien aux valeurs théoriques attendues, de $E = A$ pour les trois premiers états et
$E = -3A$ pour le dernier état~\cite{feynmann-lectures}.\\
Cette différence d'énergie est visible par exemple dans les radiations émises par les atomes
d'hydrogène, qui crée une onde particulière de $21 \ [cm]$ de longueur d'onde.
Cette onde peut être déduite de la formule $E = \hbar f$ avec $E$ la différence d'énergie entre
ces états hyperfins, soit $E = 4A$ et $f$ la fréquence de l'onde émise.
Les radiations se propageant à la vitesse de la lumière, on peut déduire la longueur d'onde
via $f = \frac{c}{\omega}$ avec $c$ la vitesse de la lumière et $\omega$ la longueur d'onde.
Ainsi, on obtient $\omega = \frac{c}{f} = \frac{c \hbar}{E} = \frac{c \hbar}{4A} \approx 21 \ [cm]$.\\ \\
Cette simulation simple à de plus l'avantage de pouvoir être menée sur un ordinateur quantique
actuel, qui via les technologies de réductions d'erreur permet des résultats avec une différence
de $\pm 2 \ \%$\footnote{Résultat basé sur une simulation du bruit d'un ordinateur quantique selon le modèle \textit{FakeVigo}}.
Cela permet de donner de bons espoirs pour des simulations plus complexes, comme par exemple
des molécules ou des propriétés quantiques des matériaux, sur des ordinateurs quantiques dans un futur proche,
en ne resolvant plus le système mathématiquement ou en effectuant des calculs qui grossissent de manière exponentielle
sur un ordinateur classique, mais en le simulant sur un ordinateur quantique.

\section{Évolution temporelle d'un système}\label{sec:evolution-temporel-d'un-systeme}

Nous pouvons également étudier l'évolution temporelle d'un système quantique.
Si l'on considère un hamiltonien $\hat{H}$ qui ne dépend pas du temps, l'équation de Schrödinger
\[
    \hat{H} \ket{\Psi} = i \hbar \pdv{t} \ket{\Psi}
\]
implique, connaissant l'état du système pour une valeur initiale $t = 0$, l'opérateur d'évolution
temporelle sera donné par $U(t) = e^{-iHt/\hbar}$ et on a donc
\[
    \ket{\Psi(t)} = U(t) \ket{\Psi(0)} = e^{-iHt/\hbar} \ket{\Psi(0)}
\]
avec l'exponentielle définie généralement par la série de Taylor, ce qui permet de l'étendre à d'autres
objets que des scalaires.\\ \\
Étudions par exemple un système de particules quantiques dont on désire étudier l'évolution de leur
spin.
Ce genre de système est très vite coûteux à simuler classiquement, car chaque particule intéragit avec
les autres et donc le vecteur d'état du système est de dimension $2^n$ avec $n$ le nombre de particules,
ce qui est une croissance exponentielle.
En revanche, on peut sur un ordinateur quantique utiliser un qubit pour représenter le spin de chaque
particule, et faire correspondre l'état $\ket{0}$ du qubit à un spin négatif et l'état $\ket{1}$ à un
spin positif, ce qui nous permet de simuler le système avec $n$ qubits, soit une croissance linéaire.\\
On peut étudier cela avec un modèle simple, le modèle d'Ising de champ transverse, qui est un modèle
de spins quantiques qui interagissent entre eux sur un réseau.
L'hamiltonien de ce modèle est donné par
\[
    H = \sum_{i=1}^{N} a_i X_i + \sum_{i=1}^{N} \sum_{j=1}^{i-1}J_{ij}Z_i Z_j
\]
avec $X_i$ et $Z_i$ les matrices de Pauli pour le qubit $i$, et $a_i$, $J_{ij}$ des coefficients
qui dépendent du système étudié.
L'évolution temporelle de ce système est donnée par l'opérateur d'évolution
\[
    U(t) = e^{-iHt/\hbar} = e^{-it/\hbar \sum_{i=1}^{N} a_i X_i + \sum_{i=1}^{N} \sum_{j=1}^{i-1}J_{ij}Z_i Z_j}
\]
et en posant pour simplifier $a_i/\hbar = 1$ et $J_{ij}/\hbar = 1$, on peut décomposer cet opérateur en
\[
    U(t) = e^{-it \sum_{i=1}^{N} X_i} e^{-it \sum_{i=1}^{N} \sum_{j=1}^{i-1}Z_i Z_j}
\]
qui peut être résumé en une série de portes quantiques et d'intéractions entre qubits.\\
La première partie de l'opérateur d'évolution est une rotation autour de l'axe $X$ pour chaque qubit,
qui est $e^{-it \sum_{i=1}^{N} X_i}$ et peut-être simulée par une porte $R_x(\theta)$ pour chaque qubit,
sachant que $R_x(\theta) = e^{-i\theta/2 X}$.
En posant $\theta = 2t$, on obtient donc la porte $R_x(2t)$ pour chaque qubit.
Pour la seconde partie de l'opérateur d'évolution, on peut la décomposer en portes $CX$ pour lier les qubits
et une porte $R_z(\theta)$ pour chaque qubit, sachant que $R_z(\theta) = e^{-i\theta/2 Z}$, avec $\theta = 2t$~\cite{cqp-ethz}.\\ \\
Voyons donc ce que cela donne en pratique pour un système de 3 particules dont on veut étudier l'évolution
par étapes de $\Delta t = 0.1$.
\begin{figure}[H]
    \centering
    \[\shorthandoff{!}
        \scalebox{1.0}{
        \Qcircuit @C=1.0em @R=0.2em @!R { \\
                \nghost{{q}_{0} :  } & \lstick{{q}_{0} :  } & \gate{\mathrm{R_X}\,(\mathrm{-0.2})} & \ctrl{1} & \qw & \ctrl{1} & \ctrl{2} & \qw & \ctrl{2} & \qw & \qw & \qw & \qw & \qw\\
                \nghost{{q}_{1} :  } & \lstick{{q}_{1} :  } & \gate{\mathrm{R_X}\,(\mathrm{-0.2})} & \targ & \gate{\mathrm{R_Z}\,(\mathrm{0.2})} & \targ & \qw & \qw & \qw & \ctrl{1} & \qw & \ctrl{1} & \qw & \qw\\
                \nghost{{q}_{2} :  } & \lstick{{q}_{2} :  } & \gate{\mathrm{R_X}\,(\mathrm{-0.2})} & \qw & \qw & \qw & \targ & \gate{\mathrm{R_Z}\,(\mathrm{0.2})} & \targ & \targ & \gate{\mathrm{R_Z}\,(\mathrm{0.2})} & \targ & \qw & \qw\\
        \\ }}
    \]
    \caption{Opérateur d'évolution temporelle pour un système de 3 particules selon le modèle d'Ising, avec $\Delta t = 0.1$ et autres paramètres égaux à 1}
    \label{fig:ising-evol}
\end{figure}
En l'appliquant successivement, on obtient l'évolution temporelle du système à chaque étape selon ce modèle.
Sur la figure~\ref{fig:ising-evol-plot}, on peut voir l'évolution en partant de l'état $\ket{101}$, avec
la mesure des qubits selon la base $Z$.
Il faut donc interpréter les résultats comme la projection du spin de la particule selon l'axe $Z$,
avec un spin $\ket{-}$ si la mesure est 0 et un spin $\ket{+}$ si la mesure est 1, avec les états
de superposition pour les autres états.
On note que pour avoir une telle évolution, il faut réaliser plusieurs mesures du circuit pour avoir un
résultat plus précis, car chaque mesure donne $\ket{0}$ ou $\ket{1}$, et donc pour avoir
une bonne représentation de la superposition, il faut réaliser plusieurs mesures et faire des statistiques.
Via la relation évoquée plus haut $\langle Z \rangle = |\langle 0|q \rangle|^2-|\langle 1|q \rangle|^2$, et
comme on a utilisé la convention $\ket{1} \sim \ket{+}$, on traite les mesures pour un qubit comme
$\expval{Z} = \frac{\#1-\#0}{N} = \frac{2 \cdot \#1}{N} - 1$ par $\#1+\#0=N$ avec $N$ le nombre de mesures et
$\#1$ le nombre de mesures de $\ket{1}$ (resp. $\#0$ pour $\ket{0}$).
\begin{figure}[H]
    \centering
    \scalebox{0.3}{\import{images/algo/mod}{ising.tex}}
    \caption{Évolution temporelle du système de 3 particules selon le modèle d'Ising, avec $\Delta t = 0.1$ et autres paramètres égaux à 1}
    \label{fig:ising-evol-plot}
\end{figure}
Une telle simulation de système quantique peut être d'un intérêt majeur pour l'étude de matériaux~\cite{Ma2020}
ou de molécules~\cite{Ollitrault2021}, car elle permet de simuler des systèmes complexes de manière efficace et
moins coûteuse que sur un ordinateur classique pour des systèmes de grande taille.
