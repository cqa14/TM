\chapter{Algorithme de Grover}\label{ch:algorithme-de-grover}

Un autre domaine d'intérêts de l'ordinateur quantique serait dans de l'organisation logistique.
L'algorithme de Grover~\cite{grover-article, wiki:grover} en est un bon exemple, car il permet de trouver un élément dans une base
de données non triée.
Comme discuté en introduction, la complexité de manière classique est $\order{n}$ dans le cas
d'une base de données non structurée, alors que l'algorithme de Grover permet de trouver
l'élément en $\order{\sqrt{n}}$.\\ \\
L'exemple type de problème de recherche dans une base de données est le problème de la recherche
d'un numéro de téléphone dans un annuaire téléphonique.
En effet, si l'annuaire référence les noms de manière alphabétique, trouver un numéro de téléphone
revient à rechercher quelque chose sans pouvoir se situer dans l'annuaire.\\
De plus, l'algorithme de Grover permet d'accélérer la recherche d'une solution à un problème
dont on sait vérifier la validité d'une solution, ceux de classe $NP$\@.
Malgré cela, un problème de classe $NP$ est généralement exponentiel avec le nombre d'éléments
mis en jeu dans le problème (typiquement $N=2^n$ si $n$ est le nombre d'éléments), ce qui fait
que vis à vis d'un ordinateur classique le résolvant en $2^n$, l'algorithme de Grover ne permet
que de le résoudre en $\sqrt{2^n} = 2^{\frac{n}{2}}$ qui demeure exponentiel.\\
On peut citer plusieurs problèmes type pouvant être résolus par l'algorithme de Grover, comme
les sudokus, les problèmes de satisfiabilité, les problèmes de coloration de graphes, etc.

\section{Principe}\label{sec:principe2}

L'algorithme de Grover est constitué de trois étapes principales, comme illustré sur la
figure~\ref{fig:grov-principe}.
Il y a tout d'abord la préparation de l'espace de recherche, qui correspond à toutes les
solutions possibles du problème, puis l'application d'un oracle qui permet de marquer
les solutions valides, et enfin l'application d'un opérateur de diffusion qui permet
d'amplifier l'effet de l'oracle afin de pouvoir mesurer la solution à la fin de
l'algorithme.
\begin{figure}[H]
    \centering
    \[\shorthandoff{!}
    \scalebox{1.0}{
        \Qcircuit @C=1.0em @R=0.2em @!R { \\
        \nghost{\ket{0}} & \lstick{\ket{0}} & {/^n} \qw & \gate{\mathrm{H^{\otimes n}}} & \gate{Oracle} & \gate{Diffuser} & \meter & \qw \gategroup{2}{5}{2}{6}{.7em}{--} \\
        & & & &  \mbox{$\qquad \qquad \quad$ Répéter $\order{\sqrt{n}}$ fois} & & & \\
        \\ }}
    \]
    \caption{Principe de l'algorithme de Grover}
    \label{fig:grov-principe}
\end{figure}
Au début, on obtient un espace de recherche qui correspond à une superposition de tous
les états possibles, $\ket{s} = \frac{1}{\sqrt{N}} \sum_{x=0}^{N-1} \ket{x}$, avec une
même probabilité, $\frac{1}{N} = \frac{1}{2^n}$, d'obtenir n'importe quel état.
Dans ce cas, on a une chance sur $2^n$ de mesurer la bonne solution, soit en moyenne
$2^{n-1}$ essais pour trouver la solution.\\
La procédure d'amplification vise à augmenter l'amplitude de la bonne solution, et
diminuer celle des mauvaises solutions, jusqu'à avoir une probabilité presque certaine.
De plus, cet algorithme a une jolie interprétation géométrique, via des réflexions
qui entraînent des rotations dans le plan.\\
Considérons donc deux vecteurs $\ket{w}$ et $\ket{s}$, avec $\ket{w}$ le vecteur
correspondant aux solutions valides, et $\ket{s}$ le vecteur de départ.
Ils forment donc un plan à deux dimensions dans l'espace de Hilbert $\mathbb{C}^N$.
Ils ne sont néanmoins pas orthogonaux, et on peut donc définir un vecteur $\ket{s'}$
en enlevant $\ket{w}$ à $\ket{s}$ et en normalisant le résultat.
\begin{enumerate}
    \item On commence l'algorithme par une superposition uniforme de tous les états $\ket{s}$.
        \import{images/algo/grover}{step1-geo.tex}
        Le graphique de gauche correspond au plan défini par $\ket{w}$ et $\ket{s'}$ qui permet
        d'exprimer l'état initial $\ket{s} = \sin \theta \ket{w} + \cos \theta \ket{s'}$, avec
        $\theta = \arcsin \braket{s}{w} = \arcsin \frac{1}{\sqrt{N}}$.
        Le graphique de droite montre les amplitudes de l'état $\ket{s}$.
    \item On applique ensuite l'oracle $U_f$.
        \import{images/algo/grover}{step2-geo.tex}
        Géométriquement, cela correspond à une réflexion de $\ket{s}$ autour de $\ket{s'}$.
        De fait, l'amplitude de $\ket{w}$ devient négative, l'amplitude moyenne diminue donc.
    \item On applique finalement le diffuseur $U_s$, qui correspond à une réflexion autour
        de $\ket{s}$.
        \import{images/algo/grover}{step3-geo.tex}
        Les deux réflexions successives correspondent toujours à une rotation dans le plan.
        La transformation $U_s U_f$ va donc faire tourner le vecteur $\ket{s}$ plus près de $\ket{w}$.
        L'action de $U_s$ peut être prise comme une réflexion sur l'amplitude moyenne, ce qui va
        booster l'amplitude de $\ket{w}$ qui est négative et lui faire prendre près de trois fois sa
        valeur précédente, tout en diminuant les autres amplitudes.
\end{enumerate}
On répète ensuite les deux dernières étapes à plusieurs reprises, jusqu'à ce que la probabilité
de mesurer $\ket{w}$ soit suffisamment grande.\\
Après $t$ étapes, nous sommes donc dans l'état $\ket{\Psi_t} = (U_s U_f)^t \ket{s}$.
Pour trouver le nombre idéal $t$ d'étapes, on peut utiliser la formule suivante :
\[
    t = \lfloor \frac{\pi}{4} \sqrt{\frac{N}{m}} \rfloor
\]
où $m$ est le nombre de solutions valides, et $N$ la taille de l'espace de recherche.
Cette formule est démontrable via la représentation géométrique de l'algorithme.\\ \\
Pour l'oracle $U_f$, il faut créer un opérateur qui inverse le signe si l'état sur lequel
on l'applique est une solution valide.
On peut le faire à partir d'une fonction qui vaut 0 si l'état n'est pas une solution valide,
et 1 sinon, et via le retour de phase, comme pour l'algorithme de Deutsch-Jozsa, on peut la
transformer en $U_f \ket{x} = (-1)^{f(x)} \ket{x}$.\\ \\
Le diffuseur $U_s$ peut être construit très simplement, en appliquant d'abord les mêmes portes
que pour initialiser l'état $\ket{s}$, ce qui a pour effet de le ramener à l'état $\ket{0}$,
et il suffit donc de créer une symétrie par rapport à $\ket{0}$, ce qui peut être fait avec
en encadrant une porte multicontrolée $Z$ avec des portes $X$.
Finalement, on remet le circuit dans l'état du début en appliquant l'inverse de l'initialisation.

\section{Comparaison avec une implémentation classique}\label{sec:comparaison-avec-une-implementation-classique}

Nous pouvons appliquer l'algorithme de Grover avec différentes tailles d'entrées, et comparer
avec une implémentation classique.
L'implémentation classique consiste à tester toutes les solutions possibles, et à s'arrêter
quand on en trouve une qui fonctionne.
Celle quantique est similaire à celle classique car il faut une logique similaire pour l'oracle,
mais il intègre ensuite le diffuseur.
Afin de faciliter la compréhension, nous choisirons des oracles faciles à implémenter.

\subsection{2 entrées}\label{subsec:2-entrees}

Avec deux entrées, nous allons créer un circuit qui détecte si les deux entrées sont à 1.
L'oracle sera donc simplement une $CZ$ qui change le signe de l'état si les deux entrées sont
à 1.
Le circuit d'amplification sera également comme décrit précédemment, avec une initialisation
à l'aide de portes d'Hadamard.\\
Afin de connaitre le nombre d'itérations à effectuer, nous allons utiliser la formule
de l'angle $\theta = \arcsin \frac{1}{\sqrt{N}}$, avec $N = 4$.
Ainsi $\theta = \frac{\pi}{6}$, et après $t$ étapes, nous avons l'état
$(U_s U_f)^t = \sin \theta_t \ket{w} + \cos \theta_t \ket{s'}$ et l'on sait que
$\theta_t = (2t + 1) \theta$ par la figure~\ref{fig:step3-geo}.
Pour obtenir une probabilité de mesure de $\ket{w}$ de 1, il faut donc $\theta_t = \frac{\pi}{2}$,
ce qui donne connaissant $\theta = \frac{\pi}{6}$ que $t = 1$.
On voit avec cela que l'on aura une réponse exacte en une seule itération, qui est la mesure
de $\ket{11}$.
\begin{figure}[H]
    \centering
    \[\shorthandoff{!}
    \scalebox{1.0}{
        \Qcircuit @C=1.0em @R=0.2em @!R { \\
        \nghost{{q}_{0} :  } & \lstick{{q}_{0} :  } & \gate{\mathrm{H}} \barrier[0em]{1} & \qw & \ctrl{1} \barrier[0em]{1} & \qw & \gate{\mathrm{H}} & \gate{\mathrm{X}} & \ctrl{1} & \gate{\mathrm{X}} & \gate{\mathrm{H}} \barrier[0em]{1} & \qw & \meter & \qw & \qw & \qw\\
        \nghost{{q}_{1} :  } & \lstick{{q}_{1} :  } & \gate{\mathrm{H}} & \qw & \control\qw & \qw & \gate{\mathrm{H}} & \gate{\mathrm{X}} & \control\qw & \gate{\mathrm{X}} & \gate{\mathrm{H}} & \qw & \qw & \meter & \qw & \qw\\
        \nghost{\mathrm{{meas} :  }} & \lstick{\mathrm{{meas} :  }} & \lstick{/_{_{2}}} \cw & \cw & \cw & \cw & \cw & \cw & \cw & \cw & \cw & \cw & \dstick{_{_{\hspace{0.0em}0}}} \cw \ar @{<=} [-2,0] & \dstick{_{_{\hspace{0.0em}1}}} \cw \ar @{<=} [-1,0] & \cw & \cw\\
        \\ }}
    \]
    \caption{Algorithme de Grover avec 2 entrées, sortie 11}
    \label{fig:grov-2}
\end{figure}
On peut également préciser que la recherche classique se ferait avec une porte $AND$ entre les
deux entrées, et que l'on doit les tester toutes si on veut être sûr de trouver la bonne,
et en moyenne la moitié avec de la chance.

\subsection{3 entrées}\label{subsec:3-entrees}

Dans le circuit suivant, on remarque les mêmes patterns que dans le circuit précédent.
\begin{figure}[H]
    \centering
    \[\shorthandoff{!}
    \scalebox{1.0}{
        \Qcircuit @C=1.0em @R=0.2em @!R { \\
        \nghost{{q}_{0} :  } & \lstick{{q}_{0} :  } & \gate{\mathrm{H}} \barrier[0em]{2} & \qw & \ctrl{1} & \ctrl{2} \barrier[0em]{2} & \qw & \gate{\mathrm{H}} & \gate{\mathrm{X}} & \ctrl{1} & \gate{\mathrm{X}} & \gate{\mathrm{H}} \barrier[0em]{2} & \qw \barrier[0em]{2} & \qw & \meter & \qw & \qw & \qw & \qw\\
        \nghost{{q}_{1} :  } & \lstick{{q}_{1} :  } & \gate{\mathrm{H}} & \qw & \control\qw & \qw & \qw & \gate{\mathrm{H}} & \gate{\mathrm{X}} & \ctrl{1} & \gate{\mathrm{X}} & \gate{\mathrm{H}} & \qw & \qw & \qw & \meter & \qw & \qw & \qw\\
        \nghost{{q}_{2} :  } & \lstick{{q}_{2} :  } & \gate{\mathrm{H}} & \qw & \qw & \control\qw & \qw & \gate{\mathrm{H}} & \gate{\mathrm{X}} & \control\qw & \gate{\mathrm{X}} & \gate{\mathrm{H}} & \qw & \qw & \qw & \qw & \meter & \qw & \qw\\
        \nghost{\mathrm{{meas} :  }} & \lstick{\mathrm{{meas} :  }} & \lstick{/_{_{3}}} \cw & \cw & \cw & \cw & \cw & \cw & \cw & \cw & \cw & \cw & \cw & \cw & \dstick{_{_{\hspace{0.0em}0}}} \cw \ar @{<=} [-3,0] & \dstick{_{_{\hspace{0.0em}1}}} \cw \ar @{<=} [-2,0] & \dstick{_{_{\hspace{0.0em}2}}} \cw \ar @{<=} [-1,0] & \cw & \cw\\
        \\ }}
    \]
    \caption{Algorithme de Grover avec 3 entrées}
    \label{fig:grov-3}
\end{figure}
Pour le parallèle avec l'ordinateur classique, on va essayer de trouver les sorties escomptées.
On voit que ce coup-ci, l'oracle est composé de deux $CZ$, l'une entre les deux premières entrées,
et l'autre entre la première et la troisième.
De fait, si seuls les deux premiers qubits sont à 1, alors le premier $CZ$ va inverser la phase
et cela devient un résultat possible.
De même pour la seconde porte, et finalement si les deux agissent en même temps, alors les
effets s'annulent et on se retrouve avec une phase inchangée, ce qui fait que ce n'est pas
une des solutions possibles.\\
De fait, on en conclut que l'on devrait mesurer soit $\ket{011}$, soit $\ket{101}$.
De plus, on peut calculer $t = \lfloor \frac{\pi}{4} \sqrt{\frac{N}{m}} \rfloor = 1$
avec $N = 2^3$ et $m=2$.
La simulation du circuit nous donne bien les deux résultats désirés.
\begin{figure}[H]
    \centering
    \import{images/algo/grover}{grover_3bit.tex}
    \caption{Mesure de l'algorithme de Grover avec 3 entrées}
    \label{fig:grov-3-sim}
\end{figure}

\subsection{4 entrées}\label{subsec:4-entrees}

Pour celui à 4 entrées, ça se complique.
Considérons le circuit suivant, qui a un oracle dont on ne connait pas le nombre de solutions.
\begin{figure}[H]
    \centering
    \[\shorthandoff{!}
    \scalebox{1.0}{
        \Qcircuit @C=1.0em @R=0.2em @!R { \\
        \nghost{{q}_{0} :  } & \lstick{{q}_{0} :  } & \gate{\mathrm{H}} \barrier[0em]{3} & \qw & \ctrl{1} & \ctrl{2} & \ctrl{3} \barrier[0em]{3} & \qw & \gate{\mathrm{H}} & \gate{\mathrm{X}} & \ctrl{1} & \gate{\mathrm{X}} & \gate{\mathrm{H}} \barrier[0em]{3} & \qw \barrier[0em]{3} & \qw & \meter & \qw & \qw & \qw & \qw & \qw\\
        \nghost{{q}_{1} :  } & \lstick{{q}_{1} :  } & \gate{\mathrm{H}} & \qw & \control\qw & \qw & \qw & \qw & \gate{\mathrm{H}} & \gate{\mathrm{X}} & \ctrl{1} & \gate{\mathrm{X}} & \gate{\mathrm{H}} & \qw & \qw & \qw & \meter & \qw & \qw & \qw & \qw\\
        \nghost{{q}_{2} :  } & \lstick{{q}_{2} :  } & \gate{\mathrm{H}} & \qw & \qw & \control\qw & \qw & \qw & \gate{\mathrm{H}} & \gate{\mathrm{X}} & \ctrl{1} & \gate{\mathrm{X}} & \gate{\mathrm{H}} & \qw & \qw & \qw & \qw & \meter & \qw & \qw & \qw\\
        \nghost{{q}_{3} :  } & \lstick{{q}_{3} :  } & \gate{\mathrm{H}} & \qw & \qw & \qw & \control\qw & \qw & \gate{\mathrm{H}} & \gate{\mathrm{X}} & \control\qw & \gate{\mathrm{X}} & \gate{\mathrm{H}} & \qw & \qw & \qw & \qw & \qw & \meter & \qw & \qw\\
        \nghost{\mathrm{{meas} :  }} & \lstick{\mathrm{{meas} :  }} & \lstick{/_{_{4}}} \cw & \cw & \cw & \cw & \cw & \cw & \cw & \cw & \cw & \cw & \cw & \cw & \cw & \dstick{_{_{\hspace{0.0em}0}}} \cw \ar @{<=} [-4,0] & \dstick{_{_{\hspace{0.0em}1}}} \cw \ar @{<=} [-3,0] & \dstick{_{_{\hspace{0.0em}2}}} \cw \ar @{<=} [-2,0] & \dstick{_{_{\hspace{0.0em}3}}} \cw \ar @{<=} [-1,0] & \cw & \cw\\
        \\ }}
    \]
    \caption{Algorithme de Grover avec 4 entrées}
    \label{fig:grov-4}
\end{figure}
Rappelons-nous que l'application $U_s U_f$ fait une rotation d'un angle de $2 \theta$.
Or, on sait retrouver l'angle d'un opérateur en utilisant l'estimation de phase.
De fait, en créant un circuit avec cette opération que l'on contrôle, on peut connaitre
l'angle de rotation provoqué par l'opérateur~\cite{wiki:quant-count}.
\begin{figure}[H]
    \centering
    \[\shorthandoff{!}
    \scalebox{0.7}{
        \Qcircuit @C=1.0em @R=0.2em @!R { \\
        \nghost{{q}_{0} :  } & \lstick{{q}_{0} :  } & \gate{\mathrm{H}} & \ctrl{5} & \qw & \qw & \qw & \qw & \multigate{4}{\mathrm{QFT^\dagger}}_<<<{0} & \meter & \qw & \qw & \qw & \qw & \qw & \qw\\
        \nghost{{q}_{1} :  } & \lstick{{q}_{1} :  } & \gate{\mathrm{H}} & \qw & \ctrl{4} & \qw & \qw & \qw & \ghost{\mathrm{QFT^\dagger}}_<<<{1} & \qw & \meter & \qw & \qw & \qw & \qw & \qw\\
        \nghost{{q}_{2} :  } & \lstick{{q}_{2} :  } & \gate{\mathrm{H}} & \qw & \qw & \ctrl{3} & \qw & \qw & \ghost{\mathrm{QFT^\dagger}}_<<<{2} & \qw & \qw & \meter & \qw & \qw & \qw & \qw\\
        \nghost{{q}_{3} :  } & \lstick{{q}_{3} :  } & \gate{\mathrm{H}} & \qw & \qw & \qw & \ctrl{2} & \qw & \ghost{\mathrm{QFT^\dagger}}_<<<{3} & \qw & \qw & \qw & \meter & \qw & \qw & \qw\\
        \nghost{{q}_{4} :  } & \lstick{{q}_{4} :  } & \gate{\mathrm{H}} & \qw & \qw & \qw & \qw & \ctrl{1} & \ghost{\mathrm{QFT^\dagger}}_<<<{4} & \qw & \qw & \qw & \qw & \meter & \qw & \qw\\
        \nghost{{q}_{5} :  } & \lstick{{q}_{5} :  } & \gate{\mathrm{H}} & \multigate{3}{\mathrm{Grover\string^1}}_<<<{0} & \multigate{3}{\mathrm{Grover\string^2}}_<<<{0} & \multigate{3}{\mathrm{Grover\string^4}}_<<<{0} & \multigate{3}{\mathrm{Grover\string^8}}_<<<{0} & \multigate{3}{\mathrm{Grover\string^16}}_<<<{0} & \qw & \qw & \qw & \qw & \qw & \qw & \qw & \qw\\
        \nghost{{q}_{6} :  } & \lstick{{q}_{6} :  } & \gate{\mathrm{H}} & \ghost{\mathrm{Grover\string^1}}_<<<{1} & \ghost{\mathrm{Grover\string^2}}_<<<{1} & \ghost{\mathrm{Grover\string^4}}_<<<{1} & \ghost{\mathrm{Grover\string^8}}_<<<{1} & \ghost{\mathrm{Grover\string^16}}_<<<{1} & \qw & \qw & \qw & \qw & \qw & \qw & \qw & \qw\\
        \nghost{{q}_{7} :  } & \lstick{{q}_{7} :  } & \gate{\mathrm{H}} & \ghost{\mathrm{Grover\string^1}}_<<<{2} & \ghost{\mathrm{Grover\string^2}}_<<<{2} & \ghost{\mathrm{Grover\string^4}}_<<<{2} & \ghost{\mathrm{Grover\string^8}}_<<<{2} & \ghost{\mathrm{Grover\string^16}}_<<<{2} & \qw & \qw & \qw & \qw & \qw & \qw & \qw & \qw\\
        \nghost{{q}_{8} :  } & \lstick{{q}_{8} :  } & \gate{\mathrm{H}} & \ghost{\mathrm{Grover\string^1}}_<<<{3} & \ghost{\mathrm{Grover\string^2}}_<<<{3} & \ghost{\mathrm{Grover\string^4}}_<<<{3} & \ghost{\mathrm{Grover\string^8}}_<<<{3} & \ghost{\mathrm{Grover\string^16}}_<<<{3} & \qw & \qw & \qw & \qw & \qw & \qw & \qw & \qw\\
        \nghost{\mathrm{{c} :  }} & \lstick{\mathrm{{c} :  }} & \lstick{/_{_{5}}} \cw & \cw & \cw & \cw & \cw & \cw & \cw & \dstick{_{_{\hspace{0.0em}0}}} \cw \ar @{<=} [-9,0] & \dstick{_{_{\hspace{0.0em}1}}} \cw \ar @{<=} [-8,0] & \dstick{_{_{\hspace{0.0em}2}}} \cw \ar @{<=} [-7,0] & \dstick{_{_{\hspace{0.0em}3}}} \cw \ar @{<=} [-6,0] & \dstick{_{_{\hspace{0.0em}4}}} \cw \ar @{<=} [-5,0] & \cw & \cw\\
        \\ }}
    \]
    \caption{Recherche de l'angle de rotation de l'opérateur $U_s U_f$}
    \label{fig:grov-4-num}
\end{figure}
En voyant les résultats de la simulation, on peut voir que l'angle ne semble pas être une
valeur que l'on peut calculer exactement avec peu de ressources.
\begin{figure}[H]
    \centering
    \import{images/algo/grover}{grover_4bit_angle.tex}
    \caption{Simulation de l'estimation de phase de l'opérateur de Grover}
    \label{fig:grov-4-est}
\end{figure}
Néanmoins ces résultats sont suffisants pour pouvoir calculer le bon nombre de solutions
possibles.
On observe donc que le nombre le plus mesuré est 27 (11011 en base 2).
La fraction de l'angle est donc $\frac{27}{2^5}$, que l'on multiplie par $2 \pi$ pour obtenir
l'angle de rotation de l'opérateur $U_s U_f$, i.e. $2 \theta = \frac{27}{2^5} 2 \pi \approx 5.3$.
Finalement, pour obtenir le nombre de solutions $m$, on reprend la figure~\ref{fig:step1-geo}.
On constate alors que par définition, $\ket{s} = \sqrt{\frac{m}{N}} \ket{w} + \sqrt{1 - \frac{m}{N}} \ket{s'}$,
de plus $\ket{s} = \sin \theta \ket{w} + \cos \theta \ket{s'}$.
Ainsi nous pouvons conclure que $\braket{s'}{s} = \sqrt{1 - \frac{m}{N}} = \cos \theta \Leftrightarrow m = N \sin^2 \theta$.
De la valeur de $2 \theta$ calculée précédement, on peut donc déduire que $m \approx 3.6$ d'où
l'on peut penser que le nombre de solutions est 4.\\
De cette conclusion, le calcul du nombre d'itérations de l'algorithme de Grover est donc
$t = \lfloor \frac{\pi}{4} \sqrt{\frac{2^4}{4}} \rfloor = 1$.
En faisant une simulation du circuit présenté en figure~\ref{fig:grov-4}, on obtient
bien 4 solutions différentes, qui sont bien les 4 solutions de l'oracle utilisé.
\begin{figure}[H]
    \centering
    \import{images/algo/grover}{grover_4bit.tex}
    \caption{Résultats du circuit~\ref{fig:grov-4}}
    \label{fig:grov-4-sim}
\end{figure}

\subsection{Différence de complexité}\label{subsec:difference-de-complexite}

On voit donc que l'algorithme de Grover permet de trouver une solution à un problème
de manière plus efficace que de manière classique.
Il y a tout d'abord le nombre d'opérations à effectuer qui est réduit, de manière classique
en moyenne la moitié des possibilités doivent être testées donc $2^{n-1}$ opérations,
alors que cette méthode permet de le faire en $2^{\frac{n}{2}}$, ce qui est non-négligeable
quand $n$ est grand.
On peut également noter d'autres avantages.
Par exemple, connaître le nombre de solutions n'est pas possible de manière classique
à moins de tester toutes les solutions possibles, alors que via ce que l'on a vu,
on peut le faire avec un protocole standard avec un ordinateur quantique.

\section{Résolution d'un sudoku}\label{sec:resolution-d'un-sudoku}

Un autre exemple, assez récréatif, est la résolution d'un sudoku.
En effet, un sudoku est un problème de satisfiabilité, et donc résoluble par l'algorithme
de Grover.
Prenons la grille suivante, qui est un sudoku trivial pour illustrer le principe.
\begin{figure}[H]
    \centering
    \begin{tabular}{|c|c|c|}
        \hline
        \textbf{1} & \textit{2} & \textit{3} \\
        \hline
        \textbf{2} & \textit{3} & \textbf{1} \\
        \hline
        \textit{3} & \textbf{1} & \textit{2} \\
        \hline
    \end{tabular}
    \caption{Sudoku à résoudre (réponses attendues en italique)}
    \label{fig:sudoku}
\end{figure}
Notons que la grille est choisie de sorte à éviter d'encoder chaque nombre sur 2 bits,
car tous les 1 sont placés, ce qui serait au-delà des ressources disponibles, car cela
demanderait soit un superordinateur, soit un ordinateur quantique à la pointe de la technologie.\\ \\
On va donc établir une convention pour encoder les nombres : les 2 vont apparaître comme des
0 et les 3 comme des 1 dans la sortie.
De plus, la sortie va donner les nombres par ligne de haut en bas et de gauche à droite, soit
le bit de gauche de la sortie correspond au nombre en haut à gauche de la grille.\\ \\
L'initialisation du circuit se fait en appliquant des portes d'Hadamard sur tous les qubits
qui encodent les nombres à trouver.
Notons également le dernier qubit qui est initialisé à $\ket{-}$ afin de créer un retour
de phase permettant de marquer les solutions valides et donc de créer l'oracle.\\ \\
L'oracle est constitué de porte $CX$ qui vérifient les différentes contraintes du sudoku.
On va donc avoir sur la première vérification de condition une $CX$ avec le qubit correspondant
à en bas à droite de la grille, et une avec le qubit correspondant au nombre en bas à gauche.
De fait, si les deux sont différents comme attendu, le qubit de verification sera inversé, et
sinon il demeurera à 0.
Nous faisons de même pour les autres contraintes, et nous obtenons donc un qubit de vérification
par contrainte.
Pour les cases qui dépendent de données déjà présentes dans la grille, nous initialisons le
qubit de vérification à la valeur de la case, et nous appliquons une $CX$ avec le qubit
qui en dépend.\\
Pour finir, nous utilisons une porte multicontrôlée $X$ pour inverser le qubit visant à créer
le retour de phase si tous les qubits de vérification sont à 1, ce qui correspond à une solution
valide, puis nous réappliquons toutes les portes $CX$ précédentes pour remettre les qubits
de vérification à 0.\\ \\
Le diffuseur est construit de manière similaire aux précédents, outre l'implémentation de
la porte multicontrôlée $Z$ qui est faite via une $X$ dont la cible est encadrée par des
portes $H$.\\ \\
Finalement, de par l'unicité des solutions du sudoku, nous savons que nous n'aurons qu'une
seule solution, et donc il faut appliquer l'algorithme de Grover $t = \lfloor \frac{\pi}{4} \sqrt{2^5} \rfloor = 4$.
\begin{sidewaysfigure}
    \centering
    \[\shorthandoff{!}
        \scalebox{0.26}{
            \Qcircuit @C=1.0em @R=0.2em @!R { \\
            \nghost{{input}_{0} :  } & \lstick{{input}_{0} :  } & \gate{\mathrm{H}} & \qw \barrier[0em]{11} & \qw & \ctrl{5} & \qw & \qw & \ctrl{6} & \qw & \qw & \qw & \qw & \qw & \qw & \qw & \ctrl{5} & \qw & \qw & \ctrl{6} & \qw & \qw & \qw & \qw & \qw & \qw \barrier[0em]{11} & \qw & \gate{\mathrm{H}} & \gate{\mathrm{X}} & \qw & \ctrl{1} & \gate{\mathrm{X}} & \gate{\mathrm{H}} & \qw \barrier[0em]{11} & \qw & \ctrl{5} & \qw & \qw & \ctrl{6} & \qw & \qw & \qw & \qw & \qw & \qw & \qw & \ctrl{5} & \qw & \qw & \ctrl{6} & \qw & \qw & \qw & \qw & \qw & \qw \barrier[0em]{11} & \qw & \gate{\mathrm{H}} & \gate{\mathrm{X}} & \qw & \ctrl{1} & \gate{\mathrm{X}} & \gate{\mathrm{H}} & \qw \barrier[0em]{11} & \qw & \ctrl{5} & \qw & \qw & \ctrl{6} & \qw & \qw & \qw & \qw & \qw & \qw & \qw & \ctrl{5} & \qw & \qw & \ctrl{6} & \qw & \qw & \qw & \qw & \qw & \qw \barrier[0em]{11} & \qw & \gate{\mathrm{H}} & \gate{\mathrm{X}} & \qw & \ctrl{1} & \gate{\mathrm{X}} & \gate{\mathrm{H}} & \qw \barrier[0em]{11} & \qw & \ctrl{5} & \qw & \qw & \ctrl{6} & \qw & \qw & \qw & \qw & \qw & \qw & \qw & \ctrl{5} & \qw & \qw & \ctrl{6} & \qw & \qw & \qw & \qw & \qw & \qw \barrier[0em]{11} & \qw & \gate{\mathrm{H}} & \gate{\mathrm{X}} & \qw & \ctrl{1} & \gate{\mathrm{X}} & \gate{\mathrm{H}} & \qw \barrier[0em]{11} & \qw & \meter & \qw & \qw & \qw & \qw & \qw & \qw\\
            \nghost{{input}_{1} :  } & \lstick{{input}_{1} :  } & \gate{\mathrm{H}} & \qw & \qw & \qw & \qw & \ctrl{4} & \qw & \qw & \qw & \ctrl{6} & \qw & \qw & \qw & \qw & \qw & \qw & \ctrl{4} & \qw & \qw & \qw & \ctrl{6} & \qw & \qw & \qw & \qw & \gate{\mathrm{H}} & \gate{\mathrm{X}} & \qw & \ctrl{1} & \gate{\mathrm{X}} & \gate{\mathrm{H}} & \qw & \qw & \qw & \qw & \ctrl{4} & \qw & \qw & \qw & \ctrl{6} & \qw & \qw & \qw & \qw & \qw & \qw & \ctrl{4} & \qw & \qw & \qw & \ctrl{6} & \qw & \qw & \qw & \qw & \gate{\mathrm{H}} & \gate{\mathrm{X}} & \qw & \ctrl{1} & \gate{\mathrm{X}} & \gate{\mathrm{H}} & \qw & \qw & \qw & \qw & \ctrl{4} & \qw & \qw & \qw & \ctrl{6} & \qw & \qw & \qw & \qw & \qw & \qw & \ctrl{4} & \qw & \qw & \qw & \ctrl{6} & \qw & \qw & \qw & \qw & \gate{\mathrm{H}} & \gate{\mathrm{X}} & \qw & \ctrl{1} & \gate{\mathrm{X}} & \gate{\mathrm{H}} & \qw & \qw & \qw & \qw & \ctrl{4} & \qw & \qw & \qw & \ctrl{6} & \qw & \qw & \qw & \qw & \qw & \qw & \ctrl{4} & \qw & \qw & \qw & \ctrl{6} & \qw & \qw & \qw & \qw & \gate{\mathrm{H}} & \gate{\mathrm{X}} & \qw & \ctrl{1} & \gate{\mathrm{X}} & \gate{\mathrm{H}} & \qw & \qw & \qw & \meter & \qw & \qw & \qw & \qw & \qw\\
            \nghost{{input}_{2} :  } & \lstick{{input}_{2} :  } & \gate{\mathrm{H}} & \qw & \qw & \qw & \ctrl{6} & \qw & \qw & \ctrl{7} & \qw & \qw & \qw & \qw & \qw & \qw & \qw & \ctrl{6} & \qw & \qw & \ctrl{7} & \qw & \qw & \qw & \qw & \qw & \qw & \gate{\mathrm{H}} & \gate{\mathrm{X}} & \qw & \ctrl{1} & \gate{\mathrm{X}} & \gate{\mathrm{H}} & \qw & \qw & \qw & \ctrl{6} & \qw & \qw & \ctrl{7} & \qw & \qw & \qw & \qw & \qw & \qw & \qw & \ctrl{6} & \qw & \qw & \ctrl{7} & \qw & \qw & \qw & \qw & \qw & \qw & \gate{\mathrm{H}} & \gate{\mathrm{X}} & \qw & \ctrl{1} & \gate{\mathrm{X}} & \gate{\mathrm{H}} & \qw & \qw & \qw & \ctrl{6} & \qw & \qw & \ctrl{7} & \qw & \qw & \qw & \qw & \qw & \qw & \qw & \ctrl{6} & \qw & \qw & \ctrl{7} & \qw & \qw & \qw & \qw & \qw & \qw & \gate{\mathrm{H}} & \gate{\mathrm{X}} & \qw & \ctrl{1} & \gate{\mathrm{X}} & \gate{\mathrm{H}} & \qw & \qw & \qw & \ctrl{6} & \qw & \qw & \ctrl{7} & \qw & \qw & \qw & \qw & \qw & \qw & \qw & \ctrl{6} & \qw & \qw & \ctrl{7} & \qw & \qw & \qw & \qw & \qw & \qw & \gate{\mathrm{H}} & \gate{\mathrm{X}} & \qw & \ctrl{1} & \gate{\mathrm{X}} & \gate{\mathrm{H}} & \qw & \qw & \qw & \qw & \meter & \qw & \qw & \qw & \qw\\
            \nghost{{input}_{3} :  } & \lstick{{input}_{3} :  } & \gate{\mathrm{H}} & \qw & \qw & \qw & \qw & \qw & \qw & \qw & \ctrl{3} & \qw & \qw & \ctrl{7} & \qw & \qw & \qw & \qw & \qw & \qw & \qw & \ctrl{3} & \qw & \qw & \ctrl{7} & \qw & \qw & \gate{\mathrm{H}} & \gate{\mathrm{X}} & \qw & \ctrl{1} & \gate{\mathrm{X}} & \gate{\mathrm{H}} & \qw & \qw & \qw & \qw & \qw & \qw & \qw & \ctrl{3} & \qw & \qw & \ctrl{7} & \qw & \qw & \qw & \qw & \qw & \qw & \qw & \ctrl{3} & \qw & \qw & \ctrl{7} & \qw & \qw & \gate{\mathrm{H}} & \gate{\mathrm{X}} & \qw & \ctrl{1} & \gate{\mathrm{X}} & \gate{\mathrm{H}} & \qw & \qw & \qw & \qw & \qw & \qw & \qw & \ctrl{3} & \qw & \qw & \ctrl{7} & \qw & \qw & \qw & \qw & \qw & \qw & \qw & \ctrl{3} & \qw & \qw & \ctrl{7} & \qw & \qw & \gate{\mathrm{H}} & \gate{\mathrm{X}} & \qw & \ctrl{1} & \gate{\mathrm{X}} & \gate{\mathrm{H}} & \qw & \qw & \qw & \qw & \qw & \qw & \qw & \ctrl{3} & \qw & \qw & \ctrl{7} & \qw & \qw & \qw & \qw & \qw & \qw & \qw & \ctrl{3} & \qw & \qw & \ctrl{7} & \qw & \qw & \gate{\mathrm{H}} & \gate{\mathrm{X}} & \qw & \ctrl{1} & \gate{\mathrm{X}} & \gate{\mathrm{H}} & \qw & \qw & \qw & \qw & \qw & \meter & \qw & \qw & \qw\\
            \nghost{{input}_{4} :  } & \lstick{{input}_{4} :  } & \gate{\mathrm{H}} & \qw & \qw & \qw & \qw & \qw & \qw & \qw & \qw & \qw & \ctrl{5} & \qw & \ctrl{6} & \qw & \qw & \qw & \qw & \qw & \qw & \qw & \qw & \ctrl{5} & \qw & \ctrl{6} & \qw & \gate{\mathrm{H}} & \gate{\mathrm{X}} & \gate{\mathrm{H}} & \targ & \gate{\mathrm{H}} & \gate{\mathrm{X}} & \gate{\mathrm{H}} & \qw & \qw & \qw & \qw & \qw & \qw & \qw & \qw & \ctrl{5} & \qw & \ctrl{6} & \qw & \qw & \qw & \qw & \qw & \qw & \qw & \qw & \ctrl{5} & \qw & \ctrl{6} & \qw & \gate{\mathrm{H}} & \gate{\mathrm{X}} & \gate{\mathrm{H}} & \targ & \gate{\mathrm{H}} & \gate{\mathrm{X}} & \gate{\mathrm{H}} & \qw & \qw & \qw & \qw & \qw & \qw & \qw & \qw & \ctrl{5} & \qw & \ctrl{6} & \qw & \qw & \qw & \qw & \qw & \qw & \qw & \qw & \ctrl{5} & \qw & \ctrl{6} & \qw & \gate{\mathrm{H}} & \gate{\mathrm{X}} & \gate{\mathrm{H}} & \targ & \gate{\mathrm{H}} & \gate{\mathrm{X}} & \gate{\mathrm{H}} & \qw & \qw & \qw & \qw & \qw & \qw & \qw & \qw & \ctrl{5} & \qw & \ctrl{6} & \qw & \qw & \qw & \qw & \qw & \qw & \qw & \qw & \ctrl{5} & \qw & \ctrl{6} & \qw & \gate{\mathrm{H}} & \gate{\mathrm{X}} & \gate{\mathrm{H}} & \targ & \gate{\mathrm{H}} & \gate{\mathrm{X}} & \gate{\mathrm{H}} & \qw & \qw & \qw & \qw & \qw & \meter & \qw & \qw\\
            \nghost{{cond}_{0} :  } & \lstick{{cond}_{0} :  } & \qw & \qw & \qw & \targ & \qw & \targ & \qw & \qw & \qw & \qw & \qw & \qw & \qw & \ctrl{1} & \targ & \qw & \targ & \qw & \qw & \qw & \qw & \qw & \qw & \qw & \qw & \qw & \qw & \qw & \qw & \qw & \qw & \qw & \qw & \targ & \qw & \targ & \qw & \qw & \qw & \qw & \qw & \qw & \qw & \ctrl{1} & \targ & \qw & \targ & \qw & \qw & \qw & \qw & \qw & \qw & \qw & \qw & \qw & \qw & \qw & \qw & \qw & \qw & \qw & \qw & \targ & \qw & \targ & \qw & \qw & \qw & \qw & \qw & \qw & \qw & \ctrl{1} & \targ & \qw & \targ & \qw & \qw & \qw & \qw & \qw & \qw & \qw & \qw & \qw & \qw & \qw & \qw & \qw & \qw & \qw & \qw & \targ & \qw & \targ & \qw & \qw & \qw & \qw & \qw & \qw & \qw & \ctrl{1} & \targ & \qw & \targ & \qw & \qw & \qw & \qw & \qw & \qw & \qw & \qw & \qw & \qw & \qw & \qw & \qw & \qw & \qw & \qw & \qw & \qw & \qw & \qw & \qw & \qw & \qw\\
            \nghost{{cond}_{1} :  } & \lstick{{cond}_{1} :  } & \qw & \qw & \qw & \qw & \qw & \qw & \targ & \qw & \targ & \qw & \qw & \qw & \qw & \ctrl{1} & \qw & \qw & \qw & \targ & \qw & \targ & \qw & \qw & \qw & \qw & \qw & \qw & \qw & \qw & \qw & \qw & \qw & \qw & \qw & \qw & \qw & \qw & \targ & \qw & \targ & \qw & \qw & \qw & \qw & \ctrl{1} & \qw & \qw & \qw & \targ & \qw & \targ & \qw & \qw & \qw & \qw & \qw & \qw & \qw & \qw & \qw & \qw & \qw & \qw & \qw & \qw & \qw & \qw & \targ & \qw & \targ & \qw & \qw & \qw & \qw & \ctrl{1} & \qw & \qw & \qw & \targ & \qw & \targ & \qw & \qw & \qw & \qw & \qw & \qw & \qw & \qw & \qw & \qw & \qw & \qw & \qw & \qw & \qw & \qw & \targ & \qw & \targ & \qw & \qw & \qw & \qw & \ctrl{1} & \qw & \qw & \qw & \targ & \qw & \targ & \qw & \qw & \qw & \qw & \qw & \qw & \qw & \qw & \qw & \qw & \qw & \qw & \qw & \qw & \qw & \qw & \qw & \qw & \qw & \qw\\
            \nghost{{cond}_{2} :  } & \lstick{{cond}_{2} :  } & \qw & \qw & \qw & \qw & \qw & \qw & \qw & \qw & \qw & \targ & \qw & \qw & \qw & \ctrl{1} & \qw & \qw & \qw & \qw & \qw & \qw & \targ & \qw & \qw & \qw & \qw & \qw & \qw & \qw & \qw & \qw & \qw & \qw & \qw & \qw & \qw & \qw & \qw & \qw & \qw & \targ & \qw & \qw & \qw & \ctrl{1} & \qw & \qw & \qw & \qw & \qw & \qw & \targ & \qw & \qw & \qw & \qw & \qw & \qw & \qw & \qw & \qw & \qw & \qw & \qw & \qw & \qw & \qw & \qw & \qw & \qw & \targ & \qw & \qw & \qw & \ctrl{1} & \qw & \qw & \qw & \qw & \qw & \qw & \targ & \qw & \qw & \qw & \qw & \qw & \qw & \qw & \qw & \qw & \qw & \qw & \qw & \qw & \qw & \qw & \qw & \qw & \qw & \targ & \qw & \qw & \qw & \ctrl{1} & \qw & \qw & \qw & \qw & \qw & \qw & \targ & \qw & \qw & \qw & \qw & \qw & \qw & \qw & \qw & \qw & \qw & \qw & \qw & \qw & \qw & \qw & \qw & \qw & \qw & \qw\\
            \nghost{{cond}_{3} :  } & \lstick{{cond}_{3} :  } & \qw & \qw & \qw & \qw & \targ & \qw & \qw & \qw & \qw & \qw & \qw & \qw & \qw & \ctrl{1} & \qw & \targ & \qw & \qw & \qw & \qw & \qw & \qw & \qw & \qw & \qw & \qw & \qw & \qw & \qw & \qw & \qw & \qw & \qw & \qw & \targ & \qw & \qw & \qw & \qw & \qw & \qw & \qw & \qw & \ctrl{1} & \qw & \targ & \qw & \qw & \qw & \qw & \qw & \qw & \qw & \qw & \qw & \qw & \qw & \qw & \qw & \qw & \qw & \qw & \qw & \qw & \targ & \qw & \qw & \qw & \qw & \qw & \qw & \qw & \qw & \ctrl{1} & \qw & \targ & \qw & \qw & \qw & \qw & \qw & \qw & \qw & \qw & \qw & \qw & \qw & \qw & \qw & \qw & \qw & \qw & \qw & \qw & \targ & \qw & \qw & \qw & \qw & \qw & \qw & \qw & \qw & \ctrl{1} & \qw & \targ & \qw & \qw & \qw & \qw & \qw & \qw & \qw & \qw & \qw & \qw & \qw & \qw & \qw & \qw & \qw & \qw & \qw & \qw & \qw & \qw & \qw & \qw & \qw & \qw\\
            \nghost{{cond}_{4} :  } & \lstick{{cond}_{4} :  } & \qw & \qw & \qw & \qw & \qw & \qw & \qw & \targ & \qw & \qw & \targ & \qw & \qw & \ctrl{1} & \qw & \qw & \qw & \qw & \targ & \qw & \qw & \targ & \qw & \qw & \qw & \qw & \qw & \qw & \qw & \qw & \qw & \qw & \qw & \qw & \qw & \qw & \qw & \targ & \qw & \qw & \targ & \qw & \qw & \ctrl{1} & \qw & \qw & \qw & \qw & \targ & \qw & \qw & \targ & \qw & \qw & \qw & \qw & \qw & \qw & \qw & \qw & \qw & \qw & \qw & \qw & \qw & \qw & \qw & \targ & \qw & \qw & \targ & \qw & \qw & \ctrl{1} & \qw & \qw & \qw & \qw & \targ & \qw & \qw & \targ & \qw & \qw & \qw & \qw & \qw & \qw & \qw & \qw & \qw & \qw & \qw & \qw & \qw & \qw & \qw & \targ & \qw & \qw & \targ & \qw & \qw & \ctrl{1} & \qw & \qw & \qw & \qw & \targ & \qw & \qw & \targ & \qw & \qw & \qw & \qw & \qw & \qw & \qw & \qw & \qw & \qw & \qw & \qw & \qw & \qw & \qw & \qw & \qw & \qw\\
            \nghost{{cond}_{5} :  } & \lstick{{cond}_{5} :  } & \qw & \qw & \qw & \qw & \qw & \qw & \qw & \qw & \qw & \qw & \qw & \targ & \targ & \ctrl{1} & \qw & \qw & \qw & \qw & \qw & \qw & \qw & \qw & \targ & \targ & \qw & \qw & \qw & \qw & \qw & \qw & \qw & \qw & \qw & \qw & \qw & \qw & \qw & \qw & \qw & \qw & \qw & \targ & \targ & \ctrl{1} & \qw & \qw & \qw & \qw & \qw & \qw & \qw & \qw & \targ & \targ & \qw & \qw & \qw & \qw & \qw & \qw & \qw & \qw & \qw & \qw & \qw & \qw & \qw & \qw & \qw & \qw & \qw & \targ & \targ & \ctrl{1} & \qw & \qw & \qw & \qw & \qw & \qw & \qw & \qw & \targ & \targ & \qw & \qw & \qw & \qw & \qw & \qw & \qw & \qw & \qw & \qw & \qw & \qw & \qw & \qw & \qw & \qw & \qw & \targ & \targ & \ctrl{1} & \qw & \qw & \qw & \qw & \qw & \qw & \qw & \qw & \targ & \targ & \qw & \qw & \qw & \qw & \qw & \qw & \qw & \qw & \qw & \qw & \qw & \qw & \qw & \qw & \qw & \qw\\
            \nghost{{kick} :  } & \lstick{{kick} :  } & \gate{\mathrm{X}} & \gate{\mathrm{H}} & \qw & \qw & \qw & \qw & \qw & \qw & \qw & \qw & \qw & \qw & \qw & \targ & \qw & \qw & \qw & \qw & \qw & \qw & \qw & \qw & \qw & \qw & \qw & \qw & \qw & \qw & \qw & \qw & \qw & \qw & \qw & \qw & \qw & \qw & \qw & \qw & \qw & \qw & \qw & \qw & \qw & \targ & \qw & \qw & \qw & \qw & \qw & \qw & \qw & \qw & \qw & \qw & \qw & \qw & \qw & \qw & \qw & \qw & \qw & \qw & \qw & \qw & \qw & \qw & \qw & \qw & \qw & \qw & \qw & \qw & \qw & \targ & \qw & \qw & \qw & \qw & \qw & \qw & \qw & \qw & \qw & \qw & \qw & \qw & \qw & \qw & \qw & \qw & \qw & \qw & \qw & \qw & \qw & \qw & \qw & \qw & \qw & \qw & \qw & \qw & \qw & \targ & \qw & \qw & \qw & \qw & \qw & \qw & \qw & \qw & \qw & \qw & \qw & \qw & \qw & \qw & \qw & \qw & \qw & \qw & \qw & \qw & \qw & \qw & \qw & \qw & \qw & \qw\\
            \nghost{\mathrm{{mes} :  }} & \lstick{\mathrm{{mes} :  }} & \lstick{/_{_{5}}} \cw & \cw & \cw & \cw & \cw & \cw & \cw & \cw & \cw & \cw & \cw & \cw & \cw & \cw & \cw & \cw & \cw & \cw & \cw & \cw & \cw & \cw & \cw & \cw & \cw & \cw & \cw & \cw & \cw & \cw & \cw & \cw & \cw & \cw & \cw & \cw & \cw & \cw & \cw & \cw & \cw & \cw & \cw & \cw & \cw & \cw & \cw & \cw & \cw & \cw & \cw & \cw & \cw & \cw & \cw & \cw & \cw & \cw & \cw & \cw & \cw & \cw & \cw & \cw & \cw & \cw & \cw & \cw & \cw & \cw & \cw & \cw & \cw & \cw & \cw & \cw & \cw & \cw & \cw & \cw & \cw & \cw & \cw & \cw & \cw & \cw & \cw & \cw & \cw & \cw & \cw & \cw & \cw & \cw & \cw & \cw & \cw & \cw & \cw & \cw & \cw & \cw & \cw & \cw & \cw & \cw & \cw & \cw & \cw & \cw & \cw & \cw & \cw & \cw & \cw & \cw & \cw & \cw & \cw & \cw & \cw & \cw & \cw & \dstick{_{_{\hspace{0.0em}0}}} \cw \ar @{<=} [-12,0] & \dstick{_{_{\hspace{0.0em}1}}} \cw \ar @{<=} [-11,0] & \dstick{_{_{\hspace{0.0em}2}}} \cw \ar @{<=} [-10,0] & \dstick{_{_{\hspace{0.0em}3}}} \cw \ar @{<=} [-9,0] & \dstick{_{_{\hspace{0.0em}4}}} \cw \ar @{<=} [-8,0] & \cw & \cw\\
            \\ }}
    \]
    \caption{Circuit de résolution du sudoku \ref{fig:sudoku}}
    \label{fig:grov-sudoku}
\end{sidewaysfigure}
Ce qui nous donne après simulation des résultats presque parfaits.
\begin{figure}[H]
    \centering
    \import{images/algo/grover}{sudoku_result.tex}
    \caption{Résultat de l'algorithme de Grover pour le sudoku \ref{fig:sudoku}}
    \label{fig:grov-sudoku-sol}
\end{figure}
En effet, selon les conventions que nous avons établies, la liste 01110 correspond à la
solution attendue 23332.